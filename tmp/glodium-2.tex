
\chapter{Data visualization}

L'enorme quantità di dati -- small data, big data o smart data -- che
oggi si offrono alla continua fruizione degli utenti del Web e che li
distolgono dal secernere i dati significativi da quelli superflui, trova
una pratica consultazione nella rappresentazione grafica, infografica,
di liste o mappe che trasforma suddetti dati in `parlanti'. L'insieme
delle tecniche atte a rappresentare i dati in maniera interattiva è
detto \emph{Data} \emph{Visualization}.

L'esigenza di semplificare mediante la rappresentazione visiva un
quantitativo di informazioni altrimenti caotica e disorganizzata è di
antica datazione. La sostituzione del dato scritto con quello visivo ha
sì semplificato la ricezione delle informazioni, ma ha anche consentito
una loro più rapida elaborazione; ha aumentato la trasparenza della
comunicazione; ha fornito innovativi punti di vista grazie a nuovi input
di analisi e ha consentito una più svelta risoluzione delle
problematiche.

Coniata da William Fetter per la prima volta nel 1960, l'espressione
\emph{computer} \emph{graphics} inaugura la \emph{Data Visualization}
digitale che si è evoluta negli anni esponenzialmente. I primi progetti
di rappresentazione digitale nell'ambito delle Digital Humanities --
come l'analisi testuale fatta da John Burrow dei versi del XVII e del
XVIII secolo -- utilizzavano la rappresentazione grafica per
interpretare la mole di dati disponibili all'indagine umanistica. Un
progetto di \emph{Data Visualization} attualmente in vigore è il
\emph{Mapping the Republic of Letters}, che utilizza rappresentazioni
grafiche per aiutare gli studiosi ad analizzare oltre 55.000 lettere e
documenti archiviati digitalmente nel proprio database
(\url{http://republicofletters.stanford.edu/index.html}).

Tra i principali tool delle \emph{Data Visualization} si annoverano:

\begin{enumerate}
\def\labelenumi{\arabic{enumi}.}
\item
  Tableau: (a pagamento) in grado di produrre grafici e visualizzazioni
  interattive avanzate;
\item
  Infogram: utile per la creazione di infografiche, semplice e veloce da
  utilizzare;
\item
  Carto: adatto alla pubblicazione di mappe online;
\item
  Google Chart Tools: che consente la creazione di un'ampia gamma di
  grafici~embeddabili all'interno dei siti web.
\end{enumerate}

L'efficacia delle \emph{Data Visualization} trova conferma, oltre che in
quello scientifico, anche nel settore delle DH. La possibilità di
individuare connessioni, che normalmente non potrebbero essere viste, ha
infatti incentivato l'indagine pioneristica in questo campo, per
consentire anche ai ricercatori umanistici di servirsi di nuovi mezzi
all'avanguardia. Il TACC (\emph{Texas Advanced Computing Center}), ad
esempio, si propone la creazione di software grafici specifici per gli
studi umanistici e promuove la combinazione di testo, video, grafica 3D,
animazione, audio etc. per rimediare, in questo modo, alla scarsità di
mezzi di cui dispongono gli umanisti nel campo grafico
(\url{https://www.tacc.utexas.edu/research-development/tacc-projects/a-thousand-words}).

\section*{Bibliografia Consigliata}
{\parindent0pt 
Fischetti T. \emph{et al}., \emph{Data Analysis and Visualization},
Birmingham, Packt, 2016

Murray S., \emph{Interactive Data Visualization for the Web}, Sebastopol
(CA), O'Reilly Media, 2013

Stephen A. T., \emph{Data Visualization with Javascript}, USA, No Starch
Press, 2015

Telea A. C., \emph{Data Visualization. Principles and Practice}, Boca
Raton (FL), CRC, 2015
}

\section*{Sitografia}
{\parindent0pt 
*\emph{Data Visualization} (\emph{s.v}.), in \emph{Wikipedia},
\url{https://en.wikipedia.org/wiki/Data_visualization}

\emph{Mapping the Republic of Letters,
\url{http://republicofletters.stanford.edu/index.html}
Minguzzi G., \emph{Data Visualization: quando i numeri raccontano le storie},
\url{https://www.merita.biz/data-visualization/}

Minto P., \emph{Data Visualization, una storia/1},
\url{https://www.rivistastudio.com/data-visualization-una-storia/}

Minto P., \emph{Data Visualization, una storia/2},
\url{https://www.rivistastudio.com/data-visualization-2/}

\emph{TACC} - \emph{Texas Advanced Computing Center},
\url{https://www.tacc.utexas.edu/research-development/tacc-projects/a-thousand-words}

Zambon M., \emph{Glossario: Data Visualization},
\url{https://www.tagmanageritalia.it/glossario-data-visualization/\#gref}

Zepel T., \emph{Visualization as Digital Humanities?},
\url{https://www.hastac.org/blogs/tzepel/2013/05/02/visualization-digital-humanities}
}

\hrulefill 

{[}\emph{Alessandra Di Meglio}{]}


