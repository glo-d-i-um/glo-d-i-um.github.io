\chapter{Introduzione}
\section*{Le grandi quantità di dati fanno paura a molte discipline:
informatica, economia, matematica, statistica, filosofia. Perché non
dovrebbero terrorizzare anche noi umanisti?}

Ci avviciniamo con cautela a un futuro pieno di informazioni
immediatamente disponibili, lasciandoci sempre più alle spalle un
passato -- che idealizziamo, senza troppa memoria storica -- in cui i
dati a disposizione erano davvero pochi, chiusi per di più nel cassetto
di chi decideva di volta in volta come usarli e se concederne
l'usufrutto.

Le paure si affrontano affrontando di volta in volta le situazioni che
generano in noi questo senso di perdizione: non certo evitandole;
qualsiasi esperto in psiche umana potrà comodamente dirci, in caso
contrario, che non c'è niente di peggio dai \emph{loop} derivati dalla
paura della paura. Sembra proprio che la diffidenza verso l'informatica
umanistica sia uno di questi \emph{loop}: temiamo i dati, la loro
sconvolgente capacità di confonderci, e dunque evitiamo di cimentarci
con essi, rinunciando così sin troppe volte alle sconvolgenti
opportunità di conoscere e migliorare che essi ci offrono. È facile
gestire poche informazioni (e coronarsi maestri d'esse), ma è davvero
una blasfemia ermeneutica rinunciare ad aumentare la conoscenza generale
di una disciplina per paura di non riuscire a gestirne numero maggiore.

Eppure non abbiamo bisogno dell'informatica umanistica solo per
orientarci nella selva dei \emph{big data} che le nuove tecnologie --
\emph{obtorto collo} -- ci offrono: ne abbiamo anche necessità per
andare oltre i difetti convenzionali dell'isolazionismo disciplinare. I
vari settori scientifico-disciplinari sono -- nella sezione del sapere
che convenzionalmente chiamiamo umanistica -- sin troppo isolati gli uni
dagli altri, forse perché è meno sentito tra gli umanisti il concetto di
squadra (che rivive -- ad esempio -- in un laboratorio scientifico).
Tramite l'informatica sarebbe, invece, possibile creare piattaforme
collaborative in cui ognuno è chiamato a svolgere una mansione,
nell'ottica della ricerca generale; ciascuno secondo le proprie
competenze e abilità di partenza, che mai e poi mai dovranno essere
snaturate da ingenui principi di ``tuttologia''.

Orientarsi tra i \emph{big data}, gestire più informazioni
contemporaneamente, creare filtri sempre più precisi per i vari
risultati delle ricerche, offrire piattaforme e modelli di
collaborazione ai ricercatori: a fronte di tutti questi vantaggi, sembra
ancora assurdo che manipoli di intellettuali duri e puri continui a
tuonare contro la tecnologia applicata alla scienze umane; come se --
tra parentesi -- le scienze umane non fossero pioniere stesse del
concetto di tecnologia. L'informatica è del resto bastata sul concetto
stesso di grammatica, dunque quanto può essere impreciso -- per non dire
peggio -- un umanista che rifugge una disciplina pronipote di una
categoria logica di sua stessa competenza?!

Bisogna altresì spendere almeno un paio di parole sui profondi
miglioramenti che l'informatica umanistica potrebbe offrire al concetto
stesso di didattica. Le stesse piattaforme utili per i ricercatori con
qualche modifica potrebbero essere utili per mantenere costanti rapporti
tra studenti e docenti, per condividere sul momento materiali di studio,
per offrire l'interazione necessaria perché l'istruzione abbia i suoi
maggiori frutti, supponendo che il modello didattico più valido sia
ancora quello basato su domanda/risposta. Anche da questo punto di vista
si assiste, però, a una fiera dell'oscena cecità intellettuale, che
culminano di solito in qualche crociata per l'abolizione dell'uso degli
\emph{smartphone} nelle scuole, come se prima del telefonino gli
studenti non sapessero procacciarsi altrettante distrazioni dal loro
percorso.

Queste sono poche considerazioni introduttive, che intendono offrire --
a mo' di volo d'aquila -- una panoramica generale sui possibili vantaggi
di non chiudersi a riccio di fronte all'informatica umanistica. Non si
vuole entrare nel particolare, ad esempio nel confronto tra i commenti
ai testi editi in era pre-informatica coi commenti stilati con l'ausilio
di \emph{corpora} elettronici: sarebbe addirittura imbarazzante
continuare a lodare la precisione dei nuovi strumenti. Certo, ci
vogliono ricercatori e studiosi capaci di usare le nuove tecnologie
senza mai considerarle una scorciatoia. I rovesci della medaglia
esistono, del resto, da quando noi umani ci siamo resi conto che nel
conio il minor numero di facce è due (pur supponendo che una delle due
sia completamente vuota).

Linneo -- grande profeta della classificazione tassonomica -- si
espresse eccellentemente dicendo \emph{nomina si nescis, perit et
cognitio rerum}. Ciò vale anche per l'informatica umanistica: non
conoscere i nomi dei concetti di cui si avvale equivale a fare una
grandissima confusione.

Per concludere davvero: se l'informatica umanistica può essere utile,
può essere ancor più utile avere a disposizione uno strumento di
orientamento interno che sappia definire ogni istanza al suo interno, in
modo da renderla un campo funzionante e funzionale a qualsiasi prospetto
di studio o ricerca in ambito umanistico. Speriamo che il GloDIUm possa
dare in tal senso una consistente mano.

{[}\emph{Antonello Fabio Caterino}{]}