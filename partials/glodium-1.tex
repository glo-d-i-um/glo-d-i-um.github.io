

\chapter{Archivio}

Il termine \emph{archivio} può avere numerose accezioni: rimandare a
un'istituzione; a un'organizzazione; ad un luogo che ospita uno o più
complessi documentali; al materiale documentale prodotto da uno
specifico soggetto (magistrature, organi e uffici centrali e periferici
dello Stato; enti pubblici; istituzioni private, famiglie o persone).

Focalizzando l'attenzione sull'ultima definizione, l'\emph{archivio} è
«il complesso dei documenti prodotti e acquisiti da un soggetto
nell'esercizio delle sue attività». La natura del soggetto produttore si
riflette nella tipologia d'\emph{archivio}, caratterizzata poi da
fattori quali l'ordine interno, la rilevanza giuridica, le modalità di
formazione e sedimentazione.

Come scrive Monica Grossi: «preme sottolineare come elemento distintivo
di qualunque archivio la sua natura organica, il suo essere costituito
da più oggetti (i documenti) legati tra loro da relazioni complesse
(espresse nel vincolo archivistico che lega le singole parti al tutto) e
fondamentali per la piena comprensione di questo bene culturale
vulnerabile e multiforme, crocevia di interessi pratici, giuridici e
storici di una comunità eterogenea e diacronica».

Il trattamento e la cura degli archivi non possono inoltre prescindere
dalle recenti modalità di produzione documentaria che con l'introduzione
dei documenti digitali, che si è andata affiancando ai documenti
cosiddetti analogici, ha modificato dai punti di vista amministrativo,
normativo, tecnologico e culturale i rapporti tra supporto e testo del
documento, tra forma e contenuto, oltre a incidere sulle relazioni tra i
documenti e i loro autori e i processi di formazione, gestione e
conservazione degli archivi.

Indipendentemente dal formato e dal supporto, comunque, la nostra teoria
archivistica distingue tre fasi nel ciclo vitale dell'\emph{archivio}:
corrente, deposito e storico, cui corrispondono sul piano operativo ben
precise e distinte attività. Giova sottolineare~che questa distinzione
ha un valore meramente operativo ma non serba alcuna rilevanza sul piano
teorico: il valore storico, e quindi l'interesse culturale del documento
nasce sin dal formarsi degli archivi e coesiste sin dall'inizio con il
fine pubblico di garantire la certezza del diritto.

La fase corrente è quella in cui l'\emph{archivio} si forma e si
organizza attraverso la produzione e la ricezione dei documenti e la
formazione delle unità archivistiche. In questo stadio si perseguono
fini di efficienza, garanzia del diritto e trasparenza amministrativa ma
si impostano anche i presupposti per la tutela della futura memoria del
soggetto produttore. Nell'\emph{archivio} di deposito confluisce la
documentazione che ha terminato la sua fase attiva e che, pur non
risultando più necessaria all'espletamento dell'attività quotidiana,
conserva interesse dal punto di vista operativo e non è ancora pronta ad
essere destinata ad un uso prevalentemente culturale. L'\emph{archivio}
storico infine accoglie la documentazione relativa ad affari~esauriti da
almeno trent'anni. Tale documentazione, svuotata dalle esigenze di
servizio, è destinata a conservazione permanente e alla consultazione a
fini di studio e a fini culturali.

Questo collaudato modello è però oggetto di riflessione: le particolari
caratteristiche degli archivi digitali impongono infatti un ripensamento
della tradizionale articolazione del ciclo vitale e una nuova e
particolare attenzione da destinare alla progettazione dei sistemi
documentari e alla definizione dei requisiti descrittivi finalizzati
alla gestione, alla selezione e alla conservazione dei documenti.

L'\emph{archivio} inoltre è andato assumendo, sempre più nettamente,
nuovi significati: è `luogo' di per sé simbolico di un passato
collettivo, insieme di documenti-fonte per la riappropriazione da parte
di singoli cittadini di tratti della loro storia individuale o per la
rivendicazione di diritti alla trasparenza, all'informazione, alla
tutela della privacy.

Gli stessi archivisti hanno sentito la necessità di rendere sempre
maggiormente accessibile il patrimonio archivistico, superando la
cerchia degli addetti ai lavori e delle funzionalità di mera ricerca
scientifica. Tale esigenza ha trovato prime forme di esplicitazione
nelle mostre documentarie per poi allargarsi ai laboratori didattici, a
quelli di scrittura narrativa e a quelli di teatralizzazione
documentale, nonché all'istituzione di musei aggregati agli archivi.

\section*{Bibliografia consultata}
{\parindent0pt 
Duranti L., \emph{Il documento archivistico}, in Giuva L.- Guercio M.,
\emph{Archivistica. Teorie, metodi, pratiche}, Roma, Carocci, 2014, pp.
19-33

Grossi M., \emph{L'archivio in formazione}, in Giuva L.- Guercio M.,
\emph{Archivistica. Teorie, metodi, pratiche}, Roma, Carocci, 2014, pp.
35-52
}

\section*{Bibliografia Consigliata}
{\parindent0pt 
Carucci P. - Guercio M., \emph{Manuale di archivistica}, Roma, Carocci,
2008

Valacchi F., \emph{Diventare archivisti}, Milano, Editrice
Bibliografica, 2015
}

\section*{Sitografia}
{\parindent0pt 
*\emph{Archivio} (\emph{s.v}.), in \emph{DGA. Direzione Generale
Archivi},
\url{http://www.archivi.beniculturali.it/index.php/abc-degli-archivi/glossario}

\emph{*Archivio} (\emph{s}.\emph{v}.), in \emph{Lombardia Beni
Culturali. Glossario},
\url{http://www.lombardiabeniculturali.it/archivi/glossario/}

\emph{ICA. International Council on Archives},
\url{https://www.ica.org/en}
}

\hrulefill 

{[}\emph{Concetta Damiani}{]}

\chapter{ASCII}

(ingl. Acronimo per \emph{American Standard Code for Information
Interchange})

È il primo standard, sviluppato nel 1968, che è ancora oggi l'unica
pagina di caratteri che viene interpretata e trasmessa univocamente su
qualsiasi sistema informatico. Il suo funzionamento è pressoché
semplice, \emph{ASCII} codifica ogni carattere in codice binario in modo
da permettere al computer di comprendere ed immagazzinare quelle
informazioni.

Alla sua nascita si basava su un sistema a 7 bit, ciò vuol dire che era
in grado di codificare 27=128 caratteri: da 0 a 32 e il 127 non sono
stampabili, in quanto rappresentano dei caratteri di controllo che
permettono di compiere alcune azioni, come l'andare a capo; da 65 a 90
rappresentano le maiuscole; da 97 a 122 sono le minuscole.~

Nonostante la sua effettiva utilità \emph{ASCII} si rivelò insufficiente
per la circolazione di informazioni in un ambiente condiviso più ampio
del sistema nazionale statunitense. Si cercò, quindi, di creare un
sistema di codifica, chiamato `\emph{ASCII} esteso', a 8 bit in modo da
ampliare la gamma di caratteri codificabili (28=256). Nemmeno l'aggiunta
di un bit nella codifica riuscì a risolvere i problemi causati da un
sistema di scambio dell'informazione sempre più complesso.

Per questi motivi, nel 1991, venne sviluppato un nuovo sistema di
codifica chiamato Unicode, il cui sistema non è esente da difetti, ma
resta comunque lo standard universale con cui bisogna fare i conti nel
momento in cui ci si approccia all'informatica e all'editoria digitale.

\section*{Bibliografia consultata}
{\parindent0pt 
Kreines D. C., \emph{Oracle SQL: The Essential Reference}, USA, O'Reilly
Media, 2000, p. 162 

Tissoni F., \emph{Lineamenti di Editoria Multimediale}, Milano, Edizioni
Unicopli, 2009, p. 61 \par
}


\hrulefill 

{[}\emph{Alessia Marini}{]}



\chapter{Beta code}

Con l'espressione \emph{Beta code} si fa riferimento ad un sistema di
codifica che usa caratteri ASCII, ideato per la trascrizione di testi
greci su piattaforme e browser. Si tratta di un pre-unicode standard e
non va confuso con una semplice `romanizzazione' dell'alfabeto greco.
Per riprendere l'ottima definizione proposta da Fusi (2017, p. 24), il
\emph{Beta code} è un esempio di metacodifica, «che sovrappone ad uno
standard di codifica testuale un ulteriore livello di interpretazione»;
esso «con tutto il necessario apparato di simboli specialistici e con
una essenziale formattazione tipografica, rappresenta un modo per
superare le limitazioni del repertorio di ASCII senza definire una
codifica numerica completamente arbitraria» (Fusi, \emph{ibid}.).~

La prima convenzione di codifica e formattazione dei caratteri greci fu
ideata da David W. Packard negli anni '70 e fu definita come `Alpha
code', nome che successivamente fu mutato in \emph{Beta code}. Il
\emph{Beta code} fu adottato, nel 1981 dal TLG e -- benché ritenuto da
alcuni ormai obsoleto (cfr. \emph{e.g.} Borgna-Musso 2017, p. 72, n. 5;
ma, in merito ai limiti del \emph{Beta code}, si rimanda anche a Fusi
2017, p. 28) -- è utilizzato ancora oggi nell'àmbito di diversi
progetti, quali il \emph{Perseus Project}, il \emph{Packard Humanities
Institute}, il \emph{Duke Databank of Documentary Papyri et alii}, oltre
il summenzionato TLG, che dispone di un'apposita `sezione' contenente il
\emph{Beta code manual}, nonché una comoda \emph{Quick Reference Guide}
scaricabile in formato pdf.

Nel \emph{Beta code} le lettere greche sono rappresentate dai soli
caratteri maiuscoli di ASCII; ove possibile, vi è una corrispondenza tra
grafemi greci e grafemi romani, con l'ovvia eccezione delle lettere eta,
theta, xi, psi, omega e digamma (che corrispondono rispettivamente ai
caratteri H, Q, C, Y, W e V). Alcuni caratteri sono re-interpretati come
segni diacritici: lo spirito dolce corrisponde a \textbf{)}, quello
aspro a \textbf{(}; l'accento acuto corrisponde a \textbf{/}, l'accento
grave a \textbf{\textbackslash{}}, l'accento circonflesso a =; la iota
sottoscritta corrisponde al segno \textbar{}.

È ormai divenuta topica l'esemplificazione dell'uso del \emph{Beta code}
attraverso la beta-coded version dei primi versi dell'\emph{Iliade}:

*MH=NIN A)/EIDE, QEA/, *PHLHI+A/DEW *)AXILH=OS\\
OU)LOME/NHN, H(\textbackslash{} MURI/' *)AXAIOI=S A)/LGE' E)/QHKE

Il \emph{Beta code} può anche essere facilmente convertito in altri
formati (ad esempio nel formato Unicode), anche attraverso estensioni
gratuite ed open source (cfr. \emph{Greek Beta code converter}).

\section*{Bibliografia consultata}
{\parindent0pt 
Borgna A.- Musso S., \emph{Le sfide di una biblioteca digitale del
latino tardoantico. Struttura, canone e questioni aperte di codifica e
visualizzazione}, in P. Mastandrea (a cura di), \emph{Strumenti digitali
e collaborativi per le Scienze dell'Antichità}, Venezia, Edizioni Ca'
Foscari, 2017, pp. 69-94,
\url{http://edizionicafoscari.unive.it/media/pdf/books/978-88-6969-183-6/978-88-6969-183-6_IhKmUbC.pdf}

Fusi D., \emph{Tecnologie informatiche per l'umanista digitale}, Roma,
Nuova Cultura, 2017, pp. 24-29
}

\section*{Sitografia}
{\parindent0pt 
\emph{ASCII} (\emph{s.v.}), in \emph{GloDIUM},
\url{https://fri.hypotheses.org/glossario-di-informatica-umanistica}

Battaglino G., \emph{Note minime sul} TLG\emph{: brevi cenni sulle
`origini' del} TLG \emph{e piccolo} vademecum, in \emph{FRI} --
\emph{Filologia Risorse Informatiche} \emph{(Carnet de recherche and
online journal} -- \emph{Italian Digital Humanities} ,
\url{https://fri.hypotheses.org/1391}

\emph{Beta code} (\emph{s.v}.), in \emph{Wikipedia},
\url{https://en.wikipedia.org/wiki/Beta_Code}

\emph{Duke Databank of Documentary Papyri},
\url{https://papyri.info/docs/ddbdp}

\emph{Greek Beta code converter},
\url{https://chrome.google.com/webstore/detail/greek-beta-code-converter/abelaepcjjpjpkhpbbeggnijcccphpnp}

\emph{Packard Humanities Institute} (official website), 
\url{https://packhum.org/}

*\emph{Packard Humanities Institute} (\emph{s.v.}), in \emph{Wikipedia},
\url{https://en.wikipedia.org/wiki/Packard_Humanities_Institute}


\emph{Perseus Project} (official website), \textless{}
\url{http://www.perseus.tufts.edu/hopper/}

\emph{Perseus Project} (\emph{s.v.}), in \emph{Wikipedia}, \textless{}
\url{https://en.wikipedia.org/wiki/Perseus_Project}

\emph{The Beta code manual}, in \emph{TLG},
\url{http://www.tlg.uci.edu/encoding/}

\emph{Thesaurus Linguae Graeciae} (\emph{s.v.}), in \emph{Wikipedia},
\url{https://en.wikipedia.org/wiki/Thesaurus_Linguae_Graecae}

\emph{TLG Beta code quick Reference Guide},
\url{https://www.tlg.uci.edu/encoding/quickbeta.pdf}

\emph{TLG} (\emph{sub-corpus} del) in \emph{open access}: \emph{Canon of
Greek Authors and Works (Abridged TLG}),
\url{http://stephanus.tlg.uci.edu/Iris/canon/csearch.jsp}


\emph{TLG} (ultima versione digitale del \emph{TLG: Thesaurus Linguae
Graeciae. Digital Library}. Ed. Maria C. Pantelia, University of
California, Irvine), \textless{}
\url{http://stephanus.tlg.uci.edu/}

\emph{Unicode} (\emph{s.v.}), in \emph{Wikipedia},
\url{https://it.wikipedia.org/wiki/Unicode}
}


\hrulefill 

{[}\emph{Giovanna Battaglino}{]}

\chapter{Biblioteca digitale}

L'espressione \emph{biblioteca digitale} -- come sottolinea Faba Pérez
(1999) -- si configura come un «concetto ambiguo, come la maggior parte
dei concetti che si pongono in relazione con la tecnologia e con le
informazioni». A lungo, il concetto di \emph{biblioteca digitale} è
stato connesso a quello di `biblioteca virtuale' (cfr. Basili-Pettenati
1994).

Ad oggi, non esiste una definizione univoca, giacché molteplici sono gli
approccî, muovendo dai quali si è provato a definire una
\emph{biblioteca digitale}; fra essi è possibile annoverare, accanto ad
un approccio prettamente informatico (che valorizza la dimensione della
digitalizzazione), l'approccio seguìto dalla biblioteconomia (che
valorizza quanto attiene alla fruizione ed alla organizzazione
documentaria), nonché numerosi approcci di ricerca, sovente legati a
specifici progetti di digitalizzazione.

La prima definizione è stata elaborata dalla Borgman (1993), che
definisce la biblioteca digitale come una `struttura', nella quale si
combinano ed interagiscono un `servizio', una `architettura di rete', un
insieme di `risorse informative' -- costituite da banche-dati testuali e
numeriche, immagini statiche o dinamiche --, nonché un insieme di
strumenti per localizzare e reperire informazioni. La studiosa, in un
successivo contributo (Borgman 1999; ma cfr. anche Borgman 2000 e 2003),
riprende e precisa tale definizione, descrivendo la \emph{biblioteca
digitale} sia come «estensione e potenziamento dei sistemi di
conservazione delle informazioni» (in quanto `set' di risorse
elettroniche e competenze tecniche, finalizzate alla creazione, alla
ricerca e all'uso delle informazioni), sia come «estensione e
potenziamento di tutti i luoghi atti a conservare, organizzare e
diffondere informazioni», quali musei, biblioteche, archivi, scuole,
laboratori. È evidente che la definizione di \emph{biblioteca digitale}
rechi con sé, da un punto di vista concettuale, sia un aspetto
informativo-digitale/strumentale che un aspetto più propriamente
informativo-sociale.

Tra le prime definizioni va annoverata quella elaborata dalla Digital
Library Federation (DLF\emph{,} 1998), ancora chiaramente orientata su
una dimensione biblioteconomica: «le biblioteche digitali sono
organizzazioni che forniscono risorse, compreso il personale
specializzato, per selezionare, organizzare, interpretare, distribuire,
preservare l'integrità delle collezioni digitali, assicurandone la
persistenza nel tempo, sì che esse possano essere prontamente ed
economicamente accessibili per essere usate da una specifica comunità o
da una serie di comunità». Tale definizione è stata successivamente
(DLF, 2014) ripresa ed ampliata.

Le successive formulazioni definitorie -- quali quella di
Oppenheim-Smithson (1999) e di Arms (2000) --, al contrario, puntano
soprattutto alla valorizzazione della dimensione informativa. Pertanto,
la \emph{biblioteca digitale} è concepita sostanzialmente come `sistema
informativo', le cui `risorse', disponibili in formato digitale, possono
essere acquisite, archiviate, conservate/preservate, recuperate,
attraverso il ricorso alle `tecnologie digitali'.

Una prima definizione più articolata è quella proposta da
Marchionini-Fox (1999), i quali affermano che «il lavoro di una
biblioteca digitale si realizza in uno `spazio' determinato dalla
convergenza di quattro dimensioni: comunità (di utenti), tecnologia,
servizi e contenuti». La dimensione comunitaria poggia sul bisogno
tipicamente umano di cercare informazioni per conoscere e per
comunicare; la dimensione tecnologica fa riferimento ai progressi
tecnologici, che hanno trasformato anche il modo di conservare e
condividere le informazioni; per quanto concerne i servizi, i due
studiosi fanno riferimento alla facilitazione nel reperire informazioni,
anche attraverso l'aiuto in linea; infine, la dimensione contenutistica
consta dei documenti cercati, che possono presentarsi in varie
forme/media.

Anche in Italia il concetto di \emph{biblioteca digitale} (che ha
cominciato ad affermarsi non prima della fine degli anni '90) comincia a
divenire noto soprattutto in relazione alla sua dimensione
`strutturale': in Italia, il sintagma è stato introdotto da Malinconico
(1998), il quale parla di «nuova infrastruttura dell'informazione».

Successivamente anche il dibattito italiano relativo all'ontologia di
una biblioteca digitale si concentra sulla dimensione informativa, come
emerge dalla definizione offerta da Salarelli-Tammaro (2000): «la
biblioteca digitale è uno spazio informativo in cui le collezioni
digitali, i servizi di accesso e le persone interagiscono a supporto del
ciclo di creazione, preservazione, uso del documento digitale».
Naturalmente, tale `spazio' è caratterizzato da una specifica
organizzazione, funzionale alla dimensione informativa, come emerge
dall'ampia definizione elaborata da Ciotti-Roncaglia (2002), che
descrivono la \emph{biblioteca digitale} come una «collezione di
documenti digitali strutturati (sia prodotti mediante digitalizzazione
di originali materiali, sia realizzati \emph{ex novo}), dotati di
un'organizzazione complessiva coerente di natura semantica e tematica,
che si manifesta mediante un insieme di relazioni interdocumentali e
intradocumentali e mediante un adeguato apparato meta-informativo».

Nel 2005, la Tammaro propone una definizione -- per così dire --
`problematizzata': la studiosa sottolinea che perché una
\emph{biblioteca digitale} sia tale (e sia efficace) non basta la
conversione digitale della documentazione cartacea, ma occorre la
concomitanza di una serie di elementi, quali la presenza di un preciso
punto di accesso alle risorse digitali in rete, la definizione di una
chiara finalità di servizio (mission), la dichiarazione di una precisa
politica di sviluppo della collezione digitalizzata, l'adeguatezza
dell'organizzazione dell'informazione digitale resa disponibile e la
presenza di nuovi servizi d'accesso, che facilitino l'utenza nella fase
di ricerca.

Pur nella \emph{varietas} e nella problematicità delle definizioni
proposte, è, senza dubbio, possibile trarre alcune conclusioni, che
mirano non ad `imbrigliare' il concetto di \emph{biblioteca digitale},
ma ad enuclearne gli aspetti imprescindibili. È evidente che il concetto
di \emph{biblioteca digitale} non implichi una semplice trasposizione
digitale di una biblioteca reale. Essa è funzionale ad offrire ad una
comunità di utenti più o meno ampia, per il tramite delle tecnologie
informatiche, un contenuto culturale di validità scientifica. Ciò è
posto in atto grazie alla sinergia di `assi strategici', quali la
digitalizzazione, la tesaurizzazione (ed organizzazione) digitale e
l'accessibilità on-line.

\section*{Bibliografia consultata}
{\parindent0pt 
Borgman C. L., \emph{Digital libraries and the continuum of scholarly
communication}, in \emph{Journal of Documentation}, 56, 4, 2000, pp.
412-430

Borgman C. L., \emph{National electronic library report}, in Fox E. A.
(ed.) \emph{Sourcebook on Digital Libraries: report for the national
science foundation}, Blacksburg, Computer Science Department, 1993, pp.
126-147

Borgman C. L., \emph{The invisible library: paradox of the global
information infrastructure}, in \emph{Library trends}, 51, 4, 2003, p.
652

Borgman C. L., \emph{What are digital libraries? Competing visions}, in
\emph{Pergamon. Information Processing and Managment}, 35, 1999, pp.
227-243,
\url{https://pdfs.semanticscholar.org/d0e6/90b74b3b9513d9d1f97cf366e31a3920a4bf.pdf}

Faba Pérez C., \emph{Bibliotecas digitales: concepto y principales
proyectos}, in \emph{Investigación Bibliotecológica}, 13, 26, 1999, pp.
64-78

Marchionini G. - Fox E. A., \emph{Progress towards digital libraries:
augmentation through innovation}, in \emph{Information processing and
management}, 35, 3, 1999, pp. 219-225
\url{https://ac.els-cdn.com/S0306457398000582/1-s2.0-S0306457398000582-main.pdf?_tid=25a411f0-fe0b-11e7-9e69-00000aacb35d\&acdnat=1516471137_5866310621e56d197c1a9a06dce0917d}

Oppenheim C.-Smithson D., \emph{What is the hybrid library?}, in
\emph{Journal of Information Science}, 25, 2, 1999, pp. 97-112
}

\section*{Bibliografia Consigliata}
{\parindent0pt 
Arms M. Y., \emph{Digital Libraries}, Cambridge, MIT Press, 2000

Basili C. - Pettenati C., \emph{La biblioteca virtuale}, Milano,
Editrice Bibliografica, 1994

Ciotti F. - Roncaglia G., \emph{Il mondo digitale: introduzione ai nuovi
media}, Roma-Bari, Laterza, 2002

Malinconico S. M., \emph{Biblioteche digitali: prospettive e sviluppo},
in \emph{Bollettino AIB. Rivista Italiana di biblioteconomia e scienze
dell'informazione}, 38, 3, 1998
\url{http://bollettino.aib.it/article/view/8394/7498}

Salarelli, A. - Tammaro, A. M., \emph{Biblioteca digitale}, Milano,
Editrice Bibliografica, 2000

Tammaro A. M., \emph{Che cos'è una biblioteca digitale?}, in
\emph{Digitali Web. Rivista del Digitale nei Beni Culturali}, 1, 2005,
\url{http://digitalia.sbn.it/article/view/325/215}
}

\section*{Sitografia}
{\parindent0pt 
\emph{Biblioteca digitale} (\emph{s.v.}), in \emph{Treccani},
\url{http://www.treccani.it/enciclopedia/biblioteca-digitale_(Enciclopedia-Italiana)/}


\emph{Biblioteca virtuale} (\emph{s.v.}), in \emph{Treccani},
\url{http://www.treccani.it/enciclopedia/biblioteca-virtuale_(XXI-Secolo)/}


\emph{BNCF - Biblioteca Nazionale Centrale di Firenze},
\url{http://thes.bncf.firenze.sbn.it/termine.php?id=30683}

\emph{DLF - Digital Library Federation},
\url{https://old.diglib.org/about/dldefinition.htm}

\emph{DLF - Digital Library Federation},
\url{https://www.diglib.org/?s=definition+of+digital+librar}

\emph{ICCU}, \emph{Studio di fattibilità per la realizzazione della
Biblioteca Digitale}:
\url{http://www.iccu.sbn.it/upload/documenti/BDI-SDF.pdf}
(sezione prima);
\url{http://www.iccu.sbn.it/upload/documenti/BDI-SDF-Prog.pdf}
(sezione seconda)
}

\hrulefill 
 
{[}\emph{Giovanna Battaglino}{]}

\chapter{Big data}

Il concetto di \emph{Big Data} non è ancora oggetto di una definizione
univoca e gli ambiti di applicazione non sono ancora stati perimetrati
in maniera definitiva. Nell'Oxford Dictionary, si legge «Extremely large
data sets that may be analysed computationally to reveal patterns,
trends, and associations, especially relating to human behaviour and
interactions». Si tratta di una definizione esplicitamente riferita ai
dati, in senso quantitativo; l'insieme dei dati è definito `dataset'. In
letteratura, si usano espressioni quali `Big Dataset' o in alcuni casi
`Big Digital Objects'. I dati possono comprendere sia risorse
strutturate (es. i database, vd. database) che non strutturate (ess.
immagini, testi nei social network, vd. social network).

Quando tuttavia si fa riferimento ai \emph{Big Data}, si tende a
comprendere non solo la quantità, quanto anche l'insieme delle
tecnologie e delle procedure per la gestione di flussi enormi di
informazioni, a grandi velocità (\emph{Big Data Analytics}). Il
trattamento massivo dei dati infatti deve comprendere sia una quantità
di informazione importante, sia una ragionevole velocità di
processazione delle informazioni stesse. Solitamente, quando si fa
riferimento ai \emph{Big Data}, si considerano ordini di grandezze
enormi, comprendenti anche migliaia di terabyte (zettabyte).

Le finalità sono legate allo studio di comportamenti e tendenze umane in
una data comunità, a scopo descrittivo o predittivo. In questo senso, i
\emph{Big Data} hanno visto uno sviluppo significativo certamente nel
campo delle scienze sociali, ma si stanno espandendo in molti settori,
ad esempio medico e farmaceutico.

Nelle Digital Humanities è in piena attuazione un processo di
codificazione degli ambiti di ricerca; il quadro non è organico, proprio
perché i \emph{Big Data} richiedono un mutamento sia da un punto di
vista tecnologico -- in altre parole, sistemi capaci di organizzare
enormi quantità di informazione, agendo anche su migliaia di server --
sia dal punto di vista delle procedure. Consideriamo la nostra
conoscenza in merito a enormi dataset di tipo culturale: i milioni di
volumi digitalizzati da Google, le informazioni da Google Maps, i
miliardi di foto digitalizzate, le informazioni estraibili dai social
network utilizzabili. Per un singolo studioso è difficile delimitare
mentalmente quantitativi di informazione tanto elevati, per la maggior
parte sconosciuti e potenzialmente sempre in espansione. Le enormi sfide
epistemologiche che pongono si affrontano anche considerando i diversi
passaggi del processo di digitalizzazione degli oggetti, ma i metodi per
un loro efficace utilizzo sono ancora da esplorare, tanto dai data
scientist che dagli umanisti. Una disciplina simile non sarebbe
sbagliato considerarla a parte rispetto alle DH propriamente dette.

Esiste però un altro punto di vista per considerare i \emph{Big Data}
nell'ambito delle Digital Humanities, pensando alla rivalutazione delle
metodiche di tipo quantitativo. I \emph{Big Data} possono essere tali
anche in senso relativo, rispetto all'oggetto di studio e al fatto che
non possano essere collezionati e analizzati seguendo un approccio
tradizionale. I \emph{Big Data} lavorano inoltre in condizione di
interoperabilità o comunque sono collegati tra di loro.~Gli studi che
adottano questo approccio sono necessariamente data-driven (vd.
data-driven) e la loro diffusione ha portato anche a considerare
l'avvento di un nuovo paradigma scientifico, nel quale i dati saranno in
quantità tali da `parlare da sé', portando a obsolescenza concetti quali
`ipotesi', `teoria', `modello'. Un'ipotesi su cui in realtà converrebbe
essere estremamente cauti, per due importanti problemi da affrontare: le
difficoltà dietro alla selezione di materiali non strutturati e
all'interpretazione degli stessi. Per chiarire meglio: sicuramente la
linguistica è la disciplina che attualmente guarda ai \emph{Big Data}
con enorme interesse, proprio considerando la sua attitudine a lavorare
con i dati, rinforzata dal sempre maggiore utilizzo dei corpora (vd.
\emph{Corpora}). Eppure, in un'ideale scala di valori dei dati,
l'informazione estraibile da Google Books o dai social network e quella
da collezioni di testi densamente annotati e delimitati a un determinato
scopo si troverebbero sicuramente ai due estremi opposti.

In queste condizioni, la sfida con i \emph{Big Data} si giocherà
probabilmente su diversi fronti: da un lato, la realizzazione di
strumenti sempre più sofisticati per estrarre valore dai dati;
dall'altro, preparazione e conoscenza approfondita della tematica
oggetto di studio, per un'interpretazione consapevole anche dal punto di
vista qualitativo; inoltre i modi in cui creare interazione tra dataset
e utente, attraverso l'uso di diverse interfacce. Infine, non da ultimo,
una profonda riflessione interna sulle singole discipline, come la
filologia, che richiedono per loro statuto un approccio di tipo
qualitativo.

Esempio: MARTIN (\emph{Monitoring and Analysing Real-time Tweets in
Italian Natural language}) è uno strumento sviluppato dalla Fondazione
Bruno Kessler per estrarre informazione da un dataset non strutturato
come è il social network Twitter, comparare e analizzare il linguaggio
dei tweet, le co-occorenze e i concetti-chiave. Questo software è stato
presentato in una sua versione durante la sesta conferenza dell'AIUCD,
per un'indagine sul sentimento del dibattito su determinati eventi
storici su Twitter.

\hrulefill 

{[}\emph{Flavia Sciolette}{]}

\chapter{Browser}

Si tratta di un'applicazione (software) utilizzata per accedere e
usufruire dei siti Web. Tra i più comuni si trovano Microsoft Internet
Explorer, Google Chrome, Mozilla Firefox e Apple Safari. Le principali
funzioni del \emph{browser} sono: recuperare l'informazione per
l'utente, presentare l'informazione e accedere ad altre informazioni.
Gli oggetti presenti sul Web vengono infatti identificati attraverso un
URL che il \emph{browser} si occupa di identificare e rendere
disponibile alla lettura.

Per svolgere queste funzioni il \emph{browser} si occupa di convertire
il linguaggio HTML e gli altri formati che compongono una pagina web,
come PNG o JPEG. In questo modo il file di testo e le immagini che
compongono la pagina vengono visualizzati nella finestra del
\emph{browser}. Per eseguire questa funzione il \emph{browser} utilizza
un motore di rendering (engine), che interpreta le informazioni in
ingresso codificate secondo uno specifico formato e, elaborandone, ne
fornisce una rappresentazione grafica. Il modo in cui il \emph{browser}
interpreta e elabora graficamente i file HTML dipende dall'HTML stesso e
dalle specifiche CSS contenute nella risorsa web.

Per svolgere la sua terza funzione, la navigazione, il \emph{browser} si
appoggia sempre ad un motore di ricerca (search engine) che analizza un
insieme di dati e restituisce un indice dei contenuti disponibili
classificandoli in modo automatico. I \emph{Web} \emph{browser} entrano
in comunicazione con i \emph{Web server} principalmente tramite il
protocollo HTTP che gli permette di inviare informazioni e recuperare le
pagine che contengono tali informazioni.

I primi \emph{browser} come Netscape Navigator e Mosaic potevano
semplicemente convertire il file HTML e aggiungere segnalibri per
ricordare le pagine visitate. Le loro iterazioni più moderne supportano
diversi formati HTML (come HTML 5 o XHTML), sistemi di encrypting per
l'accesso a siti sicuro e funzioni Javascript. Lo sviluppo di dei
\emph{browser} ha permesso la creazione di siti più interattivi e con
sistemi di visualizzazione più avanzata da un punto di vista grafico.

\section*{Sitografia}
{\parindent0pt 
*\emph{Browser} (\emph{s}.\emph{v}.), in \emph{Tech Terms},
\url{https://techterms.com/}

\emph{How Browsers Work: Behind the scenes of modern web browsers},
\url{https://www.html5rocks.com/en/tutorials/internals/howbrowserswork/\#The_browser_main_functionality}

*\emph{Web Browser} (\emph{s}.\emph{v}.), in \emph{Intro to ITC},
\url{http://openbookproject.net/courses/intro2ict/web/web_browsers.html}

}

\hrulefill 

{[}\emph{Antonio Marson Franchini}{]}


\chapter{Concordanze }

(o \emph{Concordances})

Per \emph{concordanza verbale} (o spoglio lessicale) si intende il
repertorio alfabetico delle parole presenti in un'opera o nelle opere di
uno stesso autore (cfr. \emph{Dizionario di Italiano},
Sabatini-Coletti), accanto alle quali sono indicati i passi in cui esse
ricorrono adeguatamente concordate al contesto; si definisce, invece,
\emph{concordanza~reale} il repertorio di passi che rimandano a concetti
o argomenti ordinati tematicamente.

Se in passato le \emph{concordanze} necessitavano di schedature
lemmatiche eseguite a mano a notevoli costi di tempo e denaro, a partire
dagli anni '90, la disponibilità sempre più ampia di \emph{concordanze}
informatizzate ha reso possibile ricerche sistematiche di tipo lessicale
che destituissero l'ancestrale \emph{auctoritas} amanuense a favore dei
programmi informatici (come \emph{Concordance}) di facile e rapida
consultazione, capaci di ottenere in un tempo minimo le
\emph{concordanze} cercate. In un primo momento solo i testi sacri e
pochi profani (i classici greci e latini, Shakespeare o Joyce) furono
sottoposti alla selezione delle \emph{concordanze}. La leggenda,
infatti, narra che la prima \emph{concordanza} fu realizzata da Hughes
de St-Cher, erudito domenicano e membro della facoltà dell'Università di
Parigi, per la \emph{Vulgata}, con la collaborazione di cinquecento
monaci che terminarono la schedatura nel 1230 con il titolo di
\emph{Concordantiae} \emph{Sacrorum} \emph{Bibliorum}; lavoro ventennale
fu invece quello di John Bartelett e sua moglie per le
\emph{concordanze} dell'intera opera shakespeariana.

Un simile dispendio di tempo -- e di denaro -- durò fino a quando Padre
Roberto Busa pensò di indicizzare l'\emph{opera omnia} di Tommaso
d'Aquino con il supporto dell'International Business Machines (meglio
nota come IBM), sancendo l'atto di nascita delle Digital Humanities.

Stilare un repertorio di \emph{concordanze}, più che un mero processo
meccanico, quale certamente in parte è, è un \emph{modus}
\emph{cogitandi} e \emph{operandi} che stimola lo studioso a ricercare
nelle liste di frequenza dei termini contenuti in un'opera (o in opere)
di uno stesso autore dei segnali forti sulla sua psicologia, la poetica
e gli intenti.

Disporre di un repertorio di \emph{concordanze} assolve quindi a varie
funzioni: osservare gli usi di un termine, esaminare i contesti
(semantici, sintattici o testuali) in cui esso è inserito, analizzare la
regolarità con cui questo si accompagna ad altri termini nel suo cotesto
ma, soprattutto, contribuire all'esegesi dell'autore e della sua
produzione.

L'introduzione di una \emph{concordanza} elettronica, superiore a quella
stampata, ha migliorato \emph{sine} \emph{dubio} le prestazioni della
ricerca essenzialmente per due ragioni: per le sue dimensioni, dal
momento che può includere un'ampia quantità di testo e permettere allo
studioso di analizzare la \emph{concordanza} adeguatamente inserita in
un brano di ampio respiro, e perché può essere facilmente modificata,
mentre una \emph{concordanza} stampata è fissa. Spesso confuse con
l'\emph{index}, le \emph{concordanze} sono sì un repertorio alfabetico
di termini, ma, diversamente dall'indice -- che associa a ciascuna
parola il numero di pagina in cui è inserita -- sono affiancate
dall'indicazione dei passi (del capitolo e/o il versetto) in cui esse
ricorrono, nonché dal passo stesso, assolvendo tutt'al più alla funzione
di `indice contestualizzato'.

\section*{Bibliografia Consigliata}
{\parindent0pt 
Gigliozzi G., \emph{Introduzione all'uso del computer negli studi
letterari}, a cura di Fabio Ciotti, Milano, Bruno Mondadori, 2003

Schweickard W. (a cura di), \emph{Nuovi media e lessicografia storica.
Atti del colloquio in occasione del settantesimo compleanno di Max
Pfister}, Tübingen, Max Niemeyer Verlag, 2006

Stella F., \emph{Testi letterari e analisi digitale}, Roma, Carocci
editore, 2018
}
\section*{Sitografia}
{\parindent0pt 
*\emph{Concordanza} (\emph{s}.\emph{v}.), in \emph{Dizionario italiano
Sabatini Coletti},
\url{http://dizionari.corriere.it/dizionario_italiano/C/concordanza.shtml}

*\emph{Concordance} (\emph{s.v}.), in \emph{Wikipedia} (EN),
\url{https://en.wikipedia.org/wiki/Concordance_(publishing)}

\emph{The rise of the} \emph{machines} in \emph{Humanities}. \emph{The
Magazine of the National Endowment for the humanities},
\url{https://www.neh.gov/humanities/2013/julyaugust/feature/the-rise-the-machines}
}

\hrulefill 

{[}\emph{Alessandra Di Meglio}{]}


\chapter{Corpus}

(lat. plur. \emph{corpora})

Genericamente, un \emph{corpus} è un insieme di testi rappresentativi di
un certo ambito. In ambito digitale, il termine ha un significato
specifico: insieme di testi (orali, scritti o multimediali) disponibile
su supporto elettronico digitale -- online o da terminale -- divisi
singolarmente nelle loro unità minime (dette `token'), annotati secondo
le convenzioni di un linguaggio detto di markup (il più diffuso è XML,
per gli scopi delle DH nello standard TEI) e interrogabili attraverso
un'interfaccia pensata per introdurre stringhe di caratteri od operatori
logici. L'annotazione permette alla macchina di interpretare il testo
secondo le sue specificità testuali (es. prosa o verso). I
\emph{Corpora} sono uno strumento di studio della Linguistica cosiddetta
`dei \emph{corpora}'\emph{,} basata su testi giudicati autentici e
rappresentativi; tuttavia è interessante notare come, posizione ben
argomentata da Francesco Sabatini, la linguistica italiana abbia sempre
avuto un approccio basato su \emph{Corpora} testuali, pur se non
digitale, fin dai tempi del Dizionario della Crusca.

Esempio: CT -- \emph{Corpus Taurinense} (direzione: Manuel Barbera): un
\emph{corpus} di testi in fiorentino del secondo Duecento molto
annotato.

\hrulefill 

{[}\emph{Flavia Sciolette}{]}


\chapter{Crowdsourcing}

Mutuato dall'ambito economico, il termine \emph{crowdsourcing}, usato
per la prima volta da~Jeff Howe nell'articolo \emph{The Rise of
Crowdsourcing} apparso nel 2006 sulla rivista~\emph{Wired}, individua
una nuova organizzazione del lavoro che consente la formazione di una
rete capace di mettere in contatto numerosi volontari, hobbisti, o
esperti senza che siano fisicamente presenti nello stesso luogo.
Ciononostante questi possono partecipare all'ideazione e alla
realizzazione di progetti lavorativi mediante l'uso degli strumenti
informatici e proporre soluzioni innovative ad uno stesso problema. Il
termine \emph{crowdsourcing}, nato dalla fusione di `crowd' (folla) e
`(out)sourcing' (esternalizzazione dell'attività fuori dalla propria
impresa), in breve tempo, -- anche grazie alla pubblicazione nel 2008
del libro \emph{Crowdsourcing -- Why the power of the crowd the future
of business} di Howe -- è entrato a pieno titolo nella terminologia
corrente del settore finanziario,~estendendo il suo campo semantico alle
Digital Humanities.

Howe distingue quattro tipologie di \emph{crowdsourcing}:

\begin{enumerate}
\def\labelenumi{\arabic{enumi}.}
\item
  l'intelligenza collettiva~o~saggezza della folla, che si giova della
  conoscenza dei gruppi, in quanto superiore a quella dei singoli;
\item
  crowd-creation: si serve della creatività della~folla~per lo
  svolgimento delle attività;
\item
  crowd-voting: ordina le informazioni in base alle scelte della folla;
\item
  crowd-funding, che permette ai gruppi di raccogliere
  auto-finanziamenti.
\end{enumerate}

Henk van Ess~ha sostenuto che il \emph{crowdsourcing} consiste nella
possibilità, data ad esperti, di risolvere problemi da condividere
liberamente con chiunque; Estellés e González nel 2012 hanno definito il
\emph{crowdsourcing} una tipologia di attività online nella quale una
persona, istituzione o organizzazione, senza scopo di lucro, propone ad
un gruppo di individui, mediante un annuncio aperto, la realizzazione
libera e volontaria di un compito specifico. Il prodotto ottenuto dal
\emph{crowdsourcing} è, secondo alcuni, altamente specializzato e ad
alto contenuto tecnologico, pari a quello ottenibile con l'utilizzo di
professionisti pagati per eseguire la stessa prestazione, e in tempi
anche minori (Antonio Dini); secondo altri, esso non garantirebbe alcuna
professionalità né qualità, essendoci aperta partecipazione da parte di
chiunque, anche di chi non ha adeguate competenze. I vantaggi del
\emph{crowdsourcing} sono vari e vanno dai costi, limitatamente bassi,
alla possibilità di avvalersi di un nutrito numero di esperti (o
appassionati) interessati alla ricerca, mentre sono da considerarsi
degli svantaggi il rischio di fallimento, l'assenza di un contratto e/o
la difficoltà a collaborare con un gran numero di persone. Tra i primi
esempi di \emph{crowdsourcing} nell'ambito delle Digital Humanities c'è
il progetto \emph{Transcribe Bentham}, iniziato nel 2010, che consiste
nella digitalizzazione dei manoscritti di Jeremy Bentham, ma anche
\emph{Ancient Lives}, che trascrive i papiri greci, etc.

Il \emph{crowdsourcing} trova applicazione soprattutto nell'ambito
creativo e nel finanziamento di progetti (o crowfunding) che
diversamente rischierebbero di non essere attuati a causa del tempo e
delle esigue risorse economiche. \emph{Wikipedia}, ad esempio, è un
progetto di \emph{crowdsourcing} che si avvale di numerosi
professionisti sul territorio nazionale e che è, oggi, tra quelli di
maggior successo. Bruno Pellegrini, tra i più autorevoli esponenti del
\emph{crowdsourcing} europeo, ha realizzato l'\emph{Italian}
\emph{Crowdsourcing} \emph{Landscape}, una mappa che raccoglie i vari
progetti attivi in Italia, che chiunque può aggiornare mediante un'app
per iOS o Android. A questo si aggiungono numerosi altri progetti,
nazionali ed europei, che attestano l'importanza del
\emph{crowdsourcing}. Ad esempio, Greg Cane, direttore del \emph{Perseus
Project}, ha lanciato un appello agli studenti: \emph{Give Us Editors!},
invitandoli a collaborare alla digitalizzazione e alla modifica dei
libri scansionati avendo questi competenze adeguate a questo compito.
Tuttavia, come sostengono Hedges e Dunn: «in un tempo in cui il web sta
trasformando simultaneamente il modo in cui le persone collaborano e
comunicano, fondendo gli spazi che le comunità accademiche e non
accademiche occupano, non è mai stato più importante considerare il
ruolo che le comunità pubbliche -- collegate o meno -- giocano nella
ricerca accademica umanistica», giacché anche all'esterno delle
accademie la collaborazione finalizzata alla ricerca risulta essere di
fondamentale importanza per il suo progresso. A tal proposito, Chris
Blackwell e Tom Martin hanno sostenuto -- non a torto -- che gli
studenti universitari potrebbero -- e dovrebbero -- pubblicare online
articoli informativi capaci di accrescere il valore delle discipline
umanistiche e di migliorare la formazione e la qualità degli interventi
di chi è interessato.

Certo è che, dati gli enormi vantaggi economici e di tempo, il
\emph{crowdsourcing} rappresenta oggi il più diffuso approccio
lavorativo. Ampiamente adottato nel settore delle Digital Humanities
esso contribuisce quotidianamente al progresso della ricerca, alla
realizzazione di progetti mastodontici e al raggiungimento di risultati
altrimenti impensabili.

\section*{Bibliografia Consigliata}
{\parindent0pt 
Andro M., \emph{Digital Libraries and Crowdsourcing}, 5, London, Wiley,
2018

Brabham D. C., \emph{Crowdsourcing}, Cambridge, The MIT press Essential
Knowledge Serie, 2013

Estellés Arolas E.- González Ladrón-de-Guevara F., \emph{Towards an
integrated crowdsourcing definition}, in \emph{Journal of Information
Science}, 38, 2, 2012, pp. 189-200

Terras M., \emph{Crowdsourcing in the Digital Humanities,} in Schreibman
S.- Siemens R.- Unsworth J. (eds.), \emph{A New Companion to Digital
Humanities}, Chichester, Wiley-Blackwell, 2016, pp. 420-439 \url{http://discovery.ucl.ac.uk/1447269/1/MTerras_Crowdsourcing\%20in\%20Digital\%20Humanities_Final.pdf}
}

\section*{Sitografia}
{\parindent0pt 
*\emph{Crowdsourcing} (\emph{s.v}.), in \emph{Wikipedia},
\url{https://it.wikipedia.org/wiki/Crowdsourcing}

Frost D. R., \emph{Crowdsourcing, Undergraduates, and Digital Humanities
Projects},
\url{https://rebeccafrostdavis.wordpress.com/2012/09/03/crowdsourcing-undergraduates-and-digital-humanities-projects/}

Howe J., \emph{The rise of crowdsourcing},
\url{https://www.wired.com/2006/06/crowds/}

Mazzini E.\emph{, Il crowdsourcing tra necessità di coordinamento e
perdita di controllo}/Capitolo 1 -- \emph{Il Crowdsourcing},
\url{https://it.wikisource.org/wiki/Il_crowdsourcing_tra_necessità_di_coordinamento_e_perdita_di_controllo/Capitolo_1_–_Il_Crowdsourcing}

Pezzali M., \emph{Crowdsourcing: quando la rete\ldots{} trova la
soluzione}, in \emph{Il Sole 24 ore},
\url{https://www.ilsole24ore.com/art/SoleOnLine4/Economia\%20e\%20Lavoro/2009/02/crowdsourcing-rete-soluzione.shtml}

Van Ess H., \emph{Harvesting Knowledge. Success criteria and strategies
for crowdsourcing},
\url{https://www.slideshare.net/StifoPers/presentatie-henk-van-ess}
}

\hrulefill 

{[}\emph{Alessandra Di Meglio}{]}

\chapter{CSS}

(ingl. Acronimo per \emph{Cascading Style Sheets})

Rappresenta un linguaggio utilizzato per definire la formattazione di un
documento HTLM, XHTML o XML. Essendo uno standard patrocinato dal W3C,
le regole per il suo utilizzo (Recommendation), emanate nel 1996, sono
liberamente consultabili sul loro sito.

I \emph{CSS} basano la loro sintassi su una gerarchia ben definita di
marcatori, non modificabili o estendibili, che utilizzano per definire
la visualizzazione dei contenuti di un documento. L'utilizzo di questi
fogli di stile a cascata, si è resa necessaria sia per separare i
contenuti della pagina HTML dalle impostazioni sulla formattazione in
modo da ottenere un documento dalla codifica più snella e dalle
dimensioni più leggere. La separazione del foglio di stile dal
documento, avvenuta nel 1999 con la creazione di HTML4, rende tutta la
fase di programmazione del documento (cioè tutte le impostazioni circa
la visualizzazione della pagina, cosa va dove e come deve essere
visualizzato dal browser) nettamente più ordinata e di più facile
lettura sia dal programma che dagli autori stessi.

Ad oggi i \emph{CSS}, e soprattutto la loro separazione dal documento
stesso, risultano estremamente utili per definire separatamente le
visualizzazioni che una stessa pagina web può (e deve) assumere sui
diversi device: Computer, Tablet e Smartphone essendo dotati di schermi
dalle dimensioni e dalle possibilità differenti, hanno bisogno ognuno
della sua formattazione in modo da rendere l'interfaccia utente il più
rispondente possibile ai principi di chiarezza e usabilità, basilari per
qualsivoglia sito web.

Con un singolo foglio di stile è possibile:

\begin{enumerate}
\def\labelenumi{\arabic{enumi}.}
\item
  Impostare la posizione dei diversi contenuti interni alla pagina web
  (porzioni di testo, immagini, link esterni, video, banner animati,
  eccetera);
\item
  definire lo stile, il carattere, il colore e il font delle titolazioni
  (definite in HTML come \emph{\textless{}h1\textgreater{}
  \textless{}h2\textgreater{} \textless{}hn}) dei
  paragrafi e di tutte le peculiarità tipografiche interne al documento.
\end{enumerate}

\section*{Bibliografia Consigliata}
{\parindent0pt 
Duckett J., \emph{HTML e CSS. Progettare e costruire siti web. Con
Contenuto digitale per download e accesso on line}, Milano, Apogeo, 2017

Tissoni F., \emph{Lineamenti di editoria multimediale}, Milano, Edizioni
Unicolpi, 2009
}

\section*{Sitografia}
{\parindent0pt 
*\emph{W3CSS} (\emph{s.v}.), in \emph{w3schools.com},
\url{http://www.w3schools.com/}
}

\hrulefill 

{[}\emph{Alessia Marini}{]}
