
\chapter{Database}

(ingl. Acronimo: \emph{DB})

Sinonimi: `base di dati', `banca dati'; fr. \emph{base de données.}

Un \emph{database} è una collezione di dati omogenei per contenuto e
formato, organizzati secondo una determinata struttura, scelta da chi
sviluppa il progetto (es. il modello relazionale, uno dei più diffusi
nell'ambito della progettazione). Un \emph{DB} è solitamente
interrogabile da un'interfaccia visuale da schermo -- online o da
terminale -- pensata per introdurre stringhe di caratteri od operatori
logici: un linguaggio di interrogazione per \emph{DB} è detto `query
language' (linguaggio di ricerca). La classe di software per la
creazione e la gestione dei \emph{DB} è detta DBMS (\emph{Database
Management System}); uno dei più diffusi e maggiormente supportati è
l'open-source Mysql, rilasciato dalla Oracle.

Esempio: \emph{BEDT -- Bibliografia Elettronica dei Trovatori}
(Direzione Scientifica: Stefano Asperti, Direzione tecnica: Luca De
Nigro) una base di dati su modello relazionale dedicata alla poesia
provenzale.

\hrulefill 

{[}\emph{Flavia Sciolette}{]}

\chapter{Digital Dark Age}

L'espressione \emph{Digital Dark Age}, di conio alquanto recente, è
attestata per la prima volta nel 1997 in una relazione presentata nella
Conferenza dell'\emph{International Federation of Library Associations
and Institutions} (IFLA) ed è stata successivamente ripresa e
`valorizzata' nel Convegno intitolato \emph{Time and Bits} a cura della
\emph{Long Now Foundation} e del \emph{Getty Conservation Institute} nel
1998. L'espressione è parzialmente mutuata dalla storiografia, in seno
alla quale si ricorre all'espressione Dark Age per indicare un periodo
di assenza di documentazione scritta con riferimento al Medioevo
Ellenico o all'Alto Medioevo.

La definizione di \emph{Digital Dark Age} è legata alla dicotomia
concettuale `digital preservation' \textasciitilde{} `digital
obsolescence'. In particolare, il concetto di Medioevo Digitale si
configura come il possibile, nefasto futuro esito dell'obsolescenza
digitale, vale a dire l'indisponibilità ed inaccessibilità di dati in
supporto digitale, dovuta o al deterioramento dei supporti (hardware) o
alla troppo rapida evoluzione degli strumenti informatici atti a creare,
modificare e leggere i dati (software). In particolare, il
deterioramento dei supporti è dovuto al fatto che essi non hanno una
durata illimitata, perché i supporti magnetici (hard disk, pen drive)
possono risentire di campi elettromagnetici esterni, mentre i supporti
ottici (CD, DVD) possono essere danneggiati dalla luce o dalla
temperatura. Ad esso si affianca il problema della `usability': è
necessario che i digital data siano sempre in un formato accessibile,
aspetto questo di non facile gestione.

La realtà, apparentemente paradossale, risiede nel fatto che le
collezioni digitali facilitano l'accesso alle informazioni, ma non la
loro conservazione. Sono stati avviati programmi di ricerca, finalizzati
ad elaborare (analytic) tools -- come il \emph{cloud-based storage} -- e
strategie di preservazione delle informazioni conservate in formato
digitale -- strategie legate alla `interoperability' tra software e
servizi (SaaS).

\section*{Bibliografia Consigliata}
{\parindent0pt 
Howell A., \emph{Perfect One Day -- Digital The Next: Challenges in
Preserving Digital Information}, in \emph{Australian Academic \&
Research Libraries,} 31, 4, 2013, pp. 121-141
\url{https://www.tandfonline.com/doi/pdf/10.1080/00048623.2000.10755130}

Ross H. D., \emph{Preserving Digital Materials}, München, Rowman \&
Littlfield, 2005
}
\section*{Sitografia}
{\parindent0pt 
*\emph{Digital Dark Ages} (\emph{s.v}.), in \emph{LisWiki} ,
\url{https://liswiki.org/wiki/Digital_Dark_Ages}

\emph{IFLA - International Federation of Library Associations and
Institutions},
\url{https://www.ifla.org/}

Kuny T., \emph{A Digital Dark Ages? Challenges in the Preservation of
Electronic Information}, relazione presentata al \emph{Workshop:
Audiovisual and Multimedia joint with Preservation and Conservation,
Information Technology, Library Buildings and Equipment, and the PAC
Core Programme}, September 4, 1997,
\url{https://archive.ifla.org/IV/ifla63/63kuny1.pdf}

*\emph{Obsolescenza digitale} (\emph{s.v}.), in\emph{Treccani} ,
\url{http://www.treccani.it/enciclopedia/obsolescenza-digitale_(Lessico-del-XXI-Secolo)/}


*\emph{Obsolescenza digitale} (\emph{s.v}.), in \emph{Wikipedia},
\url{https://it.wikipedia.org/wiki/Obsolescenza_digitale}


Rusbridge C., \emph{Excuse me... some Digital Preservation Fallacies?},
in \emph{Web Magazine for Informations Professionals},
\url{http://www.ariadne.ac.uk/issue46/rusbridge/}

Shave L., \emph{It's a digital world. Records and information Management
Professionals. Guide to avoid Digital Dark Ages},
\url{http://rimpa.com.au/assets/2015/09/eBook-Avoiding-the-digital-dark-age.pdf}

}

\hrulefill 

{[}\emph{Giovanna Battaglino}{]}



\chapter{Digital Death}

Con il termine \emph{Digital Death} si intende solitamente l'insieme
delle questioni che riguardano, in primo luogo, i modi in cui è cambiato
il rapporto tra il singolo individuo e la morte a causa dello sviluppo
delle nuove tecnologie informatiche e mediatiche, a partire soprattutto
dalla diffusione popolare del Web. In secondo luogo, le conseguenze che
ne derivano per quanto concerne la costruzione della propria identità
personale e il suo legame con la memoria in seguito alla morte di sé o
di un altro individuo.

Tre i problemi specifici su cui si concentrano gli studiosi della
\emph{Digital Death}: 1) gli effetti che la morte di un singolo
individuo produce all'interno della realtà digitale e, quindi, nella
vita `reale' di chi soffre la perdita e il lutto; 2) gli effetti che la
perdita di oggetti e informazioni digitali personali producono
all'interno della realtà fisica di un singolo individuo. In altre
parole, le conseguenze di una perdita definitiva di informazioni,
fotografie, messaggi, quando si rompe il proprio computer senza aver
fatto un preventivo backup del materiale lì contenuto; 3) l'inedito
significato che assume il concetto di immortalità in relazione tanto al
singolo individuo quanto agli oggetti e alle informazioni digitali
personali.

Questi problemi implicano un'attenzione particolare, a livello
interdisciplinare, da parte degli studiosi per tutti i progetti digitali
che cercano di farci sopravvivere alla nostra morte quali `spettri
digitali' (Eter9, Eterni.me, LifeNaut, ecc.) e per le questioni inerenti
a concetti come `patrimonio digitale', `memoria digitale', `eredità
digitale', i quali prendono forma a partire dall'uso decennale dei
social network da parte degli utenti del Web.

\hrulefill 

{[}\emph{Davide Sisto}{]}



\chapter{DOI}

(\emph{Digital Object Identifier})

Definito anche identificatore persistente, il \emph{DOI} identifica un
oggetto digitale, i suoi dati e i metadati (autore, titolo, editore,
data di pubblicazione, ma anche numero del fascicolo di una rivista,
l'elenco degli articoli in essa contenuti, etc.), in maniera persistente
e secondo uno schema strutturato ed estensibile.

L'associazione del \emph{DOI} all'oggetto è totalmente univoca:
attraverso il \emph{DOI}, infatti, si accede alla pagina Internet
contenente l'oggetto e se l'oggetto viene spostato si sposta anche il
\emph{DOI} che lo identifica. Differentemente da ISBN e ISSN, esso è
immediatamente azionabile e si configura come siringa alfanumerica
divisa in prefisso e suffisso.

Dato un \emph{DOI} di questo tipo: 10.1491/napoli081995, il prefisso è
10.1491, di cui 10 identifica il \emph{DOI}, mentre 1491 indica il
registrante; napoli081995 è il suffisso che coincide con l'oggetto.

A occuparsi della registrazione dei \emph{DOI} c'è la \emph{multilingual
European DOI Registration Agency}, meglio nota come mEDRA,
l'\emph{Agenzia Europea di Registrazione del} \emph{DOI}, nominata
dall'IDF (\emph{International DOI Foundation},
\url{http://www.doi.org/}) il primo luglio 2003.

Il successo ottenuto da mEDRA e il forte interesse mostrato dagli
editori scientifici e professionali, universitari/scolastici, ma anche
dalle università e dai centri di ricerca, dalle biblioteche e dalle
pubbliche istituzioni, ha incentivato una più ampia diffusione del
\emph{DOI}, eliminando le barriere economiche, consentendone l'utilizzo
anche ai piccoli editori e incentivandone, in questo modo, il processo
di standardizzazione.

\section*{Bibliografia Consigliata}
{\parindent0pt 
ISO 26324\emph{,} \emph{International Standard. Information and
documentation -- Digital Object Identifier system}, Switzerland, ISO,
2012
}

\section*{Sitografia}
{\parindent0pt 
*\emph{DOI-Digital Object Identifier} (\emph{s.v}.), in
\emph{Wikipedia},
\url{https://it.wikipedia.org/wiki/Digital_object_identifier}

\emph{DOI. Un código esencial para citas bibliográficas y búsquedas
científicas,
\url{http://infobib.blogspot.com/2009/04/doi-un-codigo-esencial-para-citas.html}

\emph{mEDRA},
\url{https://www.medra.org/it/DOI.htm}

\emph{mEDRA -- multilingual DOI Registration Agency},
\url{http://www.aie.it/Portals/_default/Skede/Allegati/Skeda105-1466-2007.5.2/Medra\%20brochure\%20applicazioni\%20IT.pdf}

\emph{Servizi SBA: Informazioni sul DOI},
\url{http://biblioteche.unige.it/node/255}
}

\hrulefill 

{[}\emph{Alessandra Di Meglio}{]}

\chapter{DTD}

(ingl. Acronimo per \emph{Document Type Description})

La \emph{DTD} è un linguaggio per definire la grammatica che descrive la
composizione degli elementi che fanno parte di una certa classe di
documenti XML. È un retaggio del SGML, in quanto già se ne serviva per
strutturare formalmente e gerarchicamente tutti i componenti del
linguaggio.

La \emph{DTD} è un insieme di dichiarazioni che principalmente
determinano quali elementi devono e/o possono essere usati, quali
attributi possono essere aggiunti agli elementi e soprattutto quali
relazioni legano o separano i vari elementi. Le \emph{DTD} risultano,
quindi, molto utili per la comprensione e la validazione del documento
XML, per la creazione di fogli di stile e di interfacce dinamiche e per
l'interscambio di documenti. In parole povere altro non è che un
documento in cui vengono sia inseriti tutti gli elementi e gli attributi
che il codificatore potrà utilizzare nel documento XML, sia specificati
i tipi di rapporti che intercorrono tra di loro.

Ogni \emph{DTD} può essere, arbitrariamente, interna o esterna (in
questo caso può diventare un file autonomo e interscambiabile) al
documento XML, ma in entrambi i casi essa deve essere necessariamente
dichiarata nel documento stesso, subito dopo la dichiarazione XML.

\section*{Bibliografia Consigliata}
{\parindent0pt 
Ausiello G. \emph{et al}., \emph{Modelli e linguaggi dell'informatica},
Milano, McGraw-Hill, 1991

Ciotti F., \emph{Il testo e l'automa. Saggi di teoria e critica
computazionale dei testi letterari}, Roma, ARACNE Editrice s.r.l., 2007

Møller A. - Schwartzbach M.I. - Gaburri S. (a cura di),
\emph{Introduzione a XML}, Milano, Pearson, 2007

Tissoni F., \emph{Lineamenti di editoria multimediale}, Milano, Edizioni
Unicopli, 2009
}

\hrulefill 

{[}\emph{Alessia Marini}{]}


\chapter{EAD}

(ingl. Acronimo per \emph{Encoded Archival Description})

L'\emph{Encoded Archival Description} (EAD) è uno strumento per la
conversione e la pubblicazione in formato elettronico di strumenti di
ricerca archivistici prodotti originariamente su supporto cartaceo,
nonché per l'elaborazione e lo scambio di descrizioni archivistiche in
formato nativamente digitale. L'adozione di tecnologie
come~l'\emph{eXtensible Markup Language} (XML), volte alla conservazione
e alla comunicazione dei dati, indipendentemente da specifiche
piattaforme hardware e software, consente di garantire la persistenza
della struttura e del contenuto delle descrizioni, nonché la loro
accessibilità e validità nel tempo.

Lo standard prende forma nell'ambito del \emph{Berkeley Finding Aids
Project} (BFAP), avviato nel 1993 presso la Berkeley University della
California, con l'intento di sviluppare un modello non proprietario per
la codifica in formato digitale di strumenti di ricerca quali inventari,
elenchi sommari, indici utilizzando lo~\emph{Standard Generalised
Mark-up Language} (SGML)\textbf{.} La versione 1.0 di \emph{EAD} è stata
rilasciata nel 1998, accompagnata dalla pubblicazione della~\emph{Tag
library} e delle~\emph{Application guidelines}~sul sito ufficiale del
progetto, ospitato in quello della Library of Congress che era diventata
l'agenzia responsabile del mantenimento di \emph{EAD}.

Nel 2002, dopo un attento lavoro di revisione da parte della comunità
scientifica internazionale, è stata rilasciata una nuova versione del
modello che è stata allineata, in alcune sue componenti, all'edizione
del 2000 dello standard di descrizione archivistica
ISAD(G),~\emph{General International Standard Archival Description.}

In anni recenti, anche alla luce di importanti novità -- dal rilascio
\emph{dell'Encoded Archival Context - Corporate bodies, Persons, and
Families} (EAC-CPF), alla comparsa di collection management system, alle
nuove opportunità cui sembrano aprire i \emph{Linked Open Data} -- si è
sentita la necessità di un'ulteriore revisione dello standard
finalizzata ad accrescerne chiarezza e coerenza, a garantire una
maggiore interoperabilità e un generale miglioramento delle funzionalità
in ambienti internazionali e multilingue.

Il sito ufficiale dello standard rende disponibile il modello dati,
espresso come DTD, XML Schema, Relax NG, Schematron, unitamente alla
\emph{Tag Library} e al foglio di stile che dovrebbe garantire il
passaggio da \emph{EAD} 2002 a EAD3.

Nell'estate del 2015 la versione EAD3 è stata adottata come standard
dalla \emph{Society of American Archivists} (SAA).

\section*{Bibliografia Consigliata}
{\parindent0pt 
Carucci P. - Guercio M., \emph{Manuale di archivistica}, Roma, Carocci,
2008

Michetti G. (a cura di), \emph{EAD: Encoded archival description tag
library = Descrizione archivistica codificata dizionario dei marcatori},
Roma, ICCU, 2005

Pala F., \emph{Lo standard EAD3 per la codifica dei dati archivistici:
qualche novità e molte conferme}, in \emph{J-LIS}, 8, 3, 2017, pp.
148-176

Pitti D., \emph{Encoded Archival Description: The Development of an
Encoding Standard for Archival Finding Aids}, in \emph{The American
Archivist}, 60, 3, 1997, pp. 268-283
}

\section*{Sitografia}
{\parindent0pt 
\emph{EAD - Encoded Archival Description} (official site),
\url{http://www.loc.gov/ead/index.html}

\emph{ICAR - Istituto Centrale per gli Archivi},
\url{http://www.icar.beniculturali.it/index.php?id=52}
}

\hrulefill 

{[}\emph{Concetta Damiani}{]}



\chapter{Feed RSS}

(ingl. Acronimo per \emph{Rich Site Summary/ RDF Site Summary o Really
Simple Syndication})

Quando si parla di \emph{Feed RSS} ci si riferisce a un flusso, di
un'unità di informazioni formattata secondo regole decise
precedentemente per rendere interpretabili e interscambiabili una serie
di contenuti frequentemente aggiornati tra diverse applicazioni e
piattaforme. Si tratta di un metodo per fornire e distribuire i
contenuti sul Web attraverso un formato standard XML e, come si desume
dall'acronimo, spesso in un formato che assomiglia ad un sommario con
link che rimandano al sito web originale. Si tratta quindi di una
applicazione di XML che deve conformarsi a XML 1.0 secondo le specifiche
pubblicate dal W3C (\emph{World Wide Web Consortium}).

Nel corso del tempo il formato ha subito diverse evoluzioni e, ad oggi,
esistono differenti versioni di \emph{RSS}. Una prima versione 0.90
venne sviluppata da Netscape nel 1999, seguita da una versione
semplificata, la 0.91. Il vero sviluppo avvenne però per mano del team
di `content management' di Userland che, partendo dal formato RDF
(\emph{Resource Description Framework}), diede vita a RSS 1.0. Dave
Winer, membro del team di Userland ha poi sviluppato una nuova versione
semplificata dal nome di RSS 2.0. La vasta popolarità del formato
\emph{RSS} e la sua lunga storia hanno fatto in modo che il termine
\emph{Feed RSS} venga utilizzato spesso come sinonimo di `web feed'. Si
tratta però di un uso incorretto del termine in quanto non tutti i
formati di \emph{feed} sono \emph{RSS}, ad esempio oggi un altro formato
di vasto utilizzo è Atom, un diretto concorrente di \emph{RSS}.

Nella sua forma più semplice un \emph{Feed RSS} è composto da un
elemento principale \textless{}\emph{rss}\textgreater{} seguito da un
elemento \textless{}\emph{channel}\textgreater{} e dai suoi sotto
elementi \textless{}\emph{item}\textgreater{}. L'elemento
\textless{}\emph{channel}\textgreater{} deve contenere tre informazioni
irrinunciabili: un titolo, un link, una breve descrizione che informi
l'utente delle informazioni contenute nell'aggiornamento. Gli elementi
opzionali spaziano dalla lingua delle informazioni contenute nel feed
alla mail della persona responsabile del sito web che ha creato il
\emph{Feed RSS}.

La funzione principale dei \emph{Feed RSS} è garantire la semplicità di
informazione tramite Internet rendendo automatico l'invio degli
aggiornamenti dei siti o blog che vengono visitati dall'utente. Per
usufruire dei feed generati automaticamente dai siti che interessano il
lettore è necessario fare uso di \emph{Feed Reader} o aggregatori in
grado di monitorare tutti i siti richiesti dall'utente e restituire gli
aggiornamenti, per l'appunto, in forma di sommario facilmente
consultabile.

\section*{Sitografia}
{\parindent0pt 
*\emph{RSS} (\emph{s.v}.), in \emph{Tech Terms},
\url{https://techterms.com/}

\emph{RSS Advisory Board},
\url{http://www.rssboard.org/rss-specification\#whatIsRss}

\emph{Rss20AndAtom10Compared},
\url{http://www.intertwingly.net/wiki/pie/Rss20AndAtom10Compared}
}

\hrulefill 

{[}\emph{Antonio Marson Franchini}{]}




\chapter{Filologia digitale}

La \emph{filologia} \emph{digitale} si configura come il punto di
intersezione fra l'ecdotica e l'informatica. Pertanto, la definizione di
\emph{filologia digitale} non può essere scissa dalla definizione di
\emph{edizione digitale}. In prima istanza, la \emph{filologia digitale}
potrebbe essere definita come l'uso combinato di strumenti informatici e
specifici metodi, al fine di allestire una edizione digitale.

Come sottolinea Andrews (2012, 2) è difficile definire in maniera non
`equivoca' una edizione digitale: ciò dipende dal fatto che il sintagma
può contestualmente riferirsi sia alla trascrizione di un testo critico
(elaborato seguendo l'approccio tradizionale) in formato elettronico --
definizione, invero, un po' banalizzante e priva di una vera portata
innovativa (cfr. Fiormonte 2007) --, sia alla pubblicazione di un testo
critico allestito tramite il ricorso a specifici tools e software
digitali. Pertanto, la Andrews (2012, 2) propone una definizione, per
così dire, `conciliante', asserendo che «we might call `digital
philology' an approach to textual editing that welcomes the aid of
technology wherever possible and which will usually, but not
necessarily, result in a digital publication». In tempi ancor più
recenti, Rosselli Del Turco (2017, 3) propone di intendere, in via
generale, la \emph{filologia digitale} come «l'uso di strumenti e metodi
dell'informatica applicati all'ecdotica con l'obiettivo di creare
un'edizione critica (o diplomatica) di un testo».

Una edizione digitale presenta diversi vantaggi rispetto ad una edizione
critica tradizione (cfr. Rosselli Del Turco 2010 e 2017), fra i quali --
da un punto di vista precipuamente filologico -- vanno annoverati la
possibilità di una gestione `elastica' delle varianti testuali, senza
limiti di spazio per quanto concerne l'estensione dell'apparato critico;
l'intertestualità, intesa come possibilità di prevedere collegamenti con
testi supplementari (quali commenti, edizioni critiche precedenti
\emph{et similia}); la possibilità di consultare i vari testimoni.
\emph{Brevibus verbis}, l'edizione critica digitale si presenta come
un'edizione dinamica: «l'obiettivo del supporto digitale è di sottrarre
all'immobilità della stampa la vivacità del testo criticamente
stabilito» (Giacomelli 2012, contributo al quale si rimanda anche in
merito alle problematiche della digitalizzazione di fonti papiracee; in
merito al `dinamismo' di una edizione digitale ed al suo carattere
ipertestuale, cfr. Tomasi 2018).

L'approccio filologico digitale si fa, dunque, latore di innovazioni
che, naturalmente, non si esauriscono nella \emph{facies} digitale: «la
filologia digitale si avvale di strumenti informatici, ma non è
identificabile con essi» (Rosselli Del Turco 2010). La
`digitalizzazione' del lavoro del filologo non consente soltanto di
rendere più efficienti alcune fasi del lavoro di allestimento di una
edizione critica (quali, in particolare, la \emph{recensio} e la
\emph{collatio codicum}). In tutti i contributi che hanno proposto
riflessioni di natura epistemologica (e conseguenti proposte di
definizione) in merito alla \emph{filologia} \emph{digitale} emerge
chiaramente la consapevolezza che «il passaggio del testo dalla modalità
pre-informatica e gutemberghiana a quella segnata dall'informatica o
digitale è tale da cambiare sostanzialmente non solo il concetto di
testo, ma anche la natura stessa della filologia stessa» (Mordenti 2012,
37). Il filologo digitale adotta un nuovo approccio, che sposta
l'attenzione dal testo `ideale' (per ricostruire il quale si individua
il testimone la cui \emph{auctoritas}, per vari motivi, è superiore a
quella degli altri testimoni) al testo `reale'. In altre parole, la
\emph{filologia digitale} si interessa alla individualità e alle
variazioni di ogni singolo testimone, anziché fornire un `unificato'
\emph{textus receptus}. Come ben sintetizza Mattioda (2007), «l'ambiente
digitale permette un passo in avanti nella visualizzazione e nel
confronto dei testi, ma soprattutto supera l'idea del testo definitivo:
non solo il filologo non sceglie un testo più importante, ma modifica il
suo lavoro grazie ai linguaggi di codifica e ai codici di mark
up\emph{.} Il testo che il filologo dovrà preparare {[}...{]} non è
soltanto un testo univoco, ma un ipotesto che può contenere in sé varie
trascrizioni».

Infine, non bisogna credere che l'approccio filologico digitale annulli
o ridimensioni l'azione umana. La possibilità di includere in una
edizione critica digitale un apparato critico contenente tutte le
varianti e/o i collegamenti a tutti i testimoni non annulla la
responsabilità ecdotica individuale del filologo, il quale opera sulla
base di precisi criteri editoriali (da lui selezionati e chiaramente
indicati), sulla base dei quali dà l'avvio alla propria esegesi dei dati
raccolti. La filologia, seppure in un contesto digitale, resta sempre
sostanzialmente `interpretazione'. Contestualmente, si complessifica il
compito del lettore, il quale dev'essere in grado di ricostruire il
percorso interpretativo del filologo (potendo naturalmente anche
dissentire in merito a determinate scelte). Ciò rende l'edizione
digitale un vero e proprio strumento di ricerca.

\section*{Bibliografia Consigliata}
{\parindent0pt 
Aa.Vv., \emph{Accademia Nazionale dei Lincei}, Tavola rotonda sul tema
\emph{Filologia Digitale: problemi e prospettive}, 135, Roma, Bardi
Edizioni, 2017

Cotticelli Kurras P., \emph{Linguistica e filologia digitale},
Alessandria, Edizioni dell'Orso, 2011

Fiormonte D., \emph{Scrittura e filologia nell'era digitale}, Torino,
Bollati Boringhieri, 2003

Fiormonte D., \emph{Note sul problema della filologia digitale}, in
\emph{Parole online. Quale editoria e filologia nell'era digitale? --
Nuova Informazione Bibliografica}, II, pp. 355-362,
\url{https://infolet.it/files/2009/04/per_rcicala4.pdf}

Mattioda E., \emph{Alcune considerazioni su filologia e testi digitali},
in \emph{Philologie et Politique}, 7, 2007, pp. 163-172,
\url{http://journals.openedition.org/laboratoireitalien/143}

Mordenti R., \emph{Filologia digitale (a partire dal lavoro per
l'edizione informatica dello Zibaldone Laurenziano di Boccaccio}, in
\emph{Humanist Studies \& the Digital Age}, 2, 1, 2012, pp. 37-56
\url{http://journals.oregondigital.org/hsda/article/download/2991/2676}


Spinazzè L., \emph{Filologia digitale. Dalla ricerca alla didattica.
L'informatica umanistica al servizio delle scienze dell'antichità},
Trento, Tangram Edizioni Scientifiche, 2015
}

\section*{Sitografia}
{\parindent0pt 
Andrews T. L., \emph{The third way: philology and critical edition in
the digital Age},
\url{https://boris.unibe.ch/43071/1/variants_postprint.pdf}

Giacomelli I., \emph{Edizione digitale di fonti primarie. Elegie
tirtaiche su papiro: un esempio}, in \emph{Griselda online -- Portale di
letteratura} (2012),
\url{http://www.griseldaonline.it/informatica/edizione-digitale-di-fonti-primarie.html}

*\emph{Informatica umanistica} (\emph{s.v}.), in \emph{Treccani},
\url{http://www.treccani.it/enciclopedia/informatica-umanistica_(XXI-Secolo)/}

Rosselli Del Turco R., \emph{Filologia digitale: ragioni, problemi,
prospettive di una disciplina}, III incontro di filologia digitale,
Verona 3-5 marzo 2010,
\url{http://www.dfll.univr.it/documenti/Iniziativa/dall/dall639204.pdf}

Rosselli Del Turco R., \emph{L'edizione scientifica digitale: strumenti
e progetti}, Verona 20-21 aprile 2017,
\url{http://filologiadigitale-verona.it/wp-content/uploads/2017/04/Introduzione-alla-Filologia-Digitale-1.pdf}

Tomasi F., \emph{Le nuove frontiere della filologia}, in \emph{Argo}, 2,
2018
\url{http://www.argonline.it/argo/argo-n-2/argo-n-2-_-francesca-tomasi-nuove-frontiere-filologia/}
}

\hrulefill 

{[}\emph{Giovanna Battaglino}{]}




\chapter{FTP}

(ingl. Acronimo per \emph{File Transfert Protocol})

È il protocollo TCP/IP di collegamento ad una rete che consente il
trasferimento di file tra cui quelli ASCII, EBCDIC e binari. Si tratta
di un linguaggio studiato per facilitare la comunicazione all'interno di
una architettura client-server. La prima versione venne sviluppata dal
MIT nel 1971 e di proponeva degli obiettivi primari:

\begin{enumerate}
\def\labelenumi{\arabic{enumi}.}
\item
  promuovere la condivisione di file (programmi o dati);
\item
  incoraggiare l'uso indiretto o implicito di computer remoti;
\item
  risolvere in maniera trasparente incompatibilità tra differenti
  sistemi di stoccaggio file tra host;
\item
  trasferire dati in maniera affidabile ed efficiente.
\end{enumerate}

Il protocollo utilizza delle connessioni di tipo TCP (\emph{Transition
Control Protocol}), distinte tra di loro, per trasferire i documenti e
controllare lo stato dei trasferimenti stessi; per garantirne
ulteriormente la sicurezza ad ogni accesso viene richiesta
l'autorizzazione del client attraverso un nome utente ed una password.
Negli anni è stata sviluppata una ulteriore versione del \emph{FTP},
integrandolo con un sottostrato SSL/TLS (\emph{Secure Sockets Layer/
Transport Layer Security}) che riduce il rischio di accessi non
autorizzati alle comunicazioni client-server.

La sua logica di funzionamento è relativamente semplice: \emph{FTP}
utilizza due connessioni diverse per la gestione di dati e comandi ed è
eseguibile solamente tramite l'intercessione di un software chiamato
client. Attraverso il client ci si connetterà al server \emph{FTP}
(solitamente sempre aperto alla porta 21) e, grazie a questa
comunicazione, si aprirà il canale comandi attraversi il quale il client
riuscirà a comunicare con il server. Si possono avere, quindi, due tipi
di canale, uno attivo, che utilizza un numero di porta casuale per poi
spostare il contenuto del trasferimento verso la porta standard, e uno
passivo, il quale sfrutta una porta casuale ma superiore a 1023 senza
spostare il trasferimento dei dati.

Si tratta, comunque, di un linguaggio di alto livello, quindi facilmente
comprensibile all'uomo e quindi relativamente di facile utilizzo.

\section*{Sitografia}
{\parindent0pt 
Poste J. - Reynolds J., \emph{FILE TRANSFERT PROTOCOL (FTP)},
\url{http://www.rfc.altervista.org/rfctradotte/rfc959_tradotta.txt}

}

\hrulefill 

{[}\emph{Alessia Marini}{]}




\chapter{Gold Open Access}

Si dice che una sede scientifica pubblica in \emph{gold open access}
quando i singoli contributi di ogni autore -- una volta editi -- sono
liberamente e immediatamente disponibili online.

\section*{Sitografia}:
{\parindent0pt 
\emph{*Gold Open Access} (\emph{s.v}.), in \emph{Consiglio Nazionale
delle Ricerche. Biblioteca d'Area di Bologna},
\url{http://biblioteca.bo.cnr.it/index.php/it/open-access/pubblicare-oa/item/272-gold-road}
}

\hrulefill 

{[}\emph{Antonello Fabio Caterino}{]}



\chapter{Green Open Access}

Si parla di \emph{green open access} quando l'autore contribuisce a
diffondere il proprio contributo immettendolo in repository
(istituzionali e non) liberamente consultabili. È possibile che alcuni
editori vietino espressamente questa pratica, e che dunque siano
caricate online versioni pre-print o post-print.

\section*{Sitografia}
{\parindent0pt 
\emph{*Green Open Access} (\emph{s.v}.), in \emph{Consiglio Nazionale
delle Ricerche. Biblioteca d'Area di Bologna},
\url{http://biblioteca.bo.cnr.it/index.php/it/open-access/pubblicare-oa/item/271-green-open-access}
}

\hrulefill 

{[}\emph{Antonello Fabio Caterino}{]}

\chapter{Hidden Web}

L'\emph{Hidden Web} (o Invisible Web o Deep Web o Undernet) è l'insieme
delle risorse informative del World Wide Web non indicizzate dai normali
motori di ricerca, come Google, Bing o Yahoo.

Il \emph{Web Sommerso} nasce a scopo militare, quando nel 1969 il DARPA
(\emph{Defense Advanced Research Projects Agency}), un'agenzia
governativa del Dipartimento della Difesa degli Stati Uniti incaricata
di realizzare nuove tecnologie, dà vita all'ARPANET (\emph{Advanced
Research Projects Agency NETwork}), una rete basata su un'architettura
client/server destinata all'uso militare. L'uso di questa rete, in
declino durante gli anni della guerra fredda, rimane sotto il controllo
delle università diventando utile strumento di comunicazione e di
scambio di informazioni scientifiche, fino a quando nel 1991 al CERN
(\emph{Organizzazione Europea per la Ricerca Nucleare}) di Ginevra si
definisce il protocollo HTTP (\emph{HyperText Transfer Protocol}), che
permette la nascita del World Wide Web e la diffusione di Internet. Per
proteggere le comunicazioni governative, la Marina statunitense inizia
nel 2002 a servirsi di TOR (\emph{The Onion Router}), un browser capace
di garantire l'anonimato attraverso il continuo rerouting su nodi, che
l'Elettronic Frontier Foundation, un'organizzazione no profit promotrice
dei diritti e delle libertà digitali, inizia a finanziare nel 2004
perché se ne estenda l'utilizzo al di fuori dell'ambito militare. Molti
anni dopo l'uso di questa rete è concesso anche ai civili.

L'enorme vastità del Web e l'incalcolabile quantità di dati di cui è
portatore, ha posto gli studiosi di fronte alla difficoltà di
individuare e circoscrivere i livelli che lo costituiscono riconoscendo
a ciascuno di essi delle peculiarità. I loro tentativi si riassumono
nella divisione in tre macroaree corrispondenti a:

\begin{enumerate}
\def\labelenumi{\arabic{enumi}.}
\item
  \emph{Web Surface}, il livello più superficiale che comprende i siti
  facilmente accessibili e vari circuiti privati;
\item
  \emph{Deep} \emph{Web}, a cui si accede solo tramite proxy e in cui
  sono presenti comunità di hackers, libri di qualsiasi formato e su
  qualsiasi argomento, ma anche reti internet per compagnie e
  università;
\item
  \emph{Dark} \emph{Web}, un sottoinsieme del Deep Web, a cui è
  possibile accedere mediante browser speciali, come TOR, I2P o Freenet.
\end{enumerate}

«Deep Web e Dark Web», dice Paganini, tra i massimi esperti di sicurezza
italiana a livello internazionale, «sono termini spesso abusati e
confusi e di consueto associati ad attività criminali. Con il termine
Deep Web si indica l'insieme dei contenuti presenti sul web e non
indicizzati dai comuni motori di ricerca, mentre con il termine Dark Web
si indica l'insieme di contenuti che sono ospitati in siti web il cui
indirizzo IP è nascosto, ma ai quali chiunque può accedere» tramite
browser specifici (Paganini 2012).

Il Deep Web contiene archivi digitali dei dipartimenti pubblici, sistemi
di comunicazioni private, sottoreti private come quelle universitarie o
governative, contenuti dinamici, pagine ad accesso ristretto, script,
dati medici o dei social network, etc., mentre nel Dark Web si è in
grado di trovare ogni genere di risorsa e tutti quei contenuti che per
vari motivi non posso essere pubblicati in Internet senza conseguenze
legali, come merce rubata, traffici d'armi e droga o di bitocoin, la
moneta digitale.

Il Web nel quale navighiamo, il Surface Web (o Visible Web o Indexed Web
o LightNet) rappresenta solo una minima percentuale dei contenuti reali
della rete, mentre sotto la punta dell'Iceberg c'è il Deep Web, di
contenuto 500 volte maggiore a quello consultabile in superficie.
Secondo i principali esperti di sicurezza mondiale il Surface Web è
sotto costante sorveglianza dei governi e rappresenta il 4\% dei
contenuti del world wide web, mentre il Deep Web ne rappresenta il 96\%.

L'\emph{Hidden Web} è oggi utilizzato da un'ampia gamma di persone e non
solo per intenti criminali, ma anche per eludere la censura introdotta
dai governi in aree critiche del pianeta, per garantire spazi di
confronto politico, per organizzare biblioteche di libri proibiti o per
la diffusione di informazioni. A~gennaio 2016 è uscito il primo numero
di~\emph{The Torist}, la prima rivista letteraria online pubblicata sul
Deep Web, fondata da Gerard Manley Hopkins (G.M.H.) e il Prof. Robert W.
Gehl, insieme ad altri utenti. La rivista è lunga 51 pagine e contiene
tre racconti, poesie di due autori diversi, due saggi e una prefazione
di cinque pagine fatta dagli autori. I temi maggiormente quotati
esaminano per lo più il rapporto tra l'uomo e la tecnologia, la privacy
e i rischi dell'eccessiva sorveglianza sugli individui e la cultura del
Web in generale con un approccio spesso critico e distopico.

\section*{Bibliografia Consigliata}
{\parindent0pt 
Echeverri D., \emph{Deep Web: TOR, FreeNET \& I2P: privacidad y
anonimato}, Madrid, OXWORD, 2016

Florindi E., \emph{Deep web e bitcoin. Vizi privati e pubbliche virtù
della navigazione in rete}, Reggio Emilia, Imprimatur, 2016

Frediani C., \emph{Deep Web. La rete oltre Google. Personaggi, storie,
luoghi dell'internet profonda}, Genova, Stampa Alternativa, 2014

Meggiato R., \emph{Il lato oscuro della Rete: alla scoperta del Deep Web
e del Bitcoin}, Milano, Apogeo, 2014

Paganini P., \emph{The Deep Dark Web: The Hidden World: Volume 1},
Napoli, Apogeo, 2012

Sui D.- Caverlee J.- Rudesill D., \emph{The deep web and the darknet: a
look inside the internet's massive black box}, Washington, Wilson
Center, 2015,
\url{https://www.wilsoncenter.org/sites/default/files/stip_dark_web.pdf}

Scheeren W. O., \emph{The hidden web. A sourcebook}, Santa Barbara,
Libraries Unlimited, 2012

Wojtowicz P., \emph{Darknet and deep web: il lato oscuro del web per la
privacy e la protezione dati}, Tesi di Laurea, Università di Bologna,
2013/2014,
\url{https://amslaurea.unibo.it/8456/1/wojtowicz_patryk_tesi.pdf}
}

\section*{Sitografia}
{\parindent0pt 
AaVv, \emph{Below the Surface: exploring the Deep Web}, 2015,
\url{https://documents.trendmicro.com/assets/wp/wp_below_the_surface.pdf}

\emph{Cybercrime in the Deep Web},
\url{https://www.blackhat.com/docs/eu-15/materials/eu-15-Balduzzi-Cybercrmine-In-The-Deep-Web-wp.pdf}

\emph{Motori di ricerca scientifici},
\url{https://www.unipa.it/biblioteche/Cerca-una-risorsa/Motori-di-ricerca-scientifici-/}

\emph{*Web Sommerso}~(\emph{s.v}.), in \emph{Wikipedia},
\url{https://it.wikipedia.org/wiki/Web_sommerso}

\emph{What is the Deep Web? A first trip into the abyss},
\url{http://www.aofs.org/wp-content/uploads/2013/04/130925-What-is-the-Deep-Web.pdf}
}

\hrulefill 

{[}\emph{Alessandra Di Meglio}{]}



\chapter{HTML}

(ingl. Acronimo per \emph{HyperText Markup Language})

È un linguaggio di marcatura che descrive la struttura delle pagine web
e definirne i collegamenti ipertestuali, utilizzando una serie di
marcatori predefiniti e non modificabili; rappresenta, quindi, uno
standard riconosciuto dal \emph{World Wide Web Consortium} (W3C). Come
SGML, suo progenitore, e XML, l'\emph{HTML} attua una fondamentale
distinzione tra il testo, ovvero il contenuto vero e proprio della
pagina web, e i markup che rappresentano, invece, l'ossatura del
documento che lo rendono `leggibile' dai browser.

Il linguaggio venne sviluppato da Tim Barners-Lee presso il CERN DI
Ginevra (Svizzera) durante i primissimi anni '90 in concomitanza con il
protocollo HTTP pensato per il trasferimento dei documenti codificati
con tale formato. L'idea, però, risale al 1989, quando lo stesso Lee
propose un progetto, denominato World Wide Web, che riguardava la
pubblicazione di ipertesti in un ambiente condiviso: i risultati e
l'applicazione di tale progetto furono utilizzati solamente all'interno
del CERN fino al 1991; generalmente si tende a far risalire anche la
nascita di Internet così come lo conosciamo a quella stessa data. Con
questa `liberalizzazione' ebbe inizio una sorta di `guerra dei browser'
durante la quale ogni browser esistente utilizzava la propria struttura
basata su \emph{HTML} il che, però, rendeva impossibile la corretta
visualizzazione di una pagina web su un browser diverso da quello per in
cui era stata creata. La questione venne definitivamente risolta nel
1995 quando, un anno dopo la nascita del W3C, venne proposta la versione
3.0 del linguaggio che divenne così uno standard. Nel 1999, con
l'\emph{HTML} 4, venne introdotta una novità fondamentale, ovvero la
possibilità di creare un foglio di stile, denominato CSS, che fosse
esterno al documento \emph{HTML}, in modo da rendere più snella la
codifica alleggerendo di conseguenza il documento. L'ultima versione del
linguaggio è la 5.0 e risale al 28 ottobre 2014: grazie ad essa ora il
linguaggio può avvalersi senza problemi di controlli più sofisticati
sulla resa grafica e interazioni dinamiche con l'utente, avvalendosi di
altri linguaggi come il JavaScript, il JSON o altre applicazioni
multimediali per l'animazione vettoriale o di streaming audio/video.

Al livello strutturale un documento \emph{HTML} ha bisogno di una
dichiarazione iniziale in cui si specificano la versione del linguaggio
utilizzata e la conseguente codifica caratteri (che è lo standard UTF-8)
e l'uso necessario di determinato marcatori, come
\emph{\textless{}html\textgreater{} \textless{}head} e
\emph{\textless{}body}. A differenza del fratellastro XML,
l'\emph{HTML} non dà la possibilità di ampliare la gamma di marcatori
utilizzabili, quindi la struttura di qualsiasi file \emph{HTML} risulta
grossomodo uguale:

\begin{enumerate}
\def\labelenumi{\arabic{enumi}.}
\item
  Inizia con \emph{html} e deve necessariamente
  chiudersi con \emph{/html};
\item
  successivamente deve essere utilizzato il tag
  \emph{head} nel quale vanno inserite
  informazioni scontate come il \emph{title} o
  più complesse come quelle necessarie alla visualizzazione del
  contenuto, definite dal tag \emph{style}.
\item
  Chiuso l'\emph{\textless{}/head} viene aperto il
  \emph{\textless{}body}, che rappresenta il vero `corpo'
  del documento in cui verrà inserito il testo, le immagini, i link con
  pagine esterne, eventuali video o animazioni con JavaScript.
\item
  Il testo viene definito da marcatori prettamente tipografici come
  \emph{\textless{}p} per i vari paragrafi o
  \emph{\textless{}h1\textgreater{}\textless{}h2\textgreater{}
  \textless{}h3} per i vari livelli di titolazione.
  All'interno del testo, poi, sarò possibile inserire tags come
  \emph{\textless{}i} per il corsivo o
  \emph{\textless{}b} per il grassetto; notiamo subito
  come essi siano marcatori prettamente fisici, il che rappresenta una
  delle fondamentali differenze con l'XML, il quale, invece, si serve di
  marcatori logici. Ciò non toglie che anche nella sintassi \emph{HTML}
  è possibile ritrovare marcatori logici come
  \emph{\textless{}em} (che evidenzia una porzione di
  testo più estesa del singolo rigo), ma è raro trovarli o utilizzarli.
\end{enumerate}

\section*{Bibliografia Consigliata}
{\parindent0pt 
Duckett J., \emph{HTML e CSS. Progettare e costruire siti web. Con
Contenuto digitale per download e accesso on line}, Milano, Apogeo, 2017

Tissoni F., \emph{Lineamenti di editoria multimediale}, Milano, Edizioni
Unicolpi, 2009
}

\section*{Sitografia}
{\parindent0pt 
*\emph{HTML} (\emph{s.v}.), in \emph{w3schools.com},
\url{http://www.w3schools.com/}
}


\hrulefill 
 
{[}\emph{Alessia Marini}{]}



\chapter{Informatica umanistica}

L'\emph{Informatica umanistica} o \emph{Digital Humanities} (anche detta
\emph{Humanities Computing}) è un settore di ricerca interdisciplinare
che incrocia contenuti e metodi propri delle discipline umanistiche con
il supporto delle nuove tecnologie informatiche (Catalani 2018); essa
coniuga, cioè, il campo tecnologico-informatico, che si occupa del
trattamento automatico delle informazioni, con le scienze letterarie,
che hanno per oggetto la conoscenza dell'uomo e del suo pensiero.

Tutt'oggi priva di uno statuto epistemologico e di un'identità che la
definisca, l'\emph{Informatica umanistica} parte dall'indagine critica
dei concetti e dei principi che condizionano l'informazione, le sue
dinamiche e l'etica informatica, e risponde a una necessità ben più
profonda: conseguire una riflessione consapevole e di alto spessore
culturale allo scopo di formare~soggetti concretamente -- e
culturalmente -- competenti. Nata dalla Linguistica computazionale e
dallo zelo di padre Roberto Busa, che nel 1949 iniziò
l'\href{http://www.corpusthomisticum.org/it/index.age}{\emph{Index
verborum}} dell'\emph{opera omnia}~di Tommaso d'Aquino, la nuova
disciplina~ha suscitato fin da subito curiosità e sospetto, finché non
ha abbattuto le barriere dei settori `tradizionali'
istituzionalizzandosi e moltiplicando i centri, i dipartimenti e i corsi
di studio a suo nome. Si è introdotta, in questo modo, la figura
accademica dell'umanista informatico, il cui compito, secondo Ramsay, è
di `costruire cose', riflettere sull'idea di costruzione e progettare in
modo tale che altri possano costruire (Ramsay 2011a) avviando,
attraverso il processo di `building', una fase evolutiva che ha trovato
negli anni terreno fertile. Tuttavia, la mancata istituzione di un
settore scientifico-disciplinare specifico per l'\emph{Informatica
umanistica} e la frammentarietà delle iniziative pongono ancora ostacoli
al suo consolidamento (Monella 2013). L'opinione comune stabilisce,
infatti, che le Digital Humanities non siano una disciplina con un
proprio oggetto di indagine, ma semplicemente ricerca umanistica portata
avanti con strumenti digitali. Essa, cioè, è considerata, \emph{sic et
simpliciter}, mero strumento di supporto dello studio e della ricerca,
ottemperando al ruolo di vassallo della disciplina madre. Si ignora
invece che l'uso dell'\emph{Informatica umanistica} ha trasformato la
ricerca dall'interno, i suoi obiettivi, l'\emph{iter}, i suoi traguardi
qualitativi e l'espansione quantitativa e che, essendo ancora \emph{in
fieri} e allo stato sperimentativo, essa chiede all'umanista di
riflettere e interrogarsi sul senso dell'uso informatico nelle
discipline letterarie, sulle sue possibilità e sui suoi limiti, dando --
o togliendo -- valore al supporto mediatico. È la pratica quotidiana,
infatti, che conduce a una maggiore consapevolezza dei metodi
computazionali, degli strumenti digitali e di quelli informatici, con
l'impegno di produrre risorse oggi sempre più valide.

Sicuramente lo \emph{status} teorico dell'\emph{Informatica umanistica}
pone chiunque tenti di mettere ordine alla sua materia nella scomoda
posizione di dover trattare una vastità di dati -- e di pregiudizi --
impossibili da contenere data la pluridisciplinarità delle Digital
Humanities. Tuttavia riconoscerle un ruolo è oggi importante. Essa
infatti rappresenta il futuro delle discipline letterarie, di cui
intende potenziare la tradizione culturale senza per questo modificarne
l'effettiva natura.

\section*{Bibliografia consultata}
{\parindent0pt 
Flanders J. - Unsworth J., \emph{The Evolution of Humanities Computing
Centers}, in \emph{Computers and the Humanities}, XXXVI, 4, 2002, pp.
379-380

Gold M.K., \emph{The Digital Humanities Moment}, in Gold M. W. (a cura
di), \emph{Debates in the Digital Humanities}, Minneapolis-London,
University of Minnesota Press, 2012, pp. IX-XVI
}

\section*{Bibliografia Consigliata}
{\parindent0pt 
Berry D.M., \emph{Understanding Digital Humanities}, United Kingdom,
Palgrave Macmillan, 2012

Bodard G. - Mahony S., \emph{Digital Research in the Arts and
Humanities}, Surrey: England, Taylor \& Francis Ltd., 2010

Carter B.W., \emph{Digital Humanities: current perspective, practices,
and research}, United Kingdom, Emerald Group Publishing Limited, 2013

Gardiner E. - Musto R.G., \emph{The Digital Humanities}, Cambridge,
Cambridge University Press, 2015

Numerico T. - Fiormonte D. - Tomasi F., \emph{L'umanista digitale},
Bologna, Il Mulino, 2010

Rydberg-Cox J.A., \emph{Digital libraries and the challenges of digital
humanities}, Oxford, Chandos Publishing, 2006

Schreibman S\emph{. -} Siemens R\emph{. -} Unsworth J.\emph{, A
Companion to Digital Humanities}, Oxford, Blackwell Publishing, 2004

Tomasi F.,~\emph{Metodologie informatiche e discipline umanistiche},
Roma, Carocci, 2008

Tomasi F. - Anselmi G.M.,~\emph{Informatica e letteratura}, in \emph{Le
forme e la storia}, 2011, IV, pp. 163 -182

Tomasi F.,~\emph{Informatica Umanistica: iniziative, progetti e
proposte}, in \emph{La macchina nel tempo. Studi di informatica
umanistica in onore di Tito Orlandi}, Firenze, Le Lettere, 2011, pp. 309
- 327
}

\section*{Sitografia}
{\parindent0pt 
Buzzetti D., \emph{Che cos'è, oggi, l'informatica umanistica? L'impatto
della tecnologia}, in Ciotti F. - Crupi G. (a cura di),
\emph{Dall'Informatica umanistica alle culture digitali. Atti del
convegno di studi (Roma, 27-28 ottobre 2011) in memoria di Giuseppe
Gigliozzi},
\url{https://www.academia.edu/2305364/Che_cos_e_oggi_l_informatica_umanistica_L_impatto_della_tecnologia}

Catalani L., \emph{Informatica Umanistica e Digital Humanities},
\url{https://medium.com/@luigicatalani/informatica-umanistica-e-digital-humanities-ff57c44d68be}

*\emph{Informatica} (\emph{s}.\emph{v}.), in \emph{Treccani},
\url{http://www.treccani.it/enciclopedia/informatica/}

*\emph{Informatica} \emph{umanistica} (\emph{s}.\emph{v}.), in
\emph{Wikipedia},
\url{https://it.wikipedia.org/wiki/Informatica\_umanistica}

Monella P., \emph{L'informatica umanistica tra istituzionalizzazione e
strumentalismo},
\url{http://www1.unipa.it/paolo.monella/lincei/files/where/strumenti_v2.0.pdf}

*\emph{Umanistico} (\emph{s}.\emph{v}.), in \emph{Treccani},
\url{http://www.treccani.it/vocabolario/umanistico/}
}

\hrulefill 

{[}\emph{Alessandra Di Meglio}{]}

\chapter{Interfaccia utente (UI)}

In termini generali si tratta di un termine generale per ogni sistema,
fisico o digitale che permette all'utente l'accesso ad una data
tecnologia. È quindi il punto di contatto tra l'uomo e il programma del
computer. Ogni \emph{interfaccia utente} (UI) è unica e legata alla
funzione che deve svolgere ma tutte hanno in comune certi elementi di
base. Una buona \emph{UI} deve essere user-friendly e quindi permettere
un'esperienza d'utilizzo naturale ed intuitiva che non frustri l'utente.
Lo sviluppo delle \emph{UI} ha avuto inizio a partire dalla seconda metà
del XX secolo; si può considerare il sistema di schede forate dei primi
computer come la prima elaborazione di una \emph{UI}. Ad oggi, dopo
decenni di sviluppo nel campo dell'informatica si è passati dalle
\emph{Text-Based UI} alle \emph{Graphical User Interface} (GUI) che
permettono una navigazione semplificata all'interno dei software.

Un'interfaccia è composta da un set di comandi, o menù, attraverso i
quali si riesce a comunicare con il programma software. Una \emph{UI}
\emph{command-driven} richiede che l'utente inserisca direttamente i
comandi; una \emph{menu-driven} permette un sistema di selezione dei
comandi attraverso dei menù che compaiono a schermo.

Tra le \emph{menu-driven UI} si trovano le cosiddette GUI. Si tratta
infatti di un'elaborazione grafica a schermo che permette all'utente di
non interfacciarsi direttamente con il DOS del computer come si era
obbligati a fare prima dello sviluppo di questo sistema di elaborazione.
Ad oggi questo tipo di interfacce sono considerate lo standard per lo
sviluppo di nuovi software dedicati ad un mercato di consumo di massa.
Elementi chiave sono: un puntatore, un sistema di puntamento, icone, un
desktop, finestre e menù.

\section*{Sitografia}
{\parindent0pt 
*\emph{User Interface} (\emph{s.v}.), in \emph{Tech} \emph{Terms},
\url{https://techterms.com/}

*\emph{User Interface} (\emph{s.v}.), in \emph{webopedia},
\url{https://www.webopedia.com/TERM/U/user_interface.html}

}

\hrulefill 

{[}\emph{Antonio Marson Franchini}{]}



\chapter{Machine learning}

Il \emph{Machine Learning} (ML) -- in italiano Apprendimento Automatico
-- è una branca dell'informatica, e in particolare dell'intelligenza
artificiale (IA), che realizza algoritmi in grado di imparare
autonomamente il compito richiesto -- senza essere preventivamente
programmati a farlo -- e di apprendere informazioni dall'esperienza
mediante metodi matematico -- computazionali. Tali metodi consentono
alla macchina di prendere decisioni liberamente tenendo conto delle
probabilità di accadimento di un evento e di modificare gli stessi
algoritmi man mano che riceve informazioni. Il primo a coniare il
termine \emph{Machine Learning} fu un noto scienziato americano e
pioniere dell'IA, Arthur Lee Samuel, nel 1959, anche se la definizione
oggi più accreditata è quella di Tom Michael Mitchell, direttore del
`Dipartimento Machine Learning' della Carnegie Mellon University, il
quale ha affermato che: «si dice che un programma apprende
dall'esperienza E rispetto ad una certa classe di compiti T e ad una
misura della performance P, se la sua performance nell'eseguire i
compiti in T migliora con esperienza E». In pratica, gli algoritmi del
\emph{Machine Learning} migliorano le loro prestazioni mediante
l'esperienza e gli esempi da cui apprendono.~

Le prime sperimentazioni risalgono agli inizi degli anni Cinquanta del
Novecento, quando alcuni matematici e statistici pensarono di utilizzare
i metodi probabilistici per realizzare macchine intelligenti in grado di
prendere decisioni autonomamente. Fu Alan Turing, in particolare, a
ipotizzare degli algoritmi specifici che ottemperassero a tale compito
indirizzando la ricerca in altra direzione. Infatti, se nella
programmazione tradizionale tutta la conoscenza era codificata
all'interno del programma e la macchina era in grado di applicare solo
le regole preinserite, nell'apprendimento automatico, al contrario, la
macchina costruisce da sola le regole -- previa introduzione di esempi
-- e capisce se un nuovo caso risponde o meno alla regola che ha
ricavato.

Caratteristica comune ai vari algoritmi di apprendimento automatico è la
presenza di dati in input, detti `Training Pattern', che inducono alla
generazione di output, o Istanze dell'esperienza. A seconda
dell'algoritmo utilizzato -- ossia a seconda delle modalità con cui la
macchina impara ed accumula dati e informazioni -- si distinguono
quattro diversi sistemi di apprendimento automatico identificati dallo
stesso Arthur Samuel alla fine degli anni '50:~apprendimento
supervisionato,~non supervisionato, semi-supervisionato e~per rinforzo,
utilizzati in maniera differente a seconda della macchina su cui si deve
operare, per garantire la massima performance e il migliore risultato
possibile.

\begin{enumerate}
\def\labelenumi{\arabic{enumi}.}
\item
  L'apprendimento supervisionato~(o \emph{Supervised Learning}) consiste
  nel fornire alla macchina una serie di nozioni specifiche e codificate
  (cioè input), ossia di modelli ed esempi che le permettono di
  costruire un vero e proprio~database di informazioni e di esperienze,
  a cui attinge ogniqualvolta si trovi di fronte a un problema, fornendo
  una risposta (o output) sulla base di esperienze già codificate. In
  questo tipo di apprendimento la macchina deve solo scegliere quale sia
  la migliore risposta allo stimolo che le viene dato.
\end{enumerate}

L'apprendimento supervisionato si usa soprattutto per problemi di:

\begin{itemize}
\item
  classificazione, un processo mediante il quale la macchina è in grado
  di riconoscere e categorizzare oggetti visivi e dimensionali decidendo
  a quale categoria appartiene un determinato dato;
\item
  regressione, corrispondente alla capacità della macchina di prevedere
  il valore futuro di un dato avendo il suo valore attuale (es. la
  quotazione delle valute o delle azioni di una società).
\end{itemize}

\begin{enumerate}
\def\labelenumi{\arabic{enumi}.}
\setcounter{enumi}{1}
\item
  L'apprendimento non supervisionato (o \emph{Unsupervised
  Learning}\textbf{)} prevede invece che gli input inseriti all'interno
  della macchina non siano codificati e che sia la stessa macchina a
  catalogare le informazioni in proprio possesso e a organizzarle,
  imparandone il significato e soprattutto il risultato a cui portano
  (output). In questo caso la macchina dovrà organizzare le informazioni
  in maniera intelligente e dovrà imparare quali sono i risultati
  migliori per le differenti situazioni che si presentano.
\end{enumerate}

Si ricorre all'apprendimento non-supervisionato per risolvere
principalmente due tipologie di problemi:

\begin{itemize}
\item
  raggruppamento~(o \emph{clustering}), quando occorre raggruppare dati
  che presentano caratteristiche simili. In questo caso il programma non
  utilizza dati inseriti e categorizzati in precedenza, ma ricava esso
  stesso la regola raggruppando casi analoghi;
\item
  associazione, che si pone l'obiettivo di trovare schemi frequenti,
  associazioni o correlazioni tra un insieme di oggetti o item (dati),
  tentando di formulare regole che ne predicano gli output sulla base di
  altri item simili. Essa è usata soprattutto nel Data Mining, un
  processo mediante il quale si estraggono informazioni utili da grandi
  quantità di dati attraverso metodi automatici o semiautomatici.
\end{itemize}

\begin{enumerate}
\def\labelenumi{\arabic{enumi}.}
\setcounter{enumi}{2}
\item
  L'apprendimento semi-supervisionato (o \emph{Semi-Supervised
  Learning}) prevede di fornire al computer un set di dati (come
  nell'apprendimento supervisionato), ma incompleti (come in quello non
  supervisionato), che tocca alla macchina codificare.
\end{enumerate}

\begin{enumerate}
\def\labelenumi{\arabic{enumi}.}
\setcounter{enumi}{3}
\item
  L'apprendimento per rinforzo (o \emph{Reinforcement
  Learning}\textbf{)}~rappresenta il sistema di apprendimento più
  complesso. Esso prevede che la macchina sia dotata di strumenti e
  procedimenti in grado di migliorare il proprio apprendimento e,
  soprattutto, di comprendere le caratteristiche dell'ambiente
  circostante. In questo caso alla macchina vengono forniti una serie di
  elementi di supporto, quali sensori, telecamere, GPS, che le
  permettono di rilevare l'ambiente circostante e di effettuare scelte
  per un migliore adattamento.
\end{enumerate}

Ci sono poi altre sottocategorie di \emph{Machine Learning}, tra cui il
Deep Learning (o Apprendimento Profondo), che usa
\href{https://www.ai4business.it/intelligenza-artificiale/deep-learning/reti-neurali/}{reti
neurali}~-- neuroni artificiali tra loro interconnessi che simulano il
funzionamento neurale in un sistema informatico -- per processare
informazioni e comprendere gli schemi presenti nei grandi volumi di dati
(o Big Data).

Molti settori riconoscono oggi il valore del \emph{Machine Learning}
grazie al quale sono in grado di lavorare con più efficienza e di
acquisire un vantaggio competitivo; tra questi: i servizi finanziari
(banche e altre aziende), la pubblica amministrazione (enti pubblici di
sicurezza o dei servizi), l'assistenza sanitaria, il marketing, i
trasporti, etc.

\section*{Bibliografia Consigliata}
{\parindent0pt 
Alpaydin E., \emph{Introduction to Machine Learning}, Cambridge, Mit
Press, 2009

Bishop C.M., \emph{Pattern recognition and Machine Learning}, Singapore,
Springer Verlag, 2006

Chapelle O. - Shölkopf B. - Zien A., \emph{Semi-Supervised Learning},
Cambridge (MA), Mit Press, 2006

Goutte G. \emph{et al}., \emph{Learning machine translation}, Cambridge
(MA), Mit Press Ldt., 2009

Harrington P., \emph{Machine Learning in action}, New York, Manning
Publications, 2012

Holmes D.E. - Jain L.C., \emph{Innovations in Machine Learning. Theory
and applications}, The Nethedlands, Springer, 2006

Kononenko I. - Kukar M., \emph{Machine Learning and Data Mining:
Introduction to principles and algorithms}, United Kingdom, Horwood
Publishing Chichester, 2007

Marsland S., \emph{Machine Learning: an algorithmic perspective},
Cambridge, Chapman and Hall/CRC, 2009

Powers D.M.W. - Turk C.C.R., \emph{Machine Learning of natural
language}, London, Springer, 1989

Sammut C. - Webb G.I., \emph{Encyclopedia of Machine Learning}, New
York, Sammut, 2011

Sebe N. \emph{et al}., \emph{Machine Learning in Computer vision}, The
Netherlands, Springer, 2005

Sutton R.S. - Barto A.G., \emph{Reinforcement Learning: an
introduction}, Cambridge (MA), Bradford Books, 1998
}

\section*{Sitografia}
{\parindent0pt 
*\emph{Apprendimento Automatico} (\emph{s}.\emph{v}.), in
\emph{Wikipedia},
\url{https://it.wikipedia.org/wiki/Apprendimento_automatico}

Mitchell T.M., \emph{Machine Learning}, Maidenhead 1997,
\url{https://www.cs.ubbcluj.ro/~gabis/ml/ml-books/McGrawHill\%20-\%20Machine\%20Learning\%20-Tom\%20Mitchell.pdf}

Nilsson N.J., \emph{Introduction to Machine Learning. An early draft of
a proposed textbook}, Stanford 1998
\url{https://ai.stanford.edu/~nilsson/MLBOOK.pdf}

Shai S.S.-Shai B.D., \emph{Understanding Machine Learning. From theory
to algorithms}, USA 2014
\url{https://www.cs.huji.ac.il/~shais/UnderstandingMachineLearning/understanding-machine-learning-theory-algorithms.pdf}

Stefenelli M., \emph{Apprendimento automatico nei giochi di strategia},
Tesi di Laurea, Università di Bologna Campus di Cesena, 2013/2014
\url{https://amslaurea.unibo.it/7660/1/TESI_Marco_Stefenelli.pdf}
}

\hrulefill 

{[}\emph{Alessandra Di Meglio}{]}



\chapter{Markup}

Si tratta di uno dei due sistemi di elaborazione del testo elettronico:
si contrappone al WYSIWYG (\emph{What You See Is What You Get}) e, per
via delle sue caratteristiche, non è legato ad un determinato tipo di
programma che ne garantisca la lettura. In italiano il termine si può
tradurre con linguaggio di `marcatura' o di `codifica'. Permette di
descrivere la struttura, composizione, impaginazione del testo (o di
qualsiasi altro file di contenuto) attraverso comandi visibili (sequenze
di caratteri ASCII) immessi nel file di testo.

Se ne possono distinguere due tipi: `procedurali' (basati su un
linguaggio specifico) e `dichiarativi' (basati su un linguaggio
generico). Il primo consiste in un insieme di istruzioni operative che
indicano localmente la struttura tipografica e compositiva della pagina.
Viene così definito perché indica alla macchina le procedure di
trattamento a cui sottoporre i file di testo. Ha come obiettivo
principale l'aspetto tipografico del documento. Il secondo, invece, ha
come obiettivo la descrizione della struttura astratta del documento in
cui i simboli al posto che indicare il trattamento tipografico della
pagina ne identificano le strutture interne (paragrafi, etc.). Entrambi
utilizzano un sistema di tag (marcatori) che identificano le parti del
discorso all'interno del file di testo ma solo i linguaggi di tipo
`dichiarativo' si occupano di identificarne le qualità logiche e non
tipografiche. Questo sistema di marcatori viene sempre utilizzato in una
forma gerarchizzata che dall'elemento più grande da definire giunge fino
al più piccolo.

I linguaggi di \emph{markup} sono lo strumento standard per lo sviluppo
di oggetti digitali. Il capostipite di questi linguaggi è SGML
(\emph{Standard Generalized Markup Language}) creato nel 1986 da Charles
Goldfarb e risultato tanto funzionale da diventare lo standard ISO
(\emph{International Standard Organization}) per la codifica dei testi.
Ad oggi, nell'ambito dell'informatica umanistica lo standard è il
linguaggio di \emph{markup} TEI (\emph{Text Encoding Initiative}) basato
sullo standard XML (\emph{eXstensible Markup Language}).

\hrulefill 

{[}\emph{Antonio Marson Franchini}{]}



\chapter{Metadato}

Con il termine \emph{metadato}, mutuato dall'inglese \emph{metadata} --
costituito dal greco μετὰ~(\emph{meta:} oltre, dopo) e dal plurale
neutro latino \emph{data}, informazioni -- si indica l'insieme delle
informazioni strutturate relative ai dati.

L'origine inglese del termine influenza tenacemente l'uso, tanto che la
forma singolare \emph{metadato} (ricavata all'interno del sistema
morfologico italiano) è di utilizzazione più infrequente.

I \emph{metadati} sono legati al ciclo di vita di una risorsa digitale
nel senso che ogni oggetto/risorsa digitale necessita di un bagaglio di
dati che, pur non essendo parte del suo contenuto informativo, è
indispensabile a renderlo rappresentabile attraverso specifiche
procedure di decodifica. Tali informazioni associate, dati sui dati,
prendono appunto il nome di \emph{metadati}.

I \emph{metadati} sono quindi informazioni strutturate che descrivono,
spiegano, localizzano o comunque rendono possibile il recupero, l'uso e
la gestione di una risorsa informativa. Le relazioni tra risorse
informative e \emph{metadati}, oltre ad essere indispensabili, si
prestano a svariate declinazioni: i \emph{metadati} possono essere
statici o dinamici; `embedded' -- ossia inclusi nella risorsa come parte
integrante -- o collegati ad essa in virtù di meccanismi identificativi
stabili e sicuri nel tempo; un medesimo oggetto può prevedere
l'associazione a diversi schemi di \emph{metadati} come diversi e più
oggetti possono essere associati tra loro attraverso \emph{metadati}.
Inoltre, il ciclo di vita di una risorsa digitale deve essere
accompagnato dal costante aggiornamento del \emph{corpus} dei relativi
metadati, affinché tutto quanto accade resti tracciato e documentato.

La comunità scientifica internazionale, preso atto della necessità di
associare alle entità digitali un insieme di \emph{metadati} con i
caratteri di completezza, accuratezza e appropriatezza, ha elaborato
alcuni schemi e modelli a cui fare riferimento, proponendo una
distinzione in tre categorie funzionali:

\begin{enumerate}
\def\labelenumi{\arabic{enumi}.}
\item
  \emph{metadati descrittivi} (\emph{descriptive metadata})\emph{:}
  finalizzati all'identificazione e al recupero degli oggetti digitali;
  sono costituiti da descrizioni normalizzate dei documenti fonte (o dei
  documenti digitali nativi), risiedono generalmente nelle basi dati dei
  sistemi di \emph{Information Retrieval} all'esterno degli archivi
  degli oggetti digitali e sono collegati a questi ultimi con link
  dedicati;
\item
  \emph{metadati strutturali} (\emph{structural metadata}): volti a
  descrivere la struttura interna dei documenti e a gestire le relazioni
  esistenti tra le varie parti componenti degli oggetti digitali;
\item
  \emph{metadati amministrativo-gestionali}, funzionali alla gestione
  degli oggetti digitali all'interno dell'archivio e alla cura degli
  aspetti più propriamente tecnici.
\end{enumerate}

A complemento ed integrazione dell'indicata tripartizione vi sono poi i
\emph{metadati} per la gestione dei diritti, i \emph{metadati} per la
security, i \emph{metadati} per le informazioni personali, i
\emph{metadati} per la conservazione a lungo termine, le cui funzioni
compaiono diversamente esplicitate e valorizzate in specifici modelli e
schemi.

In letteratura il riferimento più noto e ricorrente in materia di
\emph{metadati} di valenza descrittiva è lo standard \emph{ISO 15836},
comunemente noto come \emph{Dublin Core (DC, in realtà Dublin Core
Metadata Initiative, DCMI)}.

Il \emph{Dublin Core} è un insieme di \emph{metadati}, poi implementato,
progettato per la descrizione di una qualunque risorsa informativa,
indipendentemente dal dominio di appartenenza; la sua vocazione è quella
di integrarsi con altri standard di \emph{metadati}, anche all'interno
di una stessa descrizione, così da ottenere un documento descrittivo
complesso ed eterogeneo, che possa essere funzionale alla descrizione,
alla catalogazione, alla ricerca e all'individuazione di risorse
informative differenti per natura, per tipologia e per contesto d'uso.

Difficile realizzare una rassegna sistematica dei diversi modelli
tecnici di metadatazione e individuare tra essi quelli che abbiano
caratteristiche tali da assurgere ad effettivi standard; si segnalano
quindi, senza alcuna pretesa di esaustività:

\begin{enumerate}
\def\labelenumi{\arabic{enumi}.}
\item
  \emph{Metadata Object Description Schema} (MODS): schema di
  \emph{metadati} descrittivi derivato dallo standard bibliografico MARC
  e indicato per la sua flessibilità a descrivere oggetti digitali
  nativi, con un livello di granularità ampiamente compatibile con gli
  standard dei formati bibliografici;
\item
  \emph{Metadata Encoding and Trasmission Standard} (METS): standard per
  la codifica dei \emph{metadati} descrittivi, amministrativi e
  strutturali di un oggetto digitale. Un documento METS è caratterizzato
  da ampia compatibilità con gli altri schemi di \emph{metadati} e ciò
  rende questo standard particolarmente flessibile e applicabile in
  diversi contesti;
\item
  \emph{Metadati Amministrativi e Gestionali} (MAG): il progetto sotteso
  è finalizzato all'elaborazione, al mantenimento e all'evoluzione di
  uno standard italiano di \emph{metadati} amministrativi e gestionali.
  Il profilo applicativo MAG è stato adottato dall'Istituto centrale per
  gli archivi (ICAR) in qualità di standard nazionale sperimentato per
  la gestione, conservazione e diffusione via Web delle riproduzioni
  digitali dei documenti d'archivio. Il comitato MAG, inoltre, nelle
  fasi di studio ed evoluzione ha destinato grande attenzione alla
  compatibilità con il modello METS;
\item
  \emph{Open Archival Information System} (OAIS, standard ISO 14721): il
  modello prevede che le informazioni vengano organizzate per componenti
  funzionali e pacchetti informativi, individuando almeno cinque
  categorie di elementi informativi finalizzati a documentare e
  realizzare la conservazione a medio e lungo termine e la relativa
  accessibilità;
\item
  \emph{Preservation Metadata: Implementation Strategies} (PREMIS): si
  tratta dello standard più utilizzato per i \emph{metadati} di
  conservazione; tra i suoi punti di forza si annoverano la
  facilitazione nello scambio di oggetti digitali tra strutture di
  conservazione e l'integrazione con modelli non-PREMIS ottenibile con
  l'uso dei cosiddetti contenitori di estensione.
\end{enumerate}

Lo stretto, e più volte richiamato, legame esistente tra l'oggetto
digitale e i \emph{metadati} che ne rappresentano le caratteristiche,
nonché il ruolo che i \emph{metadati} svolgono nelle diverse fasi di
vita della risorsa condizionano la qualità complessiva di un prodotto
digitale e la sua capacità di rispondere nel tempo alle finalità che ne
hanno determinato la creazione. La scelta degli standard di
\emph{metadati} da utilizzare diventa poi determinante quando si
vogliano integrare collezioni di risorse con sistemi nazionali o
internazionali. Pertanto «i metadati sono al contempo il mastice che
tiene insieme le informazioni sugli oggetti digitali, garantendone la
qualità e l'accesso, e la struttura logica che consente la creazione e
produzione di nuove e inedite relazioni tra i dati» (Crupi, 2015).

\section*{Bibliografia Consigliata}
{\parindent0pt 
Crupi G., \emph{Bibloteca digitale}, in Solimine G. - Weston P.G. (a
cura di), \emph{Biblioteche e biblioteconomia. Principi e questioni},
Roma, Carocci, 2015, pp. 373-417

Feliciati P., \emph{Gestione e conservazione di dati e metadati per gli
archivi: quali standard?}, in Pigliapoco S. (a cura di),
\emph{Conservare il digitale}, Macerata, EUM, 2010, pp. 191-219

Feliciati P., \emph{I metadati nel ciclo di vita dell'archivio digitale
e l'adozione del modello PREMIS nel contesto applicativo nazionale}, in
Bonfiglio Dosio G. - Pigliapoco S. (a cura di), \emph{Formazione,
gestione e conservazione degli archivi digitali. Il Master FGCAD
dell'Università degli Studi di Macerata}, Macerata, EUM, 2015, pp.
189-208

Franzese P., \emph{Manuale di archivistica italiana}, Perugia, Morlacchi
Editore U.P., 2014

Guercio M., \emph{Archivistica informatica: i documenti in ambiente
digitale}, Roma, Carocci, 2013

Guercio M., \emph{Conservare il digitale. Principi, metodi e procedure
per la conservazione a lungo termine di documenti digitali}, Roma-Bari,
Laterza, 2013

Guercio M., \emph{La conservazione delle memorie digitali}, in Solimine
G. - Weston P.G. (a cura di), \emph{Biblioteche e biblioteconomia.
Principi e questioni}, Roma, Carocci, 2015, pp. 545-566

Michetti G., \emph{Gli standard per la gestione documentale}, in Giuva
L. - Guercio M., \emph{Archivistica. Teorie, metodi, pratiche}, Roma,
Carocci, 2014, pp. 263-286

Pigliapoco S., \emph{Progetto archivio digitale. Metodologia sistemi
professionalità}, Lucca, Civita editoriale, 2016

Weston P.G. - Sardo L., \emph{Metadati}, Roma, Associazione Italiana
Biblioteche, 2017
}

\section*{Sitografia}
{\parindent0pt 
\emph{Data Dictionary for preservation Metadata}, 3, 2015,
\url{http://www.loc.gov/standards/premis/v3/premis-3-0-final.pdf}

\emph{Linee guida per la digitalizzazione e metadati}, in \emph{ICCU --
Istituto Centrale per il Catalogo Unico, delle Biblioteche italiane e
per le informazioni bibliografiche},
\url{http://www.iccu.sbn.it/opencms/opencms/it/main/standard/metadati}
}

\emph{METS -- Metadata Encoding \& Transmission Standard},

\url{http://www.loc.gov/standards/mets/METSita.html}

Pierazzo E. (a cura di), \emph{MAG -- Metadati Amministrativi e
Gestionali. Manuale Utente},

\url{http://www.iccu.sbn.it/upload/documenti/Manuale.pdf}

\emph{Metadata Object Description Schema,}
\url{http://www.loc.gov/standards/mods/}

*\emph{OAIS Reference Model}, in \emph{Conservazione Digitale. Centro di
eccellenza italiano sulla Conservazione Digitale},
\url{http://www.conservazionedigitale.org/wp/approfondimenti/depositi-di-conservazione/oais-reference-model}
}

\hrulefill 

{[}\emph{Concetta Damiani}{]}





\chapter{N-gramma}

\emph{N-gramma} è una nozione matematica concettualmente semplice: data
una sequenza ordinata di elementi, un \emph{n-gramma} ne rappresenta una
sottosequenza di n elementi. Ma le applicazioni di questo concetto
matematico, da cui è possibile elaborare modelli (\emph{n-gram models}),
sono straordinarie ed afferiscono a campi anche molto differenti, come
la statistica, la teoria della comunicazione, la linguistica
computazionale (settore interdisciplinare prima ancora che `scienza' a
sé: cfr. Tamburini 2008) \emph{et alia}. In particolare, in campo
linguistico, un \emph{n-gramma} è una sequenza di due (bi-gramma), tre
(tri-gramma) o più (n-gramma) parole-chiave presenti all'interno di un
dato, specifico contesto; più semplicemente, si potrebbe affermare che
gli \emph{n-grammi} sono `gruppi' di parole che compaiono insieme in un
testo.

L'uso degli \emph{n-grammi} si rivela molto utile, ad esempio, nel
calcolo delle co-occorrenze, il che -- in altri termini -- consente di
lavorare sulla intertestualità. Il calcolo delle co-occorrenze tramite
\emph{n-grammi} non tiene conto dei seguenti elementi: 1. presenza di
segni di interpunzione; \emph{stop-words} (cioè parole semanticamente
vuote e o parole che, essendo comuni, se isolate, non restituiscono un
particolare significato); 3. \emph{ordo verborum}. Per fornire un
esempio concreto, che coniughi il concetto di \emph{n-grams} e le
digital humanities, basterà chiamare in causa, ancora una volta, quello
straordinario strumento che è il TLG, che dispone della specifica,
interessante funzione \emph{N-grams} (cfr. Battaglino 2019): si tratta
di una nuova funzione (un update del nuovo sito del TLG), grazie alla
quale il TLG, `lavorando' sostanzialmente sui trigrammi (eccezion fatta
per le opere di cui possediamo solo frammenti, per i quali il TLG lavora
sui bigrammi, in considerazione della loro \emph{brevitas}), consente di
individuare co-occorrenze. Le co-occorrenze possono essere individuate
(con evidenti, significative ricadute per gli studi letterari,
filologici, lessicografici): 1. tra due testi di un medesimo autore; 2.
tra due edizioni critiche di uno stesso testo (ove disponibili e
digitalizzate, naturalmente); 3. tra testi di due autori diversi, con
diverse possibilità di interrogazione: a. un testo per ciascuno dei due
autori; b. un testo per il primo autore, tutti i testi (digitalizzati)
per il secondo autore; c. tutti i testi (digitalizzati) per il primo
autore, un testo per il secondo autore; d. tutti i testi (digitalizzati)
di entrambi gli autori in questione. Oltre alla specifica funzione
\emph{N-Grams}, il TLG consente di `sfruttare' gli \emph{N-Grams} anche
nell'àmbito della funzione `Browse': ciò si rivela ancor più
interessante sul piano dello studio dell'intertestualità, giacché
consente, a partire da un dato testo di individuare i paralleli tra esso
e l'intero \emph{corpus} del TLG (nel caso di accesso al \emph{full
corpus}).

Tra le numerose applicazioni linguistico-filologiche degli
\emph{n-grammi}, va ricordata anche la possibilità di adoperarli per
l'attribuzione della paternità testuale di uno specifico testo, ad
esempio studiando la presenza, la frequenza e la distribuzione di
specifici \emph{n-grammi} accuratamente individuati e selezionati (cfr.,
a tal proposito, per la letteratura italiana: Basile-Lana 2008; Basile
\emph{et alii} 2008; Boschetti 2018; per la letteratura greca:
Gorman-Gorman 2016; per una panoramica generale e concisa: Battaglino
2018). Gli \emph{n-grammi} possono essere utilizzati anche per
determinare il `peso' culturale di uno specifico periodo su un dato
autore: ciò è reso possibile dal fatto che le idee sono espresse da
sequenze di parole, che possono essere trattate come \emph{n-grammi}. A
tal proposito, si rimanda all'interessante e recente contributo di
Knight-Tabrizi (2016), i quali, tra l'altro, fanno notare che, a seguito
della vasta opera di digitalizzazione (cfr. Caterino 2013), «Google has
created a database of~\emph{n}-grams extracted from these digitized
books and has made the database available to researchers online. This is
the first time ever that such an extensive repository of cultural data
has been made available».

\section*{Bibliografia Consigliata}
{\parindent0pt 
Basile C. - Benedetto D. - Caglioti E. - Degli Esposti M.,~\emph{An
example of mathematical authorship attribution}, in \emph{Journal of
Mathematical Physics}, 49, 2008, pp. 2-20,
\url{https://www.researchgate.net/publication/228663468_An_example_of_mathematical_authorship_attribution}

Basile C. - Lana M.,~\emph{L'attribuzione di testi con metodi
quantitativi: riconoscimento di testi gramsciani}, in \emph{AIDA
Informazioni} , 1-2, 2008, pp. 165-183,
\url{https://www.academia.edu/6604207/Lattribuzione_di_testi_con_metodi_quantitativi_riconoscimento_di_testi_gramsciani}

Battaglino G., \emph{La tessitura matematica dei testi. Filologia e
metodi matematico-statistici: l'auspicabile} σύγκρισις \emph{tra metodi
qualitativi e metodi quantitativi}, in \emph{FRI -- Filologia Risorse
Informatiche} (\emph{Carnet de recherche and online journal -- Italian
Digital Humanities}),
\url{https://fri.hypotheses.org/915}

Battaglino G., \emph{Note minime sul} TLG\emph{: brevi cenni sulle
`origini' del} TLG \emph{e piccolo} vademecum, in \emph{FRI -- Filologia
Risorse Informatiche} (\emph{Carnet de recherche and online journal --
Italian Digital Humanities}),
\url{https://fri.hypotheses.org/1391}

Boschetti F.,~\emph{Copisti digitali e filologi computazionali}, Roma,
CNR Edizioni, 2018,
\url{http://eprints.bice.rm.cnr.it/17545/1/bookBoschetti2018.pdf} 

Caterino A.F., \emph{Note minime su} Google Books, in \emph{FRI --
Filologia Risorse Informatiche} (\emph{Carnet de recherche and online
journal -- Italian Digital Humanities}),
\url{https://fri.hypotheses.org/128}

Gorman V. B. - Gorman R. J., \emph{Approaching Questions of the Text
Reuse in Ancient Greek using Computational Syntactic Stylometry}, in
\emph{Open Linguistic}, 2, 2016, pp. 500-510,
\url{https://www.degruyter.com/downloadpdf/j/opli.2016.2.issue-1/opli-2016-0026/opli-2016-0026.pdf}

Knight G. P. - Tabrizi N., \emph{Using n-Grams to Identify Time Periods
of Cultural Influence}, in \emph{Journal on Computing and Cultural
Heritage}, 9, 3, 2016, art. n. 15, pp. 1-19,
\url{https://dl.acm.org/citation.cfm?id=2940332}

Tamburini F., \emph{La linguistica computazionale: un crogiolo di
esperienze multidisciplinari}, in \emph{GRISELDAONLINE} , 2008, pp.
1-11,
\url{http://www.griseldaonline.it/informatica/la-linguistica-computazionale-tamburini.html}
}

\section*{Sitografia}
{\parindent0pt 
*\emph{Google Ngram Viewer} (\emph{s.v.}), in \emph{Wikipedia} (EN),
\url{https://en.wikipedia.org/wiki/Google_Ngram_Viewer}

*\emph{Intertestualità} (\emph{s.v.}), in \emph{Treccani},
\url{http://www.treccani.it/vocabolario/intertestualita/}

* \emph{N-gram} (\emph{s.v}.), in \emph{Wikipedia} (EN),
\url{https://en.wikipedia.org/wiki/N-gram}

*\emph{N-gramma} (\emph{s.v.}), in \emph{Wikipedia,}
\url{https://it.wikipedia.org/wiki/N-gramma}

*\emph{Stop words} (\emph{s.v.}), in \emph{Wikipedia} (EN),
\url{https://en.wikipedia.org/wiki/Stop_words}

*\emph{TLG} (\emph{sub-corpus} del) in \emph{open access}: \emph{Canon
of Greek Authors and Works (Abridged TLG}),
\url{http://stephanus.tlg.uci.edu/Iris/canon/csearch.jsp}

*\emph{TLG} (ultima versione digitale del \emph{TLG: Thesaurus Linguae
Graeciae. Digital Library}. Ed. Maria C. Pantelia. University of
California, Irvine),
\url{http://stephanus.tlg.uci.edu/}

*\emph{Thesaurus Linguae Graeciae} (\emph{s.v.}), in \emph{Wikipedia},
\url{https://en.wikipedia.org/wiki/Thesaurus_Linguae_Graecae}

}

\hrulefill 

{[}\emph{Giovanna Battaglino}{]}

\chapter{Oggetto digitale}

Si definisce \emph{oggetto digitale} qualsiasi testo, documento,
immagine, audio/video, ipertesto o base di dati che venga archiviato in
una memoria di massa attraverso una codifica che lo renda comprensibile
alla macchina e restituibile all'utente umano. La codifica avviene, a
causa della sua natura informatica, attraverso il codice binario e un
linguaggio (Unicode o ASCII tra i più usati) che rendano l'informazione
intellegibile alla macchina. Sua caratteristica peculiare è di essere
un'entità composta in modo inseparabile da uno o più file di contenuti e
i loro metadati uniti, fisicamente e/o logicamente tramite l'uso di un
`digital wrapper', un connettore digitale. Si tratta quindi, in
definitiva, dell'associazione del dato e del suo metadato che concorrono
a rendere quella sequenza di bit, comprensibile dalla macchina,
significante, individuabile e accessibile per la fruizione,
l'archiviazione, la conservazione, la disseminazione e le altre
operazioni gestionali. L'\emph{oggetto digitale} può nascere
direttamente digitale o aver subito una trasformazione in formato
digitale. A sua volta questa entità composita può essere oggetto di un
ulteriore livello di specificazione che la distingue in `oggetti
digitali semplici' e `complessi'\emph{.} Secondo il glossario della
California Digital Library, i primi sono composti da un unico file di
contenuto (e dalle sue varianti di formato e forme derivate) e dai suoi
metadati mentre secondi sono composti da due o più file di contenuto (e
dalle loro varianti di formato e forme derivate) e dai metadati
corrispondenti.

L'\emph{oggetto digitale} possiede inoltre come caratteristica la
difficoltà di conservazione per diversi motivi, due in particolare: il
degrado del supporto fisico (floppy disc e CD ad esempio) e
l'obsolescenza degli standard di codifica utilizzati all'atto della
creazione dell'oggetto digitale. È per questi due motivi che
l'\emph{oggetto} \emph{digitale} deve essere affrontato come: `oggetto
fisico'\emph{, `}oggetto logico'\emph{, `}oggetto concettuale'\emph{,
`}collezione di elementi'. Tutti questi aspetti devono essere tenuti in
considerazione all'atto di creazione dell'\emph{oggetto digitale} perché
tutti questi devono essere oggetto della conservazione sia singolarmente
che in relazione tra loro.

Da questa sommaria descrizione si comprende che, parlando di
\emph{oggetto digitale}, non ci si riferisce a un elemento definito ma,
più precisamente ad una categoria di oggetti che possono presentare
caratteristiche molto diverse fra loro ma che hanno un unico punto in
comune, la necessità di essere archiviate attraverso l'impiego di
memorie di massa.

\section*{Bibliografia Consigliata}
{\parindent0pt 
Tomasi F., \emph{Metodologie informatiche e discipline umanistiche},
Roma, Carocci, 2008

Numerico T.- Vespignani A., \emph{Informatica per le scienze
umanistiche}, Bologna, Il Mulino, 2003

Fiormonte D. - Numerico T. - Tomasi F., \emph{L'umanista digitale},
Bologna, Il Mulino, 2010

Sebastiani M., \emph{Il ``documento digitale'': analisi di un concetto
in evoluzione}, in \emph{DigItalia}. \emph{Rivista del digitale nel beni
culturali}, 1, 2008, pp. 9-31
}

\section*{Sitografia}
{\parindent0pt 
*\emph{Digital Wrapper}, in \emph{CDL -- California Digital Library's
Glossary},
\url{http://www.cdlib.org/gateways/technology/glossary.html?field=glossary\&action=search\&query=oac\#d}

Pierazzo E. (a cura di), \emph{MAG -- Metadati Amministrativi e
Gestionali. Manuale Utente},

\url{http://www.iccu.sbn.it/upload/documenti/Manuale.pdf}
}

\hrulefill 

{[}\emph{Antonio Marson Franchini}{]}



\chapter{Ontologia}

(Ingl. \emph{Ontology,} fr. \emph{Ontologie})

\emph{Ontologia} è un termine mutuato dalla filosofia, che in
informatica assume il significato di `rappresentazione di una forma di
conoscenza o parte del mondo' da modellare in un programma (dominio).
Nei sistemi di condivisione della conoscenza, l'\emph{ontologia} ha il
compito di rappresentare `le cose che esistono' in un determinato
dominio concettuale, distinguendo i concetti (concettualizzazione),
definendoli e indicando il sistema di relazioni entro le quali
interagiscono (specificazione). Le \emph{ontologie} sono importanti per
il Web Semantico, in quanto permettono di cercare i concetti piuttosto
che stringhe dei caratteri e perché creano una relazione tra il
contenuto e altre fonti di conoscenza. Il linguaggio solitamente
utilizzato è OWL (\emph{Ontology Web Language}), basato sulla sintassi
XML.

Esempio: Clavius (\emph{Istituto di Linguistica Computazionale Zampolli
di Pisa}) un insieme di manoscritti digitalizzati conservati presso
l'Archivio Storico della Pontificia Università Gregoriana, relativi a
Christophorus Clavius (1538-1612), matematico ed astronomo gesuita. I
concetti (terminologia ed entità di dominio) formano un'\emph{ontologia}
legata a risorse già disponibili in rete.

\hrulefill 

{[}\emph{Flavia Sciolette}{]}



\chapter{OPAC}

Il trasferimento dei cataloghi cartacei in banche dati digitali, a metà
degli anni Ottanta, ha comportato la diffusione su larga scala degli
\emph{OPAC}, acronimo di \emph{Online public access catalogue} (o
catalogo in linea ad accesso pubblico). Se le prime biblioteche
digitalizzate erano consultabili soltanto tramite `telnet' -- un
protocollo di rete molto diffuso prima dell'espansione di Internet --,
oggi sono accessibili direttamente dal Web.

La consultazione di un \emph{OPAC} avviene mediante la compilazione di
uno o più campi di ricerca -- ad esempio l'opera e/o l'autore -- o
l'inserimento di una o più parole chiavi che consentono l'accesso
all'elenco dei testi contenuti nella banca dati della biblioteca,
informando l'utente sul tipo di documento, sull'anno di pubblicazione,
sulla sua collocazione, etc.

Si distinguono varie tipologie di \emph{OPAC}:

\begin{enumerate}
\def\labelenumi{\arabic{enumi}.}
\item
  \emph{OPAC} di singole biblioteche, che occorre consultare mediante il
  link della biblioteca;
\item
  \emph{OPAC} collettivi, ossia cataloghi online che contengono i
  database di più biblioteche (contemporaneamente consultabili) e
  offrono l'enorme vantaggio di catalogare una volta sola un documento
  sì che, mediante un'unica descrizione, si ha accesso alle diverse
  biblioteche che lo possiedono;
\item
  multi -- e meta -- \emph{OPAC}, che interrogano più cataloghi online
  già esistenti in forma indipendente: i multiopac sono interrogabili
  uno per volta, anche se attraverso la stessa interfaccia; i~metaopac
  sono in grado di interrogare più \emph{OPAC} mediante una sola
  richiesta, e di restituire i risultati dei diversi \emph{OPAC}
  coperti. Un esempio italiano di metaopac è il MAI (\emph{Metaopac
  Azalai Italiano}),
  \url{http://www.aib.it/progetti/opac-italiani/mai-ricerca-globale/}.
\end{enumerate}

Per facilitare la ricerca alcune pagine Web raccolgono link di
\emph{OPAC}, sia italiani che stranieri (ad esempio il \emph{Gateway to
Library Catalogs}:
\url{https://www.loc.gov/z3950/gateway.html}),
nei quali sono contenuti un gran numero di cataloghi. Tra questi vi è
l'italiano \emph{OPAC SBN} (\emph{Servizio Bibliotecario Nazionale}),
ossia una rete di biblioteche italiane promossa dal MiBAC, dalle Regioni
e dalle Università, organizzata dall'ICCU (\emph{Istituto Centrale per
il Catalogo Unico}), che interroga più biblioteche contemporaneamente
(cfr. OPAC collettivi).

I vantaggi ottenuti dall'introduzione degli \emph{OPAC} sono dunque
evidenti. Essi:

\begin{enumerate}
\def\labelenumi{\arabic{enumi}.}
\item
  aiutano a trovare rapidamente i documenti presenti nelle biblioteche;
\item
  consentono di conoscere la disponibilità di un documento in tempo
  reale;
\item
  permettono la prenotazione di un libro;
\item
  consentono l'accesso a versioni elettroniche di documenti cartacei,
  nonché a scansioni fotografiche, etc.
\end{enumerate}

È indubbio che tali condizioni semplificano l'indagine e ottimizzano i
tempi della ricerca; consentono l'accesso a una banca dati di enormi
dimensioni e il reperimento di testi anche di limitata diffusione,
conferendo all'\emph{OPAC} un ruolo di imprescindibile utilità e
promuovendolo a necessario strumento delle Digital Humanities.

\section*{Bibliografia Consigliata}
{\parindent0pt 
Dhanavandan S. - Isabella Mary A., \emph{Online public access catalague
(OPAC)}, New Delhi, Write and Print publications, 2015

IFLA, \emph{Guidelines for online public access catalogue (OPAC)
displays}, in \emph{Series on Bibliographic Control}, 27, 2005

Nelson Bonnie R., \emph{Opac Directory 1998: a guide to
Internet-Accessible online public access catalogs}, Medford (NJ),
Information Today Inc., 1997

Tramullas J. - Garrido P., \emph{Library automation and OPAC 2.0:
information access and services in the 2.0 landscape}, Hershey (PA),
Information Science Reference, 2012
}

\section*{Sitografia}
{\parindent0pt 
\emph{AIB Web}, \emph{Il Web dell'Associazione Italiana Biblioteche}:
\emph{MAI},
\url{http://www.aib.it/progetti/opac-italiani/mai-ricerca-globale/}

\emph{Gateway to Library Catalogs},
\url{https://www.loc.gov/z3950/gateway.html}

Gnoli Claudio, \emph{Opac in Italia: una panoramica delle tipologie e
delle modalità di consultazione},
\url{http://www.aib.it/aib/sezioni/emr/bibtime/num-ii-1/gnoli.htm\#nota1}

*\emph{OPAC} (\emph{s.v}.), in \emph{Wikipedia},
\url{https://it.wikipedia.org/wiki/OPAC}

\emph{OPAC SBN}, \emph{Catalogo del Servizio Bibliotecario Nazionale},
\url{https://opac.sbn.it/opacsbn/opac/iccu/informazioni.jsp}

*\emph{Telnet} (\emph{s.v}.), in \emph{Wikipedia},
\url{https://it.wikipedia.org/wiki/Telnet}
}

\hrulefill 

{[}\emph{Alessandra Di Meglio}{]}




\chapter{Open Access}

\emph{Open Access} (OA) è il termine comunemente utilizzato per indicare
un movimento che promuove la libera circolazione e l'uso non restrittivo
dei risultati della ricerca e del sapere scientifico. Scopo primario
dell'\emph{OA} è riguadagnare possesso della comunicazione scientifica,
offrendo libero accesso ai risultati della ricerca.

La letteratura \emph{Open Access} (OA) è digitale, online, gratuita e
libera da alcune restrizioni dettate dalle licenze per i diritti di
sfruttamento commerciale. Queste condizioni sono possibili grazie ad
Internet e al consenso dell'autore o del titolare dei diritti d'autore.

La definizione emerge dalle risultanze di tre conferenze che hanno avuto
luogo tra il 2002 e il 2003:

• Berlin Declaration on Open Access to Knowledge in the Sciences and
Humanities;

• Bethesda Statement on Open Access Publishing;

• Budapest Open Access Initiative.

Nel 2004 la \emph{Conferenza dei Rettori delle Università Italiane}
(CRUI) ha formalizzato l'adesione alla \emph{Dichiarazione di Berlino}
con la \emph{Dichiarazione di Messina,} Documento italiano a sostegno
della Dichiarazione di Berlino sull'accesso aperto alla letteratura
accademica.

Come facilmente intuibile, la questione economica non rappresenta
l'aspetto di maggior rilevanza per la comunità scientifica
internazionale. Il concetto di apertura, infatti, non si limita alla
gratuità dell'accesso ma richiama questioni di tipo etico, come la
libertà di uso e riuso dei contenuti pubblicati e dei prodotti della
ricerca e questioni di effettiva condivisione, auspicando alla
realizzazione di «uno spazio globale, multidisciplinare e perfino oltre
i confini di università e laboratori, includendo tutti i cittadini».

\section*{Bibliografia Consigliata}
{\parindent0pt 
Castellucci P., \emph{Carte del nuovo mondo: banche dati e open access},
Bologna, Il Mulino, 2017

Di Donato F., \emph{La scienza e la rete. L'uso pubblico della ragione
nell'età del web}, Firenze, Firenze University Press, 2009
}

\section*{Sitografia}
{\parindent0pt 
Suber P., \emph{Breve introduzione all'accesso aperto}, traduzione
italiana di Susanna Mornati, 2004,
\url{http://www.iss.it/binary/bibl/cont/Suber.1110468892.pdf}

}

\hrulefill 
 
{[}\emph{Concetta Damiani}{]}




\chapter{Paratesto}

\emph{Definizione}

Il concetto di \emph{paratesto} appartiene alla teoria della letteratura
e fu introdotto dal critico francese Gérard Genette. Secondo il
Vocabolario della Lingua Italiana Garzanti il \emph{paratesto} è
l'insieme degli elementi accessori e complementari di un testo a stampa,
come il titolo, l'introduzione, le note, ed anche l'insieme delle
caratteristiche editoriali e tipografiche. L'Enciclopedia Treccani
on-line aggiunge che il termine deriva dal francese \emph{paratexte} ed
indica l'insieme di produzioni, verbali e non verbali, sia nell'ambito
del volume (il nome dell'autore, il titolo, una o più prefazioni, le
illustrazioni, i titoli dei capitoli, le note), sia all'esterno del
libro (interviste, conversazioni, corrispondenze, diari), che
accompagnano il testo vero e proprio e ne guidano il gradimento e,
aggiungiamo, la comprensione del pubblico.

Genette teorizza la nozione di \emph{paratesto} in \emph{Seuils}, Paris,
Seuil, 1987 (edizione italiana a cura di Camilla Maria Cederna, Soglie.
I dintorni del testo, Torino, Einaudi, 1989).

\emph{Etimologia}

Secondo il Vocabolario on-line Treccani il termine \emph{paratesto} è
formato dalla preposizione greca \emph{para-} (`vicino', `affine',
`contrapposto') e dal sostantivo \emph{testo} (dal latino \emph{textus},
tessuto, da \emph{texere}, tessere, intrecciare) sul modello del
francese \emph{paratexte}.

\emph{Parti}

Il genere \emph{paratesto} viene suddiviso in base alla sua ubicazione
in \emph{peritesto} ed \emph{epitesto}.

\emph{Peritesto}

Il \emph{peritesto} si trova nello spazio dell'opera, con una funzione
paratestuale quasi esclusivamente di presentazione, di indirizzo e di
commento. È il nucleo del \emph{paratesto}, ha forma e, generalmente,
posizioni fisse: all'inizio (frontespizi, titoli, dediche, epigrafi,
prefazioni), in margine (note, chiose) e alla fine del testo
(postfazioni, tavole, appendici).

Appartengono al genere \emph{peritesto} anche gli elementi direttamente
dipendenti dall'editor\href{https://it.wikipedia.org/wiki/Editore}{e}:
il formato e la composizione grafica, la collana, la copertina, le
pagine bianche del volume all'inizio e alla fine del libro (sguardie o
risguardi: separano il blocco delle pagine dalla copertina e lo tengono
ad essa legato), l'occhiello (pagina che riporta il titolo del libro, se
posta prima del frontespizio, o il titolo di un capitolo o simili, se
posta all'interno), le pagine 4-5-6 con le ulteriori indicazioni
editoriali, il frontespizio (pagina all'inizio del libro, nella quale
sono indicati l'autore, il titolo, le note tipografiche), il
\emph{colophon} nelle pagine finali (nei libri moderni la formula
`finito di stampare' con i dati d'obbligo: la data, il luogo di stampa,
il nome dello stampatore e altre notizie simili alla fine dell'opera),
la composizione tipografica, la qualità della carta. Vengono
classificati come \emph{peritesti} anche il nome dell'autore, il titolo
dell'opera, le dediche e le epistole dedicatorie, le epigrafi (le
citazioni in margine al testo, per esempio all'inizio del volume o dei
capitoli), le prefazioni e le postfazioni, gli intertitoli (i titoli di
capitoli o tomi o sezioni), le note. Infine, è considerato
\emph{peritesto} il cosiddetto \emph{prière d'insérer}, \emph{paratesto}
della consuetudine editoriale francese, un breve prospetto dell'opera,
spedito ai direttori dei giornali allo scopo di pubblicarlo, una specie
di pubblicità, e poi evolutosi in varie forme, fino all'odierno
riassunto nella quarta pagina di copertina.

\emph{Epitesto}

L'\emph{epitesto} è qualsiasi elemento paratestuale non contiguo, ma
comunque in relazione al testo: tutto quello che un autore dice o scrive
nella sua vita o sul mondo che lo circonda può avere un carattere
paratestuale.

L'\emph{epitesto} si distingue in pubblico e privato. Appartengono al
primo genere l'\emph{epitesto} editoriale con funzione promozionale
(pubblicità, inserzioni), l'allografo ufficioso (recensioni, interventi
critici), l'\emph{epitesto} autonomo (le risposte e i commenti pubblici
dell'autore) o mediatizzato (interviste, convegni, occasioni pubbliche
di rapporto con l'autore).

L'\emph{epitesto} privato è rivolto ad una persona, che si interpone tra
l'autore e il pubblico e può essere un corrispondente, un confidente o
anche l'autore stesso. L'\emph{epitesto} privato si suddivide in
confidenziale, quando è rivolto a un confidente, ed intimo, se il
destinatario è l'autore stesso.

\emph{Teoria retorica}

In \emph{Der humanistische Geleittext als Paratext -- am Beispiel von
Brants Beigaben zu Tennglers} Layen \emph{`}Spiegel' Joachim Knape
include nel \emph{paratesto} il testo di presentazione umanistico, che
aumenta il valore del testo introdotto. A nostro parere confermano
questa interpretazione per esempio anche le lettere dedicatorie
erasmiane del \emph{De copia verborum ac rerum}.

In \emph{Powerpoint in rhetoriktheoretischer Sicht} Knape analizza la
produzione del testo per Powerpoint in base alla teoria retorica della
progettazione testuale: il testo di Powerpoint viene considerato come
\emph{paratesto} oppure come testo centrale. In quanto \emph{paratesto},
funge da compendio (Kondensat), da complemento (Korrelat) e da
illustrazione (Illustrat) del testo centrale. Come testo centrale il
testo di Powerpoint rappresenta anche la ricerca di una soluzione a due
canali: acustico, la lettura dell'oratore, ed ottico, la lettura muta
del pubblico.

\section*{Bibliografia Consigliata}
{\parindent0pt 
Knape J., \emph{Der humanistische Geleittext als Paratext -- am Beispiel
von Brants Beigaben zu Tennglers} Layen Spiegel, in Deutsch A. (Hg.),
\emph{Ulrich Tenglers Laienspiegel. Ein Rechtsbuch zwischen Humanismus
und Hexenwahn}, Heidelberg, Blaukreuz-verlag, 2011, pp. 117-138

Knape J., \emph{Powerpoint in rhetoriktheoretischer Sicht}, in Bernt
Schnettler, Knoblauch H. (Hg.), \emph{Powerpoint-Präsentationen. Neue
Formen der gesellschaftlichen Kommunikation von Wissen}, Konstanz, UVK
Verlagsgesellschaft mbH, 2007

Zingarelli N., \emph{Lo Zingarelli 1995. Vocabolario della lingua
italiana di Nicola Zingarelli}, \emph{Dodicesima edizione a cura di
Mirco Dogliotti e Luigi Rosiello}, Bologna, Zanichelli, 1995
}

\section*{Sitografia}
{\parindent0pt 
*\emph{Paratesto} (\emph{s.v}.), in \emph{Garzanti Linguistica},
\url{http://www.garzantilinguistica.it/ricerca/?q=paratesto}

*\emph{Paratesto} (\emph{s.v}.), in \emph{Treccani},
\url{http://www.treccani.it/vocabolario/paratesto/}
}

\hrulefill 

{[}\emph{Cristiano Rocchio}{]}




\chapter{PHP}

Uno dei linguaggi di programmazione maggiormente diffusi e utilizzati
per applicazioni di vario tipo in diversi campi, soprattutto in
combinazione con MySql. Il \emph{PHP} è un linguaggio di scripting, con
cui dunque è possibile realizzare sequenze di comandi per risolvere un
determinato compito, da richiamare con riga di comando da terminale (vd.
Script). Come molti linguaggi di scripting, \emph{PHP} necessita di un
interprete, affinché la macchina comprenda le istruzioni contenute nel
codice in \emph{PHP}. Nel momento in cui installiamo \emph{PHP} dal sito
ufficiale, configuriamo e installiamo anche l'interprete. Un server
locale, come ad esempio Xampp, permette l'installazione dell'interprete,
assieme ad altri strumenti utili per realizzare applicazioni.

L'uso di \emph{PHP} quindi comprende la realizzazione di script a riga
di comando, da terminale (vd. Terminale) di applicazioni lato server
(vd. Lato-server), ovvero comprendenti operazioni, ad esempio, legate a
gestione, trattamento e immagazzinamento di informazioni in un database;
di applicazioni stand-alone, non legate quindi a risorse esterne.

\emph{PHP} è un linguaggio open-source (vd. Open-source), attualmente
alla versione 7.2, con un'ampia quantità di risorse reperibili in rete,
come librerie e script già pronti. Il suo punto di forza è sicuramente
la possibilità di integrazione con molti DBMS; dal punto di vista della
sicurezza, richiede un aggiornamento costante, sebbene le versioni più
recenti abbiano raggiunto un livello estremamente soddisfacente, in
confronto alle precedenti.

Nelle DH, \emph{PHP} è uno dei linguaggi più comuni per la creazione di
interfacce per database di vario tipo. (vd. Database)

Esempio:

Tlion -- \emph{Tradizione della Letteratura Italiana Online}:

Il Tlion -- assieme ad altri DB che ne condividono l'impianto -- è
sviluppato in \emph{PHP} e MySql su server Apache. Si tratta di una
banca dati a schede, catalogate per autore e opera, con notizie relative
alla tradizione dei testi della letteratura italiana e liberamente
consultabile dal sito: http://tlion.sns.it/.

\hrulefill 
 
{[}\emph{Flavia Sciolette}{]}



\chapter{Public History}

{[}assegnata a Concetta Damiani{]}



\chapter{Self-archiving}

Il processo di \emph{Self-archiving} (o autoarchiviazione o Green Road)
consiste nel deposito di documenti digitali~da parte di un autore~in
archivi ad accesso libero (Open Access o OA), per consentirne
liberamente l'utilizzo e/o la consultazione. Il
termine~\emph{self-archiving}~è stato proposto in ambito universitario
da Stevan Harnad~nel 1994, in \emph{Subversive Proposal}, pubblicato nel
1995 per massimizzare e condividere i prodotti della ricerca.

L'autoarchiviazione è eseguita in Open Archives (o E-prints Server o
Data Provider), ossia archivi aperti, che, attraverso la funzione di
harvesting (raccolta), possono essere interrogati da un servizio (il
Service Provider) atto ad indicizzare i metadati raccolti. Suddetti
archivi rientrano nell'OAI (Open Archive Initiative), un registro
preposto al deposito dei documenti scientifici in forma elettronica,
nato per rendere facilmente fruibili gli archivi contenenti documenti
prodotti in ambito accademico e per incoraggiare la loro produzione in
ambito scientifico/universitario.

Gli Open Archives si distinguono in due tipologie:

\begin{enumerate}
\def\labelenumi{\arabic{enumi}.}
\item
  Open Archive istituzionali: l'archivio raccoglie tutti i lavori di un
  ente (università, ente di ricerca, dipartimento) o una parte dei
  lavori che l'ente ritiene di conservare nel deposito. In questo caso i
  materiali raccolti coinvolgono varie discipline.
\item
  Open Archive disciplinari: come ArXiv (contenente in larga parte
  pre-prints di più di mezzo milione di articoli di Fisici) e RePEc
  (Reserch Papers in Economics, la più grande collezione decentrata di
  documenti ad Accesso Libero per l'economia) o PubMed, una banca
  bibliografica relativa alla letteratura scientifica biomedica.
  L'archivio disciplinare raccoglie i lavori in una determinata
  disciplina. Può anche trattarsi di un server di un ente che decide di
  aprire più archivi per discipline differenti. Molto spesso però si
  tratta di più soggetti (enti o anche soggetti individuali, dipende
  dall'organizzazione che si vuole adottare) che interagiscono nei
  Repositories (depositi) di una stessa disciplina o di un argomento
  specifico.
\end{enumerate}

L'elenco di tutti i Repository aperti OA si può consultare su:

\begin{enumerate}
\def\labelenumi{\arabic{enumi}.}
\item
  OpenDOAR (\emph{Directory Access Repositories}): un repertorio di
  oltre 2000 Repository ricercabile per area disciplinare, lingua,
  nazione, software utilizzato, etc.
\item
  ROAR (\emph{Registry od Open Access Repositories}): un registro di
  oltre 2000 Repository ricercabili per nazione, tipo di Repository e
  software;
\item
  PLEIADI: portale per la letteratura scientifica elettronica italiana
  su Archivi Aperti e Depositi Istituzionali.
\end{enumerate}

I documenti digitalizzati e archiviati sono gli e-prints (versione
digitale di un documento di ricerca), ma anche poster scientifici, tesi,
presentazioni, capitoli di libri, etc. Gli e-prints, a seconda dello
stadio editoriale in cui si trovano, si distinguono in:

\begin{itemize}
\item
  Pre-print (o pre-stampa): tipologia di documento, distribuito in modo
  più o meno limitato, relativa ad un lavoro tecnico spesso in forma
  preliminare, precedente la sua pubblicazione in un periodico.
\item
  Post-print (o post-stampa): versione modificata del pre-print, che ha
  passato il comitato editoriale e che è già stata sottoposta a
  refereeing.
\end{itemize}

\begin{itemize}
\item
  Camera ready (o definitivo): l'articolo nella sua ultima versione,
  come viene pubblicato dalla casa editrice.
\end{itemize}

Secondo la politica di molti editori, un ricercatore può auto-archiviare
diverse versioni del proprio documento: la versione pre-print precedente
alla peer review, e la versione post-print che è stata rivista e
accettata per la pubblicazione.

Per essere certi che l'editore a cui l'autore sottopone l'articolo non
vieti l'autoarchiviazione, è bene conoscere la politica editoriale di
archiviazione e consultare la banca dati SHERPA/RoMEO \emph{Publisher
Copyright Policies and Self-Archiving}, che, a seconda che l'editore
consenta o meno l'autorchiviazione, assegna un colore a seconda dei
diritti concessi agli autori:

\begin{itemize}
\item
  bianco: nessun diritto all'autoarchiviazione;
\item
  giallo: diritto di archiviare pre-print;
\item
  blu: diritto di archiviare post-print o una versione PDF dell'editore;
\item
  verde: diritto di archiviare pre-print e post-print e,~a volte, la
  versione elettronica della pubblicazione finale, prodotta dalla casa
  editrice.
\end{itemize}

Il colore assegnato da SHERPA costituisce solo una catalogazione
preliminare, potendo le case editrici imporre particolari condizioni a
loro discrezione.

I ricercatori usano diversi Academic social networks, come ResearchGate
(\url{https://www.researchgate.net/about})
e Academia.edu
(\url{https://www.academia.edu/about}),
principalmente utilizzati per condividere documenti, monitorare il loro
impatto e seguire i progressi della ricerca, selezionando un'area di
interesse ed esplorando i profili di utenti con interessi affini.

Dalla vendita degli articoli pubblicati in riviste scientifiche, i
ricercatori non traggono `guadagni di pubblicazione' ma ottengono
`guadagni di impatto', se gli articoli sono diffusi in modo adeguato.
L'obiettivo, infatti, è di mostrare i loro lavori al maggior numero di
persone, indipendentemente dal guadagno economico personale, e la
pubblicazione online dei loro contributi consente tale visibilità.

Come dice Stevan Harnad in \emph{Subversive} \emph{Proposal}: «i costi
elevati nell'era cartacea di Gutenberg, dispendiosa e inefficace, erano
inevitabili; ma oggi, nell'era post-Gutenberg on line, il funzionamento
alla vecchia maniera, con i suoi costi elevati deve essere mantenuto
come opzione complementare invece che come strumento indispensabile».

\section*{Bibliografia Consigliata}
{\parindent0pt 
Delle Donne R., \emph{Studi e ricerche di scienze umane e sociali},
Napoli, fedOA Press, 2014

De Robbio A., \emph{Archivi aperti e comunicazione scientifica}, Napoli,
ClioPress, 2007
}

\section*{Sitografia}
{\parindent0pt 
Antelman K., \emph{Self-archiving practice and the influence of
publisher policies in the social sciences},
\url{https://repository.lib.ncsu.edu/bitstream/handle/1840.2/83/antelman_self-archiving.pdf?sequence=1\&isAllowed=y}

*\emph{Autoarchiviazione} (\emph{s.v}.), in \emph{Wikipedia}
\url{https://it.wikipedia.org/wiki/Autoarchiviazione}

Brody T. \emph{et} \emph{al}., \emph{The effect of Open Access on
citation impact},
\url{http://opcit.eprints.org/feb19oa/brody-impact.pdf}

De Robbio A., \emph{Auto-archiviazione per la ricerca: problemi aperti e
sviluppi futuri},
\url{http://eprints.rclis.org/4096/}

Eberechukwu Eze M. - Chukwuma Okeji C. - Ejiobi Bosah G.,
\emph{Self-Archiving options on social networks: a review of options},
\url{https://www.researchgate.net/publication/328771143_Self-archiving_options_on_social_networks_a_review_of_options}

Gadd E. A. -- Troll Covey D., \emph{What does ``green'' open access
mean? Tracking twelve years of changes to journal publisher
self-archiving policies},
\url{https://dspace.lboro.ac.uk/dspace-jspui/bitstream/2134/21555/5/Article\%20-\%20What\%20does\%20green\%20mean\%20v6\%20Submitted\%20\%20reformatted\%20for\%20IR.pdf}

Harnad S., \emph{Self-Archive Unto Others as Ye Would Have them
Self-Archive Unto You}, 
\url{https://jcom.sissa.it/sites/default/files/documents/jcom0203\%282003\%29F03.pdf}

Jenkins C. \emph{et} \emph{al}., \emph{RoMEO Studies 8: Self-archiving
-- when Yellow and Blue make Green: the logic behind the colour-coding
in the Copyright Knowledge Bank},
\url{https://www.researchgate.net/publication/28693015_RoMEO_Studies_8_Self-archiving_The_logic_behind_the_colour-coding_used_in_the_Copyright_Knowledge_Bank}

Mallikarjun D. - Bulu M., \emph{Driving on the Green Road: Self --
Archiving Research for Open Access in India},
\url{https://www.questia.com/library/journal/1G1-331807690/driving-on-the-green-road-self-archiving-research}

*\emph{Self-Archiving} (\emph{s.v}.), in \emph{Wikipedia}
\url{https://en.wikipedia.org/wiki/Self-archiving}

Swan A. - Brown S., \emph{Open access self-archiving: an author study},
\url{https://eprints.soton.ac.uk/260999/1/jisc2.pdf}

}

\hrulefill 
 
{[}\emph{Alessandra Di Meglio}{]}




\chapter{SGML}

(ingl. Acronimo per \emph{Standard Generalized Markup Language})

È il progenitore dei `markup languages' utilizzati attualmente. Charles
Goldfarb inventò GML (\emph{Generalized Markup Language}) nel 1969
durante il suo lavoro per la IBM sotto la supervisione di Steve Furth, e
ne definì i presupposti teorici: «This analysis of the markup process
suggest that it should be possible to design a generalized markup
language so that markup would be useful for more than one application or
computer system. {[}\ldots{}{]} This could be done, for example, with
mnemonic ``tags''. The designation of a component as being of a
particular type would mean only that it will be processed identically to
other components of that type».

Goldfarb afferma e struttura, quindi, quelli che saranno le peculiarità
di tutti i linguaggi di codifica basati su GML: i markup devono essere,
prima di tutto, applicabili a differenti sistemi e piattaforme
informatiche e dovranno poter essere inserite all'interno del testo,
mentre tutte le istruzioni procedurali verranno inserite in un documento
a sé stante. Proprio per queste peculiarità GML diventò standard nel
1986, anche se per l'eccessiva complessità della codifica non viene più
utilizzata, in quanto si preferiscono codifiche più snelle come HTML e
XML.

\emph{SGML} venne utilizzato in molti progetti, il primo fu quello
sviluppato dall'\emph{American Association of Publishers} (AAP):
selezionarono tre tipi di testo dal vasto campo dell'editoria, un libro,
una serie di pubblicazioni e un articolo, e ne definirono le rispettive
\emph{Document Type Description} (DTD).

All'interno del documento troviamo tre elementi fondamentali:

\begin{enumerate}
\def\labelenumi{\arabic{enumi}.}
\item
  La dichiarazione \emph{SGML}, che va sempre posta come apertura di un
  nuovo documento, poiché ne definisce letteralmente il tipo. Dichiara,
  senza troppi giri di parole, che quel documento sarà codificato
  secondo le norme del \emph{SGML}.
\item
  La DTD, necessaria in ogni documento. Essa è un insieme di stringhe di
  codice che esplicitano elementi e attributi e sistemano le relazioni
  tra di loro: è una sorta di vocabolario per la codifica del documento.
\item
  L'istanza del documento, nella quale viene marcata ogni parte del
  documento utilizzando i marcatori contenuti nella DTD.
\end{enumerate}

Con l'utilizzo di questo tipo di linguaggio entriamo nell'ordine di idee
di un sistema strutturato secondo una precisa gerarchia degli oggetti
testuali; sotto questo punto di vista il testo è visto come «una
gerarchia ordinata di oggetti di contenuto» (De Rose). Ciò vuol dire che
ogni oggetto testuale ha una relazione gerarchica con gli altri e che
essi altro non sono che le strutture editoriali: a partire dal macro
contenitore del libro, scendendo via via a livelli più bassi con
capitoli, paragrafi, citazioni, note, apparati, ecc.

Questa teoria gerarchica ha i suoi limiti dati principalmente dai
diversi livelli di interpretazione che possiamo affidare al testo. Non è
possibile, infatti, creare un sistema gerarchico che sia in grado di
evidenziare, allo stesso tempo, ogni caratteristica testuale. Si è
cercato di dare una nuova interpretazione `pluralista' alla teoria
gerarchica, ma quello che qui ci preme sottolineare è l'importanza di
questa caratteristica dell'\emph{SGML} che venne come uno dei principi
fondamentali del linguaggio in \emph{Gentle Introduction of SGML} di C.
M. Sperberg-McQueen e L. Burnard. Oltre ad asserire il principio
gerarchico, gli autori parlano del suo essere `descriptive markup',
quindi un sistema di codifica che descrive la struttura logica del
documento, e della sua `data independence' che poggia, a sua volta, su
due caratteristiche: la prima è che si tratta di un metalinguaggio, la
seconda è l'uso del concetto di entità con il quale è possibile
codificare dei caratteri alfabetici diversi dallo standard inglese.

\section*{Bibliografia Consigliata}
{\parindent0pt 
DeRose S.J. - Durand D. - Milonas E. - Renear A.H., \emph{What is Text,
Really?}, in \emph{Journal of compunting in Higher Education}, 1, 2,
1990, pp. 3-26
}

\section*{Sitografia}
{\parindent0pt 
Goldfarb C.F., \emph{Design Considerations For Integrated Text
Processing System}, in \emph{IBM Cambridge }

\emph{Scientific Center -- Technical Report} \emph{No. 320-2094}, 1973,
\url{http://www.sgmlsource.com/history/G320-2094/G320-2094.html}


\emph{SGML Web Page},
\url{http://www.sil.org/sgml/sgml.html}
}

\hrulefill 
 
{[}\emph{Alessia Marini}{]}



\chapter{Tassonomia}

L'esigenza di creare un sistema di classificazione che miri
all'immediata e puntuale reperibilità delle informazioni contenute in
una pagina web, ha dato vita alla \emph{tassonomia} informatica o
informatizzata, che si occupa della classificazione degli argomenti
informatici secondo un ordine logico.

L'etimo greco del termine -- da τάξις (`ordine') e νόμος (`regola') --
designa un'idea di ordinamento che, in un sito web, segue
sostanzialmente due criteri:

\begin{itemize}
\item
  di vicinanza, qualora si uniscano due o più categorie di contenuto
  simile;
\item
  di specificità, che corrisponde all'espansione di una categoria i cui
  contenuti sono circoscritti a un solo argomento.
\end{itemize}

A questi criteri tassonomici si affiancano due tipologie di tassonomie
generali:

\begin{itemize}
\item
  \emph{tassonomia orizzontale}, che si configura come semplice elenco
  di categorie di primo livello -- in cui gli elementi elencati hanno
  pari importanza --, o come raggruppamenti tematici ottenuti mediante
  tag. Il tag coincide con una key\emph{-}words orizzontale e
  trasversale a un \emph{corpus} di contenuti che unisce diversi
  documenti, anche non pertinenti alla specifica ricerca dell'utente, ma
  affini tematicamente;
\item
  \emph{tassonomia verticale}, in cui i contenuti sono pertinenti a un
  macro-argomento di riferimento e le categorie -- che contengono a loro
  volta sottocategorie -- sono disposte gerarchicamente all'interno del
  sito in ordine di importanza.
\end{itemize}

Hanno un impianto tassonomico i vocabolari che organizzano in maniera
strutturata categorie concettuali a seconda degli ambiti di analisi, o,
in termini meno specifici, qualsiasi pagina o sito web organizzati
ordinatamente.

I vantaggi ottenuti dall'introduzione della \emph{tassonomia} consistono
nella funzionalità e nell'operabilità di un sito atte a migliorare
l'esperienza di navigazione dell'utente (User Experience o UX): la
\emph{tassonomia}, infatti, mette gli utenti nella condizione di
reperire comodamente un contenuto, consentendo al motore di ricerca di
associare il sito web a un particolare topic (Search Topic).

Data l'importanza di questa disciplina, certa è anche la sua validità
nell'ambito delle Digital Humanities: essa infatti struttura e utilizza
le informazioni importanti per le discipline umanistiche rendendole
facilmente individuabili, nonché consultabili, e dispone di un enorme
potenziale da esplicarsi soprattutto nella raccolta e nella
classificazione delle informazioni, degli strumenti, dei metodi e dei
progetti di DH necessari a facilitare la ricerca.

\section*{Bibliografia Consigliata}
{\parindent0pt 
Isone L., \emph{Strategie SEO per l'E-COMMERCE}, Milano, Hoepli, 2017,
\url{https://books.google.it/books?id=XjskDwAAQBAJ\&pg=PT99\&lpg=PT99\&dq=tassonomia+orizzontale+e+verticale\&source=bl\&ots=9GuQqPRwYg\&sig=ACfU3U0S6WNix0VGaq70l1yl21x0jk0RVQ\&hl=it\&sa=X\&ved=2ahUKEwjm3v24r5LgAhWGM-wKHUtdDJkQ6AEwCHoECAQQAQ\#v=onepage\&q=tassonomia\%20orizzontale\%20e\%20verticale\&f=false}{https://books.google.it/books?id=XjskDwAAQBAJ\&pg=PT99\&lpg=PT99\&dq=tassonomia+orizzontale+e+verticale\&source=bl\&ots=9GuQqPRwYg\&sig=ACfU3U0S6WNix0VGaq70l1yl21x0jk0RVQ\&hl=it\&sa=X\&ved=2ahUKEwjm3v24r5LgAhWGM-wKHUtdDJkQ6AEwCHoECAQQAQ\#v=onepage\&q=tassonomia\%20orizzontale\%20e\%20verticale\&f=false}
}

\section*{Sitografia}
{\parindent0pt 
AaVv, \emph{Dall'informatica umanistica alle culture digitali. In
memoria di Giuseppe Gigliozzi},
\url{http://www.editricesapienza.it/node/7688}

Minini A., \emph{Le tassonomie verticali e orizzontali},
\url{http://www.andreaminini.com/seo/la-tassonomia-del-sito-web-nella-seo}

Ruecker S., \emph{A brief taxonomy of prototypes for the Digital
Humanities},
\url{https://www.researchgate.net/publication/315502045_A_Brief_Taxonomy_of_Prototypes_for_the_Digital_Humanities}

*\emph{Tassonomia} (\emph{s.v}.), in \emph{Treccani}
\emph{Enciclopedia},
\url{http://www.treccani.it/enciclopedia/tassonomia/}

*\emph{Tassonomia} (\emph{s.v}.), in \emph{Wikipedia},
\url{https://it.wikipedia.org/wiki/Tassonomia}

Wahl Z. - Busch J., \emph{Il valore della tassonomia nella ricerca delle
informazioni},
\url{http://www.datamanager.it/rivista/searching/il-valore-della-tassonomia-nella-ricerca-delle-informazioni/999/}
}

\hrulefill 
 
{[}\emph{Alessandra Di Meglio}{]}




\chapter{TEI}

(ingl. Acronimo per \emph{Text Encoding Initiative})

È un progetto sviluppato a partire dal 1987 che si propone di conservare
le linee generali per la codifica di testi letterari e linguistici: i
suoi scopi sono stati espressi alla fine di una conferenza tenutasi al
Vassar College, N.Y., nel novembre dello stesso anno. Si tratta di una
marcatura basata sul linguaggio XML e «il suo uso è ambizioso nella sua
complessità e generalità, ma fondamentalmente non c'è differenza tra
questo e gli altri schemi di markup XML» (Burnardn, Sperberg-McQueen).
Le \emph{TEI Guidelines}, le linee guida su cui si basa il vocabolario
della codifica e che gestisce anche le regole gerarchiche, sono state
rese pubbliche nel maggio del 1994, dopo sei anni di sviluppo gestito da
centinaia di studiosi, provenienti da differenti discipline, sparsi per
il mondo. Da questa conferenza nacque anche quella semplificazione del
linguaggio, denominata \emph{TEI Lite}, che gli ideatori usarono per
fornire una dimostrazione di come lo schema di codifica \emph{TEI}
poteva essere adottato per servire le necessità dei testi e per rendere
la codifica più comprensibile ed usabile da tutti gli utenti, senza il
bisogno di una conoscenza approfondita della \emph{TEI DTD}.

Dal punto di vista pratico, la codifica si divide in due parti:

\begin{enumerate}
\def\labelenumi{\arabic{enumi}.}
\item
  un \emph{\textless{}tei header} in cui sono contenute
  tutte le informazioni relative alle notazioni bibliografiche e di
  catalogazione del testo e le specifiche della codifica;
\item
  la trascrizione del testo vero e proprio contenuto nel marcatore
  \emph{\textless{}text\textgreater{}.}
\end{enumerate}

All'interno del \emph{\textless{}text} che possiamo
identificare come il macro contenitore in cui sono contenuti
frontespizio, testo vero e proprio, apparati paratestuali, sono presenti
altri marcatori studiati appositamente per riferirsi a tutte le
componenti presenti in un'edizione, studiati sempre per garantire
organizzazione e gerarchia del documento di codifica. Troviamo, quindi,
in successione:

\begin{enumerate}
\def\labelenumi{\arabic{enumi}.}
\item
  \emph{\textless{}front} che contiene tutto il materiale
  paratestuale, come intestazioni, frontespizio, prefazioni, dediche,
  ecc., che si trova prima del testo vero e proprio;
\item
  \emph{\textless{}group} mette insieme tutti i testi
  appartenenti ad un determinato gruppo. È utile, ad esempio, quando ci
  si trova a dover codificare un'opera omnia di un autore che può,
  perciò, contenere testi in prosa, in versi o di saggistica;
\item
  \emph{\textless{}body} è riferito ad un testo unitario
  ad esclusione di tutte le aggiunte paratestuali;
\item
  \emph{\textless{}back} raggruppa invece tutti quei
  materiali che sono successivi al testo. Parliamo, in questo caso, di
  note, appendici, bibliografie, ecc.
\end{enumerate}

Tra questi sotto contenitori del \emph{\textless{}text}
ciò che ci interessa maggiormente è quello del
\emph{\textless{}body} nel quale si concentra il grosso
degli sforzi del codificatore, poiché il suo obbiettivo primario è
quello di rendere fruibile il testo per l'utente.

L'utilizzo delle \textless{}\emph{div}\textgreater{} per organizzare le
porzioni di testo è molto comune in tutti i progetti che utilizzano la
codifica \emph{TEI}, in quanto è uno degli elementi che, a loro volta,
possono essere divisi ed identificati da attributi globali inseriti
nella DTD:

\begin{itemize}
\item
  type: sta ad indicare letteralmente il tipo di categoria a cui
  appartiene la \textless{}\emph{div}\textgreater{}, che sia riferita ad
  un capitolo, ad un paragrafo, ad un sonetto, e via discorrendo;
\item
  id: specifica un unico identificatore per la divisione, che potrebbe
  essere usato per inserire del `cross references' o altri link;
\item
  n: specifica il numero della divisione, che può essere usato per
  identificare una preferenza gerarchica o la successione delle parti.
\end{itemize}

Essendo \emph{TEI} basato su XML è molto importante ricordarsi di
chiudere ogni marcatore: la mancata chiusura determinerebbe, infatti, un
errore nel documento e, conseguentemente, l'impossibilità di una
corretta visualizzazione del testo.

Oltre a tutti i marcatori specifici di ogni testo, gli ideatori della
\emph{TEI Lite} si sono anche dedicati alla creazione di elementi per
mettere in evidenza porzioni di testo:

\begin{itemize}
\item
  \textbf{\textless{}}\emph{emph}\textbf{} usato per
  espressioni che vengono messe in qualche modo in risalto.
\item
  \textbf{\textless{}}\emph{foreign}\textbf{} atto
  all'identificazione di parole o frasi in una lingua diversa da quella
  dell'intero testo.
\item
  \textbf{\textless{}}\emph{mentioned}\textbf{} indica una
  menzione o una citazione
\item
  \textbf{\textless{}}\emph{title}\textbf{} la quale
  contiene il titolo di un'opera, sia essa articolo, libro, giornale, o
  collana, compreso ogni titolo alternativo o sottotitolo. Viene spesso
  completato da degli attributi come `level' che indica se il titolo è
  di un articolo, libro, giornale, collana o materiale inedito, o il già
  citato type\textbf{.}
\end{itemize}

Il \emph{TEI header}, invece, è una sezione iniziale, posizionata
esattamente tra la dichiarazione XML e il tag
\emph{\textless{}text} e contiene informazioni di diverso
genere:

\begin{itemize}
\item
  \emph{\textless{}fileDesc} contiene una descrizione
  bibliografica dell'opera (titolo, autore, data e luogo di stampa,
  numero di pagine, ecc.). Al suo interno troviamo altri marcatori più
  specifici quali:
\end{itemize}

\emph{\textless{}titleStmt} in cui troviamo i riferimenti
a titolo, autore e responsabile della codifica.

\emph{\textless{}editionStmt} è il gruppo di informazioni
riguardanti l'edizione a testo, nelle quali troviamo anche la
descrizione delle particolarità dell'edizione stessa.

\emph{\textless{}extent} descrive quanto pesa il file in
\emph{bytes}.

\emph{\textless{}publicationStmt} è una sezione
obbligatoria, in essa sono contenute informazioni circa l'organizzazione
responsabile della pubblicazione, l'agenzia responsabile della
distribuzione e il responsabile della messa online del documento
codificato. Contiene, inoltre, una serie di informazioni relative alla
pubblicazione stessa (anno, luogo, identificazione ISBN) che, però, sono
completamente facoltativi.

\emph{\textless{}seriesStmt} è un elemento non
obbligatorio circa l'inclusione o meno del volume in una serie.

\emph{\textless{}noteStmt} contengono vari elementi
\emph{\textless{}note} in cui vengono specificate le varie
annotazioni al testo.

\emph{\textless{}sourceDesc} è un elemento obbligatorio
che registra tutte le fonti da cui il file digitale è derivato.

\begin{itemize}
\item
  \emph{\textless{}encoudingDesc} documenta la relazione
  tra il testo elettronico e la fonte o le fonti da cui è derivato.
\end{itemize}

\emph{\textless{}projectDesc} descrive nei dettagli gli
scopi e i propositi che si propone il file codificato, insieme ad altre
informazioni sul processo stesso.

\emph{\textless{}semplingDecl} contiene una spiegazione su
metodi di codifica.

\emph{\textless{}editorialDecl} descrive i principi
editoriali e le pratiche utilizzate durante la codifica.

\emph{\textless{}tagsDecl} contiene informazioni circa la
marcatura applicata ad un documento SGML.

\emph{\textless{}refsDecl} riassume i riferimenti usati
durante la codifica.

\emph{\textless{}classDecl} contiene una o più tassonomie
che definiscono ciascun codice di classificazione usato nel testo.

\begin{itemize}
\item
  \emph{\textless{}profileDesc} sottolinea gli aspetti non
  bibliografici, facendo quindi riferimento alla lingua e alle
  sottolingue utilizzate, a come è stato prodotto, ai partecipanti alla
  codifica e alle loro impostazioni.
\end{itemize}

\emph{\textless{}creation} informa su come è stato creato
il testo.

\emph{\textless{}laugUsage} regista le lingue, le
sottolingue, i dialetti, ecc presentati all'interno del testo.

\emph{\textless{}textClass} è una serie di informazioni
che descrivono la natura o l'argomento del testo.

\section*{Sitografia}
{\parindent0pt 
Burnard L. - Sperberg-McQueen C. M\emph{., TEI Lite: an intorduction to
Text Encodign for Interchange}, 1995 (revisited May 2002),
\url{https://www.tei-c.org/Vault/P4/Lite/teiu5_en.pdf}

\emph{TEI: Guidelines,} in \emph{Text Encoding Initiative},
\url{http://www.tei-c.org/Guidelines/}
}

\hrulefill 
 
{[}\emph{Alessia Marini}{]}

\chapter{Thesaurus}

Il termine \emph{thesaurus} è -- per così dire -- semanticamente figlio
dei tempi nei quali è stato utilizzato. Nel periodo medievale, ad
esempio, il termine è utilizzato per indicare repertori scientifici,
opere di carattere enciclopedico e/o divulgativo (\emph{e. g. Thesaurus
pauperum}, ricettario medico \emph{a capite ad pedes} del XIII secolo;
il \emph{Trésor} di Brunetto Latini \emph{et alia}). Successivamente, il
termine è utilizzato in riferimento a vocabolarî, soprattutto delle
lingue classiche: \emph{Thesaurus Linguae Graecae} di Henri Estienne
(1572), \emph{Thesaurus Linguae Latinae} di Robert Estienne (1532),
\emph{Thesaurus Linguae Latinae Epigraphicae} di G. N. Olcott (1904).

Oggi il termine ha un campo semantico più ampio e più generico; l'antico
concetto di tesaurizzazione lessicale, semantica, concettuale ed
enciclopedica s'intreccia con la dimensione del Web, sicché il termine
\emph{thesaurus} oggi viene utilizzato per indicare un database
terminologico relativo ad un determinato campo del sapere. Nella
definizione moderna assume particolare rilevanza anche l'utente e la sua
necessità di reperire informazioni `strutturate'. Pertanto, più
precisamente, si potrebbe affermare che il \emph{thesaurus}, nella sua
accezione moderna, sia sostanzialmente concepito come uno strumento
gnoseologico tassonomico, funzionale alla indicizzazione di documenti
(afferenti ad uno specifico settore o campo semantico) ed atto a
semplificare e rendere più efficaci le ricerche condotte dagli utenti.
Infatti, la Broughton (2006a, 4) definisce il \emph{thesaurus} come «a
tool used for the subject indexing of documents. It consists of terms
(usually in one particular subject field) than an indexer or records
manager may use to describe documents so that end-users can retrieve
relevant items when searching for material about a particular subject».
A tale definizione è possibile accostare quella elaborata più
recentemente dalla \emph{International Organization for
Standardization}, che descrive il \emph{thesaurus} come un «controlled
and structured vocabulary in which concepts are represented by terms,
organized so that relationship between concepts are made explicit, and
preferred terms are accompanied by lead-in entries for synonyms or
quasi-synonyms» (ISO 2011; cfr. ISO 2013).

L'uso di \emph{thesauri online} va sempre più accrescendosi e ciò è
legato ad una serie di concomitanti motivazioni, quali l'ingente mole di
informazioni disponibili online, la `migrazione' sul Web di molte
informazioni in precedenza fruibili solo su supporto cartaceo, la
necessità di garantire un certo livello qualitativo delle informazioni
presenti sul Web e la necessità di offrire agli utenti un sapere
`strutturato' (cfr. Shiri-Revie 2000, 273-274). In sintesi, i
\emph{thesauri} si configurano come un `bi-functional tool' e rivelano
la propria utilità sia nell'organizzazione che nel recupero delle
informazioni (cfr. Feldvari 2009). I \emph{thesauri} hanno, inoltre,
importanza trasversale, giacché potenzialmente l'uso di un
\emph{thesaurus} può rivelarsi produttivo in ogni campo del sapere, come
è evidente dalla recente definizione proposta da Ryan (2014, 6):
«\emph{thesauri} are vital and valuable tools in content discovery, and
in information organization and retrieval, activities common to all
fileds, including cultural heritage and higher education as well as
business and enterprise. \emph{Thesauri} allow information professionals
to represent content in a consistent manner and enable researchers,
employees and the public to find this content easily and quickly».

Davies (1996, 38-39) ha proposto una prima classificazione dei
\emph{thesauri online}, suddividendoli -- per quanto attiene al formato
e al processo di pubblicazione -- in `statici' e `dinamici'. Shiri-Revie
(2000, 274) hanno proposto una classificazione più articolata: «1.
thesauri in simple static formats (ASFA Thesaurus); 2. thesauri in HTML
format but still static, without effective use of hyperlinks
(Infoterms); 3. thesauri in dynamic HTML format with fully navigable
hyperlinks MeSH); 4. thesauri with advanced visual and graphical
interfaces (Plumb Design Visual Thesaurus); 5. thesauri in XML format
(Virtual HyperGlossary)». Per una accurata disamina quantitativa dei
principali \emph{thesauri} liberamente interrogabili in varie lingue
europee, cfr. Mochó Bezares-Sorli Rojo (2010, 643-663).

\section*{Bibliografia Consigliata}
{\parindent0pt 
Aitchinson J. - Gilchrist A. - Bawden D., \emph{Thesaurus construction
and use: a practical manual}, London, Fitzroy Dearden, 2005

Broughton V., \emph{Essential Thesaurus Construction}, London, Facet
Publishing, 2006

Broughton, V., \emph{The need for a faceted classification as the basis
of all methods of information retrieval}, in \emph{Aslib Proceedings:
New Information Perspectives}, 58, 1/2, 2006, pp. 49-72,
\url{https://www.researchgate.net/publication/32895215_The_need_for_a_FC_as_the_basis_of_all_methods_of_Information_retrieval}

Davies R., \emph{Publishing thesauri on the World Wide Web}, in
\emph{Proceedings of the 7th ASIS SIG/CR Classification Research
Workshop} (1996), pp. 37-48,
\url{http://journals.lib.washington.edu/index.php/acro/article/view/12688/11192}

Feldvari K., \emph{Thesauri usage in Information Retrieval Systems:
Example of LISTA and ERIC Database Thesaurus}, in \emph{INFuture
2009-Digital Resources and Knowledge Sharing}, 2009, pp. 279-288
\url{https://infoz.ffzg.hr/infuture/2009/papers/4-09\%20Feldvari,\%20Thesauri\%20usage\%20in\%20information\%20retrieval\%20systems.pdf}

International Organization for Standardization, information and
documentation, \emph{Thesauri and interoperability with other
vocabularies. Part 1: thesauri for information retrieval}, Geneva, ISO,
2011

International Organization for Standardization, information and
documentation, \emph{Thesauri and interoperability with other
vocabularies. Part 2: interoperability with other vocabularies}, Geneva,
ISO, 2013

Mochón Bezares G. - Sorli Rojo A., \emph{Thesauros en acceso abierto en
Internet. Un análisis cuantitativo}, in \emph{Revista Española de
Documentación Cientifica}, 33, 4, 2010, pp. 643-663,
\url{http://redc.revistas.csic.es/index.php/redc/article/view/675/750}

Ryan C., \emph{Thesauri Construction Guidelines: An Introduction to
thesauri and guidelines on their construction}, Dublin 2014,
\url{http://apps.dri.ie/motif/docs/guidelines.pdf}

Shiri A. A. - Revie C., Thesauri \emph{on the web: current developments
and trends}, in \emph{Online Information Review} 24, 4, 2000, pp.
273-279,
\url{https://strathprints.strath.ac.uk/1896/1/strathprints001896.pdf}

}

\section*{Sitografia}
{\parindent0pt 
*\emph{Quick Guide to publishing a Thesaurus on the Semantic Web},
\url{https://www.w3.org/TR/2005/WD-swbp-thesaurus-pubguide-20050517/}

* \emph{Thesaurus} (\emph{s. v.})\emph{,} in \emph{Treccani},
\url{http://www.treccani.it/enciclopedia/thesaurus/}

* \emph{Thesaurus} \emph{interrogabili} (\emph{s. v.})\emph{,} in
\emph{Treccani},
\url{http://www.treccani.it/vocabolario/thesaurus/}

*\emph{Thesaurus Linguae Grecae} (online),
\url{http://stephanus.tlg.uci.edu/}

*\emph{Thesaurus Linguae Latinae} (online),
\url{https://www.degruyter.com/view/db/tll}

*\emph{Thesaurus Linguae Latinae Epigraphicae} (online, scaricabile), I,
\url{https://archive.org/details/ThesaurusLinguaeLatinaeEpigraphicae}

*\emph{Thesaurus Musicarum Latinarum} (online),
\url{http://boethius.music.indiana.edu/tml/}
}

\hrulefill 
 
{[}\emph{Giovanna Battaglino}{]}




\chapter{Web Semantico}

Il \emph{Web Semantico}, detto anche `Web dei dati' è l'insieme dei
servizi e delle strutture capaci di interpretare il significato dei
contenuti del Web. In buona sostanza si tratta di un'estensione del Web
che implica un nuovo modo di concepirne i documenti, in cui le
informazioni assumono un ruolo ben preciso e in cui computer e utenti
lavorano in cooperazione, secondo le intenzioni di Tim Berners-Lee che
l'ha ipotizzata nel 2001.

Il Web attuale è un insieme di testi collegati tra loro da link, in cui
i soli utenti umani sono in grado di leggere e comprendere i contenuti
delle pagine che stanno visitando, grazie alla loro esperienza di
navigazione e alla capacità di interpretazione.

Da un punto di vista strutturale il \emph{Web Semantico} è stato
rappresentato da Berners-Lee come una piramide di sette strati, composta
da nove elementi. Tale modello architettonico sorregge tre tipologie di
informazioni: documenti autodescrittivi, dati e regole. Il significato
della nuova architettura va seguito dal basso verso l'alto: il Web può
essere concepito come un insieme di strati con standard, linguaggi o
protocolli che agiscono come piattaforme sulle quali possano poggiarsi
formalismi nuovi, più ricchi e più espressivi.

Aspetti fondamentali delle innovazioni tecnologiche che sottendono
all'architettura del \emph{Web Semantico} sono rappresentati da
metadati; identificatori di dati (URI); ontologie, reti di fiducia.

La presenza di sistemi rigorosi e standardizzati di metadati nei
contenuti delle pagine e di correlazioni tra essi, invece, consentirebbe
l'uso di automi in grado di comprendere il significato dei testi
presenti sulla rete e di guidare l'utente direttamente verso
l'informazione cercata, oppure di svolgere compiti per suo conto,
rispondendo così in maniera efficace anche ai problemi posti dalla
crescita esponenziale della quantità di informazione disponibile in rete
e dal moltiplicarsi delle sue tipologie. La semantica dei dati consiste
nel rappresentare il modello di uno specifico dominio di conoscenza
codificando le informazioni mediante ontologie, con la descrizione
formale dei concetti articolata per classi, relazioni e regole, in modo
che la macchina sia in grado d'interpretare le informazioni e di
utilizzarle correttamente.

Per definire una fondamentale caratteristica del \emph{Web Semantico} si
fa riferimento agli open data e in particolare alla disponibilità dei
dati e alla conseguente possibilità di identificarli e citarli. Il
\emph{Web Semantico} è quindi come già indicato un'estensione del Web
tradizionale, nel senso che rappresenta il successivo passaggio del
linking ed è pensato per funzionare nel contesto di un modello
relazionale di dati, in cui il link - da collegamento generico e cieco
tra due documenti - diviene capace di esprimere relazioni concettuali
che convogliano significati.

La prospettiva di un \emph{Web Semantico} realmente sviluppato non è
ancora matura, ma non vi è dubbio che la tendenza alla standardizzazione
e all'interoperabilità degli strumenti e dei parametri descrittivi abbia
ricevuto, con la crescita di Internet, un impulso notevole. I primi
passi in questa direzione sono stati compiuti garantendo
l'interoperabilità dei cataloghi ad accesso pubblico dei sistemi
bibliotecari (OPAC,~\emph{On-line public access catalogue}).

Chiaramente l'aspirazione della rete semantica non è quella di
rappresentare tutti i dati o il sapere in qualche ristretto insieme di
formalismi, ma ottenere che la possibilità di linkare i dati a nuovi
dati permetta di usarli in maniera sempre più ampia, confidando
nell'intelligenza distribuita e interconnessa dei navigatori.

\section*{Bibliografia Consigliata}
{\parindent0pt 
Berners-Lee T., \emph{L'architettura del nuovo Web. Dall'inventore della
rete il progetto di una comunicazione democratica, interattiva e
intercreativa}, Milano, Feltrinelli, 2001

Di Donato F., \emph{La scienza e la rete. L'uso pubblico della ragione
nell'età del web}, Firenze, Firenze University Press, 2009
}

\section*{Sitografia}
{\parindent0pt 
\emph{*Web semantico} (\emph{s.v}.), in \emph{Treccani} - \emph{Lessico
del XXI secolo},
\url{http://www.treccani.it/enciclopedia/web-semantico_\%28Lessico-del-XXI-Secolo\%29/}

}

\hrulefill 

{[}\emph{Concetta Damiani}{]}




\chapter{XSL}

(ingl. Acronimo per \emph{Exstensible Stylesheet Language})

Si occupa di definire, all'interno di un document XML, quale sarà la sua
visualizzazione sul browser, in quanto è in grado di contenere le
informazioni su formattazione e visualizzazione del documento, filtrare
i dati, riorganizzarli nelle gerarchie ed eseguire calcoli. Volgarmente
definito come foglio di stile, l'\emph{XSL} è più che altro un insieme
di linguaggi atti a rendere possibile la trasformazione del documento,
in modo da renderlo leggibile e visualizzabile sui browser. Tra tutti i
linguaggi, quello maggiormente interessante è l'\emph{XSLT}
(\emph{Extensible Stylesheet Languages for Transfomation}), raccomandato
ufficialmente dal W3C nel novembre del 1999.

La sua funzione primaria è quella di trasformare letteralmente il
documento XML in HTML e, per farlo, utilizza un linguaggio basato su
regole di `pattern matching', con le quali si danno delle specifiche di
visualizzazione a delle porzioni di testo, in modo tale che ogni volta
che quella determinata parte, o una simile, viene rinvenuta nel
documento, verrà immediatamente sostituita con l'impostazione data dalle
regole di visualizzazione.

Un documento \emph{XSLT} si compone di tre fasi fondamentali: nella
prima \emph{\textless{}xsl: output} è il marcatore che
specifica il tipo di formato in cui verrà visualizzato il documento
finale, ad es. HTML, mentre \emph{\textless{}xsl:template}
specifica quali parti del documento XML dovranno essere trasformate.
Nella seconda fase si ha una trasformazione strutturale dei dati: si
passa quindi dalla struttura dell'input a quella dell'output desiderato.
Nella terza ed ultima fase avviene la formattazione dei dati secondo le
specifiche dell'output selezionato. In parole più semplici il compito
dell'\emph{XSLT} è quello di prendere le informazioni da un documento di
partenza, trasformarle secondo il linguaggio del documento finale e poi
formattare questa nuova gerarchia di informazioni secondo le specifiche
del foglio di stile dell'output (si passerà, quindi, da un file.xml ad
un file.html che verrà di visualizzato ed organizzato con CSS).

\section*{Bibliografia Consigliata}
{\parindent0pt 
Ausiello G. \emph{et al}., \emph{Modelli e linguaggi dell'informatica},
Milano, McGraw-Hill Education, 1991

Ciotti F., \emph{Il testo e l'automa. Saggi di teoria e critica
computazionale dei testi letterari}, Roma, ARACNE editrice s.r.l., 2007

Moller A. - Schwartzbach M.I. - Gaburri S. (a cura di),
\emph{Introduzione a XML}, Milano, Pearson, 2007

Tissoni F., \emph{Lineamenti di editoria multimediale}, Milano, Edizioni
Unicopli, 2009
}

\section*{Sitografia}
{\parindent0pt 
*\emph{Learn W3C} (\emph{s.v}.), in \emph{w3schools.com},
\url{http://www.w3schools.com/}
}

\hrulefill 
 
{[}\emph{Alessia Marini}{]}




\chapter{XML}

(ingl. Acronimo per \emph{eXstensible Markup Language})

Assieme ad HTML (\emph{HiperText Markup Language}), è un linguaggio di
marcatura derivato da SGML (\emph{Standard Generalized Markup
Language}). Esattamente come il suo genitore è di tipo descrittivo, ciò
vuol dire che tutti i marcatori sono creati e pensati per descrivere la
funzione delle porzioni di testo che racchiudono.

Il progetto per lo sviluppo del \emph{XML} ha inizio nel 1996 con il
fine di ampliare e potenziare la capacità di gestione ed elaborazione
dei documenti sul Web; per questo venne creato un gruppo di lavoro
\emph{XML Working Group}, composto dai maggiori esperti mondiali di
SGML, che nel 1998 completarono lo sviluppo del linguaggio di marcatura.
Con la successiva sottoscrizione di Michael Sunshine, allora presidente
del W3C (\emph{Web Consortium}), il linguaggio divenne uno standard
internazionale.

\emph{XML} può essere pensato come un sottoinsieme di SGML da cui
eredita sia la sintassi sia la logica di funzionamento. Al contrario
dell'HTML, \emph{XML} non ha dei marcatori fissi e determinati, ma, come
ci suggerisce il nome, è estensibile e modificabile. Tutto ciò è
possibile grazie all'uso di una DTD (\emph{Document Type Description}):
è importante avere ben chiaro, prima di iniziare ad avventurarsi nella
pratica di questo linguaggio, che non può esistere un file \emph{XML}
privo di DTD in quanto essa ne rappresenta implicitamente le fondamenta.

Un'altra differenza fondamentale con l'HTML è che esso si preoccupa più
di descrivere i particolari fisici del~documento (utilizzando
\emph{\textless{}i\textgreater{} \textless{}/i} per
segnalare un titolo in corsivo), l'\emph{XML} invece preferisce un tipo
di descrizione degli aspetti logici, così un titolo in corsivo viene
marcato come \emph{\textless{}title} per differenziarlo
logicamente, e non graficamente, dagli altri oggetti testuali. Il
linguaggio di marcatura, quindi, non si interessa tanto di espletare il
significato del singolo elemento, quanto di definire quale sia la sua
relazione con gli altri.

Altra peculiarità dell'\emph{XML} è il foglio di stile necessario per la
visualizzazione sui browser: chiamato XSL (\emph{Exstensible Stylesheet
Language}), esso permette non solo l'inserimento di specifiche di
visualizzazione come i contenuti grafici ed il posizionamento dei
contenuti, ma principalmente di preoccupa rendere leggibile il file
\emph{XML} per il browser.

Indispensabile in ogni documento \emph{XML} è il prologo nel quale è
contenuta la dichiarazione della versione del linguaggio utilizzata e
specifica il set di caratteri usati dalla codifica caratteri.

All'interno della codifica è anche possibile utilizzare dei commenti
utili nel caso in cui un certo documento subisca più lavorazioni da
parte di mani diverse: l'uso dei commenti serve al codificatore per
segnalare delle particolarità del documento o per suggerire metodi di
lavorazione.

In ultimo va citato l'uso delle entità che aiuta a codificare tutte
quelle parti, non necessariamente testuali, che non possono essere
piegate alla logica gerarchica del documento. Possono essere, quindi, di
due tipi, quelle generali, già normalmente inserite nella DTD
(delimitate da \& e ;), e quelle parametriche, più specifiche, che
vengono usate solo nelle dichiarazioni di markup (delimitate da \% e ;).

\section*{Bibliografia Consigliata}
{\parindent0pt 
Ausiello G. \emph{et al}., \emph{Modelli e linguaggi dell'informatica},
Milano, McGraw-Hill Education, 1991

Ciotti F., \emph{Il testo e l'automa. Saggi di teoria e critica
computazionale dei testi letterari}, Roma, ARACNE editrice s.r.l., 2007

Moller A. - Schwartzbach M.I. - Gaburri S. (a cura di),
\emph{Introduzione a XML}, Milano, Pearson, 2007

Tissoni F., \emph{Lineamenti di editoria multimediale}, Milano, Edizioni
Unicopli, 2009

\section*{Sitografia}

Sorato A., \emph{Linguaggi per la rete: XML},
\url{http://www.dsi.unive.it/~asorato/SlideXML/DTD.pdf}

*\emph{Learn W3C} (\emph{s.v}.), in \emph{w3schools.com},
\url{http://www.w3schools.com/}
}

\hrulefill 

{[}\emph{Alessia Marini}{]}

